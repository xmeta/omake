\batchmode
\documentclass[dvipdfm]{book}

\usepackage{hevea}
\usepackage{code}
\usepackage{xspace}
%\usepackage[latin1]{inputenc}
\usepackage[english]{babel}
\usepackage{url}
\usepackage{index}
\usepackage{hyperref}

\addtocounter{tocdepth}{1}
\addtocounter{secnumdepth}{1}

\newcommand{\OMake}{OMake\xspace}

\newcommand{\authors}{Jason Hickey, Aleksey Nogin, \emph{et. al.}}

\title{The \OMake{} user guide and reference manual}
\author{\authors}

%
% Backwards-compatibility with the LaTeX2Man package
%
\newcommand{\Prog}[1]{\texttt{#1}}
\newcommand{\File}[1]{\texttt{#1}}
\newcommand{\Cmd}[2]{\texttt{#1}(#2)}
\newcommand{\Arg}[1]{\texttt{#1}}
\newcommand{\Opt}[1]{\texttt{#1}}
\newcommand{\OptArg}[2]{\texttt{#1 <#2>}}
\newcommand{\oOpt}[1]{[\texttt{#1}]}
\newcommand{\oArg}[1]{\texttt{#1}}
\newcommand{\oOptArg}[2]{[\texttt{#1 <#2>}]}

\newcommand\Version{0.9.8.4}
\newcommand\Date{\today}
\date{\Date}

%
% Identifiers.
%
\newcommand\idsection[1]{\subsection{#1}}
\newcommand\varlabel[2]{\paragraph{#2}\hypertarget{var:#1}{}\index{#2}\index[var]{#2}}
\newcommand\var[1]{\varlabel{#1}{#1}}
\newcommand\funref[1]{\hypertarget{fun:#1}{}\index{#1}\index[fun]{#1}}
\newcommand\fun[1]{\idsection{#1}\funref{#1}}
\newcommand\funlabel[2]{\idsection{#2}\hypertarget{fun:#1}{}\index{#2}\index[fun]{#2}}
\newcommand\twofuns[2]{\idsection{{#1}, {#2}}\funref{#1}\funref{#2}}
\newcommand\twofunslabel[4]{\idsection{{#2}, {#4}}\hypertarget{fun:#1}{}\index{#2}\index[fun]{#2}\hypertarget{fun:#3}{}\index{#4}\index[fun]{#4}}
\newcommand\threefuns[3]{\idsection{{#1}, {#2}, {#3}}\funref{#1}\funref{#2}\funref{#3}}
\newcommand\threefunslabel[6]{\idsection{{#2}, {#4}, {#6}}\hypertarget{fun:#1}{}\index{#2}\index[fun]{#2}\hypertarget{fun:#3}{}\index{#4}\index[fun]{#4}\hypertarget{fun:#5}{}\index{#6}\index[fun]{#6}}
\newcommand\sixfuns[6]{\idsection{{#1}, {#2}, {#3}, {#4}, {#5}, {#6}}\funref{#1}\funref{#2}\funref{#3}\funref{#4}\funref{#5}\funref{#6}}
\newcommand\formref[1]{\funref{#1}}
\newcommand\form[1]{\idsection{#1}\funref{#1}}
\newcommand\objref[1]{\hypertarget{obj:#1}{}\index{#1}\index[obj]{#1}}
\newcommand\obj[1]{\idsection{#1}\objref{#1}}
\newcommand\targetlabelref[2]{\index{#2}\index[target]{#2}\label{target:#1}}
\newcommand\targetref[1]{\index{#1}\index[target]{#1}\label{target:#1}}
\newcommand\target[1]{\idsection{#1}\targetref{#1}}
\newcommand\optionref[1]{\index{#1}\index[opt]{#1}\label{option:#1}}
\newcommand\option[1]{\idsection{\texttt{#1}}\optionref{#1}}

\newstyle{.verbatim}{margin-left:3ex; background:\#ddeeff;}
\newstyle{CODE}{padding-left:0.2ex; padding-right:0.2ex; background:\#eef7ff;}

\newcommand\hyperfun[1]{\hyperlink{fun:#1}{\code{#1} function}}
\newcommand\hyperfunn[1]{\hyperlink{fun:#1}{\code{#1}}}
\newcommand\hyperfunx[2]{\hyperlink{fun:#1}{\code{#2} function}}
\newcommand\hyperfunxn[2]{\hyperlink{fun:#1}{\code{#2}}}
\newcommand\hypervar[1]{\hyperlink{var:#1}{\code{#1} variable}}
\newcommand\hypervarn[1]{\hyperlink{var:#1}{\code{#1}}}
\newcommand\hypervarx[2]{\hyperlink{var:#1}{\code{#2} variable}}
\newcommand\hypervarxn[2]{\hyperlink{var:#1}{\code{#2}}}
\newcommand\hyperobj[1]{\hyperlink{obj:#1}{\code{#1} object}}
\newcommand\hypertargn[1]{\hyperlink{target:#1}{\code{#1}}}

\newindex{default}{idx}{ind}{Index}
\newindex{var}{vidx}{vind}{Index of variables}
\newindex{fun}{fids}{find}{Index of functions and special forms}
\newindex{obj}{oids}{oind}{Index of objects}
\newindex{opt}{opts}{optind}{Index of options}
\newindex{target}{tids}{tind}{Index of special targets}

\newcommand{\itemidx}[1]{\item{\code{#1}}\index{#1}}

\newcommand\bul{\begin{rawhtml}&bull;&nbsp;\end{rawhtml}}
\newcommand\heading{%
\begin{htmlonly}
\begin{tabular}{@{}l@{~~}l@{}}
Jump to:&\ahref{http://omake.metaprl.org/}{\OMake{} Home}
\bul\ahref{omake.html}{Guide Home}
\bul\ahref{omake-doc.html}{Guide (single-page)}
\bul\ahref{omake-toc.html}{Contents (short)}
\bul\ahref{omake-contents.html}{Contents (long)}\\
Index:&\ahref{omake-all-index.html}{All}
\bul\ahref{omake-var-index.html}{Variables}
\bul\ahref{omake-fun-index.html}{Functions}
\bul\ahref{omake-obj-index.html}{Objects}
\bul\ahref{omake-target-index.html}{Targets}
\bul\ahref{omake-option-index.html}{Options}
\end{tabular}
\end{htmlonly}%
}
\htmlhead{%
\begin{rawhtml}
<img src="images/omake-manual.gif" border="0" align="top" alt=""><br>
\end{rawhtml}
\heading}
\htmlfoot{\heading}

\begin{document}

\htmlprefix{\OMake{} manual: }
\maketitle

\begin{htmlonly}
\ahref{omake-doc.html}{All the documentation on a single page}

\textbf{\OMake{} table of contents}
\end{htmlonly}

\tableofcontents
\label{chapter:contents}
\cutname{omake-contents.html}

%%%%%%%%%%%%%%%%%%%%%%%%%%%%%%%%%%%%%%%%%%%%%%%%%%%%%%%%%%%%%%%%%%%%%%%%
% Short description
%
\section{Description}

\Prog{omake} is designed for building projects that might have source files in several directories.
Projects are normally specified using an \File{OMakefile} in each of the project directories, and an
\File{OMakeroot} file in the root directory of the project.  The \File{OMakeroot} file specifies
general build rules, and the \File{OMakefile}s specify the build parameters specific to each of the
subdirectories.  When \Prog{omake} runs, it walks the configuration tree, evaluating rules from all
of the \File{OMakefile}s.  The project is then built from the entire collection of build rules.

\subsection{Automatic dependency analysis}

Dependency analysis has always been problematic with the \Cmd{make}{1} program.  \Prog{omake}
addresses this by adding the \verb+.SCANNER+ target, which specifies a command to produce
dependencies.  For example, the following rule

\begin{verbatim}
    .SCANNER: %.o: %.c
        $(CC) $(INCLUDE) -MM $<
\end{verbatim}

is the standard way to generate dependencies for \verb+.c+ files.  \Prog{omake} will automatically
run the scanner when it needs to determine dependencies for a file.

\subsection{Content-based dependency analysis}

Dependency analysis in omake uses MD5 digests to determine whether files have changed.  After each
run, \Prog{omake} stores the dependency information in a file called \File{.omakedb} in the project
root directory.  When a rule is considered for execution, the command is not executed if the target,
dependencies, and command sequence are unchanged since the last run of \Prog{omake}.  As an
optimization, \Prog{omake} does not recompute the digest for a file that has an unchanged
modification time, size, and inode number.

%%%%%%%%%%%%%%%%%%%%%%%%%%%%%%%%%%%%%%%%%%%%%%%%%%%%%%%%%%%%%%%%%%%%%%%%
% References
%
See the following manual pages for more information.

\textbf{Guide} If you are new to OMake, you the \href{omake-quickstart.html}{omake-quickstart}
presents a short introduction that describes how to set up a project.  The
\href{omake-build-examples.html}{omake-build-examples} gives larger examples of build projects, and
\href{omake-language-examples.html}{omake-language-examples} presents programming examples.

\begin{description}
\item[\href{omake-quickstart.html}{omake-quickstart}]
%
   A quickstart guide to using \Prog{omake}.
\item[\href{omake-build-examples.html}{omake-build-examples}]
%
   Advanced build examples.
\item[\href{omake-language-examples.html}{omake-language-examples}]
%
   Advanced language examples.
\item[\textbf{Reference}]
\begin{description}
\item[\href{omake-language.html}{omake-language}]
%
   The \Prog{omake} language, including a description of objects, expressions, and values.
\item[\href{omake-root.html}{omake-root}]
%
   The system \File{OMakeroot} contains the default specification of how to build C, OCaml, and
   \LaTeX\ programs.
\item[\href{omake-shell.html}{omake-shell}]
%
   Using the \Prog{omake} shell for command-line interpretation.
\item[\href{omake-rules.html}{omake-rules}]
%
   Using \Prog{omake} rules to build program.
\item[\href{omake-base.html}{omake-base}]
%
   Functions and variables in the core standard library.
\item[\href{omake-system.html}{omake-system}]
%
   Functions on files, input/output, and system commands.
\item[\href{omake-pervasives.html}{omake-pervasives}]
%
   Pervasives defines the built-in objects.
\item[\href{osh.html}{osh}]
%
   The \Prog{osh} command-line interpreter.
\end{description}
\item[\textbf{Appendices}]
\begin{description}
\item[\href{omake-options.html}{omake-options}]
%
   Command-line options for \Prog{omake}.
%
\item[\href{omake-grammar.html}{omake-grammar}]
%
   A more precise specification of the OMake language.
\end{description}
\item[\href{omake-doc.html}{All the documentation on a single page}]
%
   All the OMake documentation in a single page.
\end{description}

% -*-
% Local Variables:
% Mode: LaTeX
% fill-column: 100
% TeX-master: "paper"
% TeX-command-default: "LaTeX/dvips Interactive"
% End:
% -*-

%
% Soem examples.
%
\chapter{OMake quickstart guide}
\label{chapter:quickstart}
\cutname{omake-quickstart.html}

\section{Description}

\Prog{omake} is designed for building projects that might have source files in several directories.
Projects are normally specified using an \File{OMakefile} in each of the project directories, and an
\File{OMakeroot} file in the root directory of the project.  The \File{OMakeroot} file specifies
general build rules, and the \File{OMakefile}s specify the build parameters specific to each of the
subdirectories.  When \Prog{omake} runs, it walks the configuration tree, evaluating rules from all
of the \File{OMakefile}s.  The project is then built from the entire collection of build rules.

\subsection{Automatic dependency analysis}

Dependency analysis has always been problematic with the \Cmd{make}{1} program.  \Prog{omake}
addresses this by adding the \verb+.SCANNER+ target, which specifies a command to produce
dependencies.  For example, the following rule

\begin{verbatim}
    .SCANNER: %.o: %.c
        $(CC) $(INCLUDE) -MM $<
\end{verbatim}

is the standard way to generate dependencies for \verb+.c+ files.  \Prog{omake} will automatically
run the scanner when it needs to determine dependencies for a file.

\subsection{Content-based dependency analysis}

Dependency analysis in omake uses MD5 digests to determine whether files have changed.  After each
run, \Prog{omake} stores the dependency information in a file called \File{.omakedb} in the project
root directory.  When a rule is considered for execution, the command is not executed if the target,
dependencies, and command sequence are unchanged since the last run of \Prog{omake}.  As an
optimization, \Prog{omake} does not recompute the digest for a file that has an unchanged
modification time, size, and inode number.

\section{For users already familiar with make}

For users already familiar with the \Cmd{make}{1} command, here is a list of
differences to keep in mind when using \Prog{omake}.

\begin{itemize}
\item In \Prog{omake}, you are much less likely to define build rules of your own.
  The system provides many standard functions (like \verb+StaticCLibrary+,
  described in Section~\ref{fun:StaticCLibrary} and \verb+CProgram+,
  described in Section~\ref{fun:CProgram})
  to specify these builds more simply.
\item Implicit rules using \verb+.SUFFIXES+ and the \verb+.suf1.suf2:+ are not supported.
  You should use wildcard patterns instead \verb+%.suf2: %.suf1+.
\item Scoping is significant: you should define variables and \verb+.PHONY+
  targets (see Section~\ref{target:.PHONY}) before they are used.
\item Subdirectories are incorporated into a project using the
  \verb+.SUBDIRS:+ target (see Section~\ref{target:.SUBDIRS}).
\end{itemize}

\section{Building a small C program}

To start a new project, the easiest method is to change directories to the project
root and use the command \verb+omake --install+ to install default \File{OMakefile}s.

\begin{verbatim}
    $ cd ~/newproject
    $ omake --install
    *** omake: creating OMakeroot
    *** omake: creating OMakefile
    *** omake: project files OMakefile and OMakeroot have been installed
    *** omake: you should edit these files before continuing
\end{verbatim}

The default \File{OMakefile} contains sections for building C and OCaml programs.
For now, we'll build a simple C project.

Suppose we have a C file called \verb+hello_code.c+ containing the following code:

\begin{verbatim}
    #include <stdio.h>

    int main(int argc, char **argv)
    {
        printf("Hello world\n");
        return 0;
    }
\end{verbatim}

To build the program a program \verb+hello+ from this file, we can use the
\verb+CProgram+ function (Section~\ref{fun:CProgram}).
The \File{OMakefile} contains just one line that specifies that the program \verb+hello+ is
to be built from the source code in the \verb+hello_code.c+ file (note that file suffixes
are not passed to these functions).

\begin{verbatim}
    CProgram(hello, hello_code)
\end{verbatim}

Now we can run \Prog{omake} to build the project.  Note that the first time we run \Prog{omake},
it both scans the \verb+hello_code.c+ file for dependencies, and compiles it using the \verb+cc+
compiler.  The status line printed at the end indicates how many files were scanned, how many
were built, and how many MD5 digests were computed.

\begin{verbatim}
    $ omake hello
    *** omake: reading OMakefiles
    *** omake: finished reading OMakefiles (0.0 sec)
    - scan . hello_code.o
    + cc -I. -MM hello_code.c
    - build . hello_code.o
    + cc -I. -c -o hello_code.o hello_code.c
    - build . hello
    + cc -o hello hello_code.o
    *** omake: done (0.5 sec, 1/6 scans, 2/6 rules, 5/22 digests)
    $ omake
    *** omake: reading OMakefiles
    *** omake: finished reading OMakefiles (0.1 sec)
    *** omake: done (0.1 sec, 0/4 scans, 0/4 rules, 0/9 digests)
\end{verbatim}

If we want to change the compile options, we can redefine the \verb+CC+ and \verb+CFLAGS+
variables \emph{before} the \verb+CProgram+ line.  In this example, we will use the \verb+gcc+
compiler with the \verb+-g+ option.  In addition, we will specify a \verb+.DEFAULT+ target
to be built by default.  The \verb+EXE+ variable is defined to be \verb+.exe+ on \verb+Win32+
systems; it is empty otherwise.

\begin{verbatim}
    CC = gcc
    CFLAGS += -g
    CProgram(hello, hello_code)
    .DEFAULT: hello$(EXE)
\end{verbatim}

Here is the corresponding run for \Prog{omake}.

\begin{verbatim}
    $ omake
    *** omake: reading OMakefiles
    *** omake: finished reading OMakefiles (0.0 sec)
    - scan . hello_code.o
    + gcc -g -I. -MM hello_code.c
    - build . hello_code.o
    + gcc -g -I. -c -o hello_code.o hello_code.c
    - build . hello
    + gcc -g -o hello hello_code.o
    *** omake: done (0.4 sec, 1/7 scans, 2/7 rules, 3/22 digests)
\end{verbatim}

We can, of course, include multiple files in the program.  Suppose we write a new
file \verb+hello_helper.c+.  We would include this in the project as follows.

\begin{verbatim}
    CC = gcc
    CFLAGS += -g
    CProgram(hello, hello_code hello_helper)
    .DEFAULT: hello$(EXE)
\end{verbatim}

\section{Larger projects}

As the project grows it is likely that we will want to build libraries of code.
Libraries can be built using the \verb+StaticCLibrary+ function.  Here is an example
of an \File{OMakefile} with two libraries.

\begin{verbatim}
    CC = gcc
    CFLAGS += -g

    FOO_FILES = foo_a foo_b
    BAR_FILES = bar_a bar_b bar_c

    StaticCLibrary(libfoo, $(FOO_FILES))
    StaticCLibrary(libbar, $(BAR_FILES))

    # The hello program is linked with both libraries
    LIBS = libfoo libbar
    CProgram(hello, hello_code hello_helper)

    .DEFAULT: hello$(EXE)
\end{verbatim}

\section{Subdirectories}

As the project grows even further, it is a good idea to split it into several directories.
Suppose we place the \verb+libfoo+ and \verb+libbar+ into subdirectories.

In each subdirectory, we define an \File{OMakefile} for that directory.  For example, here
is an example \File{OMakefile} for the \verb+foo+ subdirectory.

\begin{verbatim}
    INCLUDES += .. ../bar

    FOO_FILES = foo_a foo_b
    StaticCLibrary(libfoo, $(FOO_FILES))
\end{verbatim}

Note the the \verb+INCLUDES+ variable is defined to include the other directories in the project.

Now, the next step is to link the subdirectories into the main project.  The project \File{OMakefile}
should be modified to include a \verb+.SUBDIRS:+ target.

\begin{verbatim}
    # Project configuration
    CC = gcc
    CFLAGS += -g

    # Subdirectories
    .SUBDIRS: foo bar

    # The libraries are now in subdirectories
    LIBS = foo/libfoo bar/libbar

    CProgram(hello, hello_code hello_helper)

    .DEFAULT: hello$(EXE)
\end{verbatim}

Note that the variables \verb+CC+ and \verb+CFLAGS+ are defined \emph{before} the \verb+.SUBDIRS+
target.  These variables remain defined in the subdirectories, so that \verb+libfoo+ and \verb+libbar+
use \verb+gcc -g+.

If the two directories are to be configured differently, we have two choices.  The \File{OMakefile}
in each subdirectory can be modified with its configuration (this is how it would normally be done).
Alternatively, we can also place the change in the root \File{OMakefile}.

\begin{verbatim}
    # Default project configuration
    CC = gcc
    CFLAGS += -g

    # libfoo uses the default configuration
    .SUBDIRS: foo

    # libbar uses the optimizing compiler
    CFLAGS += -O3
    .SUBDIRS: bar

    # Main program
    LIBS = foo/libfoo bar/libbar
    CProgram(hello, hello_code hello_helper)

    .DEFAULT: hello$(EXE)
\end{verbatim}

Note that the way we have specified it, the \verb+CFLAGS+ variable also contains the \verb+-O3+
option for the \verb+CProgram+, and \verb+hello_code.c+ and \verb+hello_helper.c+ file will both be
compiled with the \verb+-O3+ option.  If we want to make the change truly local to \verb+libbar+, we
can put the \verb+bar+ subdirectory in its own scope using the \verb+section+ form.

\begin{verbatim}
    # Default project configuration
    CC = gcc
    CFLAGS += -g

    # libfoo uses the default configuration
    .SUBDIRS: foo

    # libbar uses the optimizing compiler
    section
        CFLAGS += -O3
        .SUBDIRS: bar

    # Main program does not use the optimizing compiler
    LIBS = foo/libfoo bar/libbar
    CProgram(hello, hello_code hello_helper)

    .DEFAULT: hello$(EXE)
\end{verbatim}

Later, suppose we decide to port this project to \verb+Win32+, and we discover that we need
different compiler flags and an additional library.

\begin{verbatim}
    # Default project configuration
    if $(equal $(OSTYPE), Win32)
        CC = cl /nologo
        CFLAGS += /DWIN32 /MT
        export
    else
        CC = gcc
        CFLAGS += -g
        export

    # libfoo uses the default configuration
    .SUBDIRS: foo

    # libbar uses the optimizing compiler
    section
        CFLAGS += $(if $(equal $(OSTYPE), Win32), $(EMPTY), -O3)
        .SUBDIRS: bar

    # Default libraries
    LIBS = foo/libfoo bar/libbar

    # We need libwin32 only on Win32
    if $(equal $(OSTYPE), Win32)
       LIBS += win32/libwin32

       .SUBDIRS: win32
       export

    # Main program does not use the optimizing compiler
    CProgram(hello, hello_code hello_helper)

    .DEFAULT: hello$(EXE)
\end{verbatim}

Note the use of the \verb+export+ directives to export the variable definitions from the
if-statements.  Variables in \Prog{omake} are \emph{scoped}---variables in nested blocks (blocks
with greater indentation), are not normally defined in outer blocks.  The \verb+export+ directive
specifies that the variable definitions in the nested blocks should be exported to their parent
block.

Finally, for this example, we decide to copy all libraries into a common \verb+lib+ directory.  We
first define a directory variable, and replace occurrences of the \verb+lib+ string with the
variable.

\begin{verbatim}
    # The common lib directory
    LIB = $(dir lib)

    # phony target to build just the libraries
    .PHONY: makelibs

    # Default project configuration
    if $(equal $(OSTYPE), Win32)
        CC = cl /nologo
        CFLAGS += /DWIN32 /MT
        export
    else
        CC = gcc
        CFLAGS += -g
        export

    # libfoo uses the default configuration
    .SUBDIRS: foo

    # libbar uses the optimizing compiler
    section
        CFLAGS += $(if $(equal $(OSTYPE), Win32), $(EMPTY), -O3)
        .SUBDIRS: bar

    # Default libraries
    LIBS = $(LIB)/libfoo $(LIB)/libbar

    # We need libwin32 only on Win32
    if $(equal $(OSTYPE), Win32)
       LIBS += $(LIB)/libwin32

       .SUBDIRS: win32
       export

    # Main program does not use the optimizing compiler
    CProgram(hello, hello_code hello_helper)

    .DEFAULT: hello$(EXE)
\end{verbatim}

In each subdirectory, we modify the \File{OMakefile}s in the library directories to install them
into the \verb+$(LIB)+ directory.  Here is the relevant change to \File{foo/OMakefile}.

\begin{verbatim}
    INCLUDES += .. ../bar

    FOO_FILES = foo_a foo_b
    StaticCLibraryInstall(makelib, $(LIB), libfoo, $(FOO_FILES))
\end{verbatim}

Directory (and file names) evaluate to relative pathnames.  Within the \verb+foo+ directory, the
\verb+$(LIB)+ variable evaluates to \verb+../lib+.

As another example, instead of defining the \verb+INCLUDES+ variable separately
in each subdirectory, we can define it in the toplevel as follows.

\begin{verbatim}
    INCLUDES = $(ROOT) $(dir foo bar win32)
\end{verbatim}

In the \verb+foo+ directory, the \verb+INCLUDES+ variable will evaluate to
the string \verb+.. . ../bar ../win32+.  In the \verb+bar+ directory,
it would be \verb+.. ../foo . ../win32+.  In the root directory it
would be \verb+. foo bar win32+.

\section{Other things to consider}

\Prog{omake} also handles recursive subdirectories.  For example, suppose the \verb+foo+
directory itself contains several subdirectories.  The \File{foo/OMakefile} would then
contain its own \verb+.SUBDIRS+ target, and each of its subdirectories would
contain its own \verb+OMakefile+.

\section{Building OCaml programs}

By default, \Prog{omake} is also configured with functions for building OCaml programs.
The functions for OCaml program use the \verb+OCaml+ prefix.  For example, suppose
we reconstruct the previous example in OCaml, and we have a file called \verb+hello_code.ml+
that contains the following code.

\begin{verbatim}
   open Printf

   let () = printf "Hello world\n"
\end{verbatim}

An example \File{OMakefile} for this simple project would contain the following.

\begin{verbatim}
    # Use the byte-code compiler
    BYTE_ENABLED = true
    NATIVE_ENABLED = false
    OCAMLCFLAGS += -g

    # Build the program
    OCamlProgram(hello, hello_code)
    .DEFAULT: hello.run
\end{verbatim}

Next, suppose the we have two library subdirectories: the \verb+foo+ subdirectory
is written in C, the \verb+bar+ directory is written in OCaml, and we need to
use the standard OCaml \verb+Unix+ module.

\begin{verbatim}
    # Default project configuration
    if $(equal $(OSTYPE), Win32)
        CC = cl /nologo
        CFLAGS += /DWIN32 /MT
        export
    else
        CC = gcc
        CFLAGS += -g
        export

    # Use the byte-code compiler
    BYTE_ENABLED = true
    NATIVE_ENABLED = false
    OCAMLCFLAGS += -g

    # library subdirectories
    INCLUDES += $(dir foo bar)
    OCAMLINCLUDES += $(dir foo bar)
    .SUBDIRS: foo bar

    # C libraries
    LIBS = foo/libfoo

    # OCaml libraries
    OCAML_LIBS = bar/libbar

    # Also use the Unix module
    OCAML_OTHER_LIBS = unix

    # The main program
    OCamlProgram(hello, hello_code hello_helper)

    .DEFAULT: hello
\end{verbatim}

The \File{foo/OMakefile} would be configured as a C library.

\begin{verbatim}
    FOO_FILES = foo_a foo_b
    StaticCLibrary(libfoo, $(FOO_FILES))
\end{verbatim}

The \File{bar/OMakefile} would build an ML library.

\begin{verbatim}
   BAR_FILES = bar_a bar_b bar_c
   OCamlLibrary(libbar, $(BAR_FILES))
\end{verbatim}

\section{The OMakefile and OMakeroot files}
\label{section:omakeroot}
\index{OMakefile}
\index{OMakeroot}

\Prog{OMake} uses the \File{OMakefile} and \File{OMakeroot} files for configuring a project.  The
syntax of these files is the same, but their role is slightly different.  For one thing, every
project must have exactly one \File{OMakeroot} file in the project root directory.  This file serves
to identify the project root, and it contains code that sets up the project.  In contrast, a
multi-directory project will often have an \File{OMakefile} in each of the project subdirectories,
specifying how to build the files in that subdirectory.

Normally, the \File{OMakeroot} file is boilerplate.  The following listing is a typical example.

\begin{verbatim}
    include $(STDLIB)/build/Common
    include $(STDLIB)/build/C
    include $(STDLIB)/build/OCaml
    include $(STDLIB)/build/LaTeX

    # Redefine the command-line variables
    DefineCommandVars(.)

    # The current directory is part of the project
    .SUBDIRS: .
\end{verbatim}

The \verb+include+ lines include the standard configuration files needed for the project.  The
\verb+$(STDLIB)+ represents the \Prog{omake} library directory.  The only required configuration
file is \verb+Common+.  The others are optional; for example, the \verb+$(STDLIB)/build/OCaml+ file
is needed only when the project contains programs written in OCaml.

The \verb+DefineCommandVars+ function defines any variables specified on the command line (as
arguments of the form \verb+VAR=<value>+).  The \verb+.SUBDIRS+ line specifies that the current
directory is part of the project (so the \verb+OMakefile+ should be read).

Normally, the \verb+OMakeroot+ file should be small and project-independent.  Any project-specific
configuration should be placed in the \verb+OMakefiles+ of the project.

\section{Multiple version support}
\index{vmount}

OMake version \verb+0.9.6+ introduced preliminary support for multiple, simultaneous versions of a
project.  Versioning uses the \verb+vmount(dir1, dir2)+ function, which defines a ``virtual mount''
of directory \verb+dir1+ over directory \verb+dir2+.  A ``virtual mount'' is like a transparent
mount in Unix, where the files from \verb+dir1+ appear in the \verb+dir2+ namespace, but new files
are created in \verb+dir2+.  More precisely, the filename \verb+dir2/foo+ refers to: a) the file
\verb+dir1/foo+ if it exists, or b) \verb+dir2/foo+ otherwise.

The \verb+vmount+ function makes it easy to specify multiple versions of a project.  Suppose we have
a project where the source files are in the directory \verb+src/+, and we want to compile two
versions, one with debugging support and one optimized.  We create two directories, \File{debug} and
\File{opt}, and mount the \File{src} directory over them.

\begin{verbatim}
    section
        CFLAGS += -g
        vmount(-l, src, debug)
        .SUBDIRS: debug

    section
        CFLAGS += -O3
        vmount(-l, src, opt)
        .SUBDIRS: opt
\end{verbatim}

Here, we are using \verb+section+ blocks to define the scope of the \verb+vmount+---you may not need
them in your project.

The \verb+-l+ option is optional.  It specifies that files form the \verb+src+ directory should be
linked into the target directories (or copied, if the system is Win32).  The links are added as
files are referenced.  If no options are given, then files are not copied or linked, but filenames
are translated to refer directly to the \verb+src/+ files.

Now, when a file is referenced in the \verb+debug+ directory, it is linked from the \verb+src+
directory if it exists.  For example, when the file \verb+debug/OMakefile+ is read, the
\verb+src/OMakefile+ is linked into the \verb+debug/+ directory.

The \verb+vmount+ model is fairly transparent.  The \verb+OMakefile+s can be written \emph{as if}
referring to files in the \verb+src/+ directory---they need not be aware of mounting.
However, there are a few points to keep in mind.

\section{Notes}

\begin{itemize}
\item When using the \verb+vmount+ function for versioning, it wise to keep the source files
distinct from the compiled versions.  For example, suppose the source directory contained a file
\verb+src/foo.o+.  When mounted, the \verb+foo.o+ file will be the same in all versions, which is
probably not what you want.  It is better to keep the \verb+src/+ directory pristine, containing no
compiled code.

\item When using the \verb+vmount -l+ option, files are linked into the version directory only if
they are referenced in the project.  Functions that examine the filesystem (like \verb+$(ls ...)+)
may produce unexpected results.
\end{itemize}

% -*-
% Local Variables:
% Mode: LaTeX
% fill-column: 100
% TeX-master: "paper"
% TeX-command-default: "LaTeX/dvips Interactive"
% End:
% -*-

%
%
%
\chapter{Additional build examples}
\label{chapter:build-examples}
\index{build model}
\cutname{omake-build-examples.html}

Let's explain the OMake build model a bit more.
One issue that dominates this discussion is that OMake is based on global project analysis.  That
means you define a configuration for the \emph{entire} project, and you run \emph{one} instance of omake.

For single-directory projects this doesn't mean much.  For multi-directory projects it means a lot.
With GNU make, you would usually invoke the \verb+make+ program recursively for each directory in
the project.  For example, suppose you had a project with some project root directory, containing a
directory of sources \verb+src+, which in turn contains subdirectories \verb+lib+ and \verb+main+.
So your project looks like this nice piece of ASCII art.

\begin{verbatim}
    my_project/
    |--> Makefile
    `--> src/
         |---> Makefile
         |---> lib/
         |     |---> Makefile
         |     `---> source files...
         `---> main/
               |---> Makefile
               `---> source files...
\end{verbatim}
                    
Typically, with GNU make, you would start an instance of \verb+make+ in \verb+my_project/+; this
would in term start an instance of \verb+make+ in the \verb+src/+ directory; and this would start
new instances in \verb+lib/+ and \verb+main/+.  Basically, you count up the number of
\verb+Makefile+s in the project, and that is the number of instances of \verb+make+ processes that
will be created.

The number of processes is no big deal with today's machines (sometimes contrary the the author's opinion, we
no longer live in the 1970s).  The problem with the scheme was that each \verb+make+ process had a
separate configuration, and it took a lot of work to make sure that everything was consistent.
Furthermore, suppose the programmer runs \verb+make+ in the \verb+main/+ directory, but the
\verb+lib/+ is out-of-date.  In this case, \verb+make+ would happily crank away, perhaps trying to
rebuild files in \verb+lib/+, perhaps just giving up.

With OMake this changes entirely.  Well, not entirely.  The source structure is quite similar, we
merely add some Os to the ASCII art.

\begin{verbatim}
    my_project/
    |--> OMakeroot   (or Root.om)
    |--> OMakefile
    `--> src/
         |---> OMakefile
         |---> lib/
         |     |---> OMakefile
         |     `---> source files...
         `---> main/
               |---> OMakefile
               `---> source files...
\end{verbatim}

\index{OMakefile}
\index{OMakeroot}
The role of each \verb+<dir>/OMakefile+ plays the same role as each \verb+<dir>/Makefile+: it
describes how to build the source files in \verb+<dir>+.  The OMakefile retains much of syntax and
structure of the Makefile, but in most cases it is much simpler.

One minor difference is the presence of the OMakeroot in the project root.  The main purpose of this
file is to indicate where the project root \emph{is} in the first place (in case \verb+omake+ is
invoked from a subdirectory).  The \verb+OMakeroot+ serves as the bootstrap file; omake starts by
reading this file first.  Otherwise, the syntax and evaluation of \verb+OMakeroot+ is no different
from any other \verb+OMakefile+.

The \emph{big} difference is that OMake performs a \emph{global} analysis.  Here is what happens
when \verb+omake+ starts.

\index{.SUBDIRS}
\begin{enumerate}
\item omake locates that OMakeroot file, and reads it.
\item Each OMakefile points to its subdirectory OMakefiles using the .SUBDIRS target.
For example, \verb+my_project/OMakefile+ has a rule,

\begin{verbatim}
    .SUBDIRS: src
\end{verbatim}

and the \verb+my_project/src/OMakefile+ has a rule,

\begin{verbatim}
    .SUBDIRS: lib main
\end{verbatim}

\verb+omake+ uses these rules to read and evaluate every \verb+OMakefile+ in the project.
Reading and evaluation is fast.  This part of the process is cheap.

\item Now that the entire configuration is read, \verb+omake+ determines which files are out-of-date
(using a global analysis), and starts the build process.  This may take a while, depending on what
exactly needs to be done.
\end{enumerate}

There are several advantages to this model.  First, since analysis is global, it is much easier to
ensure that the build configuration is consistent--after all, there is only one configuration.
Another benefit is that the build configuration is inherited, and can be re-used, down the
hierarchy.  Typically, the root \verb+OMakefile+ defines some standard boilerplate and
configuration, and this is inherited by subdirectories that tweak and modify it (but do not need to
restate it entirely).  The disadvantage of course is space, since this is global analysis after all.
In practice rarely seems to be a concern; omake takes up much less space than your web browser even
on large projects.

Some notes to the GNU/BSD make user.
\begin{itemize}
\item OMakefiles are a lot like Makefiles.  The syntax is similar, and there many of the builtin
functions are similar.  However, the two build systems are not the same.  Some evil features (in the authors'
opinions) have been dropped in OMake, and some new features have been added.

\item OMake works the same way on all platforms, including Win32.  The standard configuration does
the right thing, but if you care about porting your code to multiple platforms, and you use some
tricky features, you may need to condition parts of your build config on the \verb+$(OSTYPE)+
variable.

\item A minor issue is that OMake dependency analysis is based on MD5 file digests.  That is,
dependencies are based on file \emph{contents}, not file \emph{modification times}.  Say goodbye to
false rebuilds based on spurious timestamp changes and mismatches between local time and fileserver
time.
\end{itemize}

\section{OMakeroot vs. OMakefile}

Before we begin with examples, let's ask the first question, ``What is the difference between the
project root OMakeroot and OMakefile?''  A short answer is, there is no difference, but you must
have an OMakeroot file (or Root.om file).

However, the normal style is that OMakeroot is boilerplate and is more-or-less the same for all
projects.  The OMakefile is where you put all your project-specific stuff.

To get started, you don't have to do this yourself.  In most cases you just perform the following
step in your project root directory.

\begin{itemize}
\item Run \verb+omake --install+ in your project root.
\end{itemize}

This will create the initial OMakeroot and OMakefile files that you can edit to get started.

\section{An example C project}

To begin, let's start with a simple example.  Let's say that we have a full directory tree,
containing the following files.

\begin{verbatim}
    my_project/
    |--> OMakeroot
    |--> OMakefile
    `--> src/
         |---> OMakefile
         |---> lib/
         |     |---> OMakefile
         |     |---> ouch.c
         |     |---> ouch.h
         |     `---> bandaid.c
         `---> main/
               |---> OMakefile
               |---> horsefly.c
               |---> horsefly.h
               `---> main.c
\end{verbatim}

Here is an example listing.

\begin{verbatim}
my_project/OMakeroot:
    # Include the standard configuration for C applications
    open build/C
    
    # Process the command-line vars
    DefineCommandVars()
    
    # Include the OMakefile in this directory.
    .SUBDIRS: .

my_project/OMakefile:
    # Set up the standard configuration
    CFLAGS += -g

    # Include the src subdirectory
    .SUBDIRS: src

my_project/src/OMakefile:
    # Add any extra options you like
    CFLAGS += -O2

    # Include the subdirectories
    .SUBDIRS: lib main

my_project/src/lib/OMakefile:
    # Build the library as a static library.
    # This builds libbug.a on Unix/OSX, or libbug.lib on Win32.
    # Note that the source files are listed _without_ suffix.
    StaticCLibrary(libbug, ouch bandaid)

my_project/src/main/OMakefile:
    # Some files include the .h files in ../lib
    INCLUDES += ../lib

    # Indicate which libraries we want to link against.
    LIBS[] +=
        ../lib/libbug

    # Build the program.
    # Builds horsefly.exe on Win32, and horsefly on Unix.
    # The first argument is the name of the executable.
    # The second argument is an array of object files (without suffix)
    # that are part of the program.
    CProgram(horsefly, horsefly main)

    # Build the program by default (in case omake is called
    # without any arguments).  EXE is defined as .exe on Win32,
    # otherwise it is empty.
    .DEFAULT: horsefly$(EXE)
\end{verbatim}

Most of the configuration here is defined in the file \verb+build/C.om+ (which is part of the OMake
distribution).  This file takes care of a lot of work, including:
\begin{itemize}
\item Defining the \verb+StaticCLibrary+ and \verb+CProgram+ functions, which describe the canonical
way to build C libraries and programs.
\item Defining a mechanism for \emph{scanning} each of the source programs to discover dependencies.
That is, it defines .SCANNER rules for C source files.
\end{itemize}

Variables are inherited down the hierarchy, so for example, the value of CFLAGS in
src/main/OMakefile is ``\verb+-g -O2+''.

\section{An example OCaml project}

Let's repeat the example, assuming we are using OCaml instead of C.
This time, the directory tree looks like this.

\begin{verbatim}
    my_project/
    |--> OMakeroot
    |--> OMakefile
    `--> src/
         |---> OMakefile
         |---> lib/
         |     |---> OMakefile
         |     |---> ouch.ml
         |     |---> ouch.mli
         |     `---> bandaid.ml
         `---> main/
               |---> OMakefile
               |---> horsefly.ml
               |---> horsefly.mli
               `---> main.ml
\end{verbatim}

The listing is only a bit different.

\begin{verbatim}
my_project/OMakeroot:
    # Include the standard configuration for OCaml applications
    open build/OCaml
    
    # Process the command-line vars
    DefineCommandVars()
    
    # Include the OMakefile in this directory.
    .SUBDIRS: .

my_project/OMakefile:
    # Set up the standard configuration
    OCAMLFLAGS += -Wa

    # Do we want to use the bytecode compiler,
    # or the native-code one?  Let's use both for
    # this example.
    NATIVE_ENABLED = true
    BYTE_ENABLED = true

    # Include the src subdirectory
    .SUBDIRS: src

my_project/src/OMakefile:
    # Include the subdirectories
    .SUBDIRS: lib main

my_project/src/lib/OMakefile:
    # Let's do aggressive inlining on native code
    OCAMLOPTFLAGS += -inline 10

    # Build the library as a static library.
    # This builds libbug.a on Unix/OSX, or libbug.lib on Win32.
    # Note that the source files are listed _without_ suffix.
    OCamlLibrary(libbug, ouch bandaid)

my_project/src/main/OMakefile:
    # These files depend on the interfaces in ../lib
    OCAMLINCLUDES += ../lib

    # Indicate which libraries we want to link against.
    OCAML_LIBS[] +=
        ../lib/libbug

    # Build the program.
    # Builds horsefly.exe on Win32, and horsefly on Unix.
    # The first argument is the name of the executable.
    # The second argument is an array of object files (without suffix)
    # that are part of the program.
    OCamlProgram(horsefly, horsefly main)

    # Build the program by default (in case omake is called
    # without any arguments).  EXE is defined as .exe on Win32,
    # otherwise it is empty.
    .DEFAULT: horsefly$(EXE)
\end{verbatim}

In this case, most of the configuration here is defined in the file \verb+build/OCaml.om+.  In this
particular configuration, files in \verb+my_project/src/lib+ are compiled aggressively with the
option \verb+-inline 10+, but files in \verb+my_project/src/lib+ are compiled normally.

\section{Handling new languages}
\index{cats and dogs}

The previous two examples seem to be easy enough, but they rely on the OMake standard library (the
files \verb+build/C+ and \verb+build/OCaml+) to do all the work.  What happens if we want to write a
build configuration for a language that is not already supported in the OMake standard library?

For this example, let's suppose we are adopting a new language.  The language uses the standard
compile/link model, but is not in the OMake standard library.  Specifically, let's say we have the
following setup.

\begin{itemize}
\item Source files are defined in files with a \verb+.cat+ suffix (for Categorical Abstract Terminology).
\item \verb+.cat+ files are compiled with the \verb+catc+ compiler to produce \verb+.woof+ files
(Wicked Object-Oriented Format).
\item \verb+.woof+ files are linked by the \verb+catc+ compiler with the \verb+-c+ option to produce
a \verb+.dog+ executable (Digital Object Group).  The \verb+catc+ also defines a \verb+-a+ option to
combine several \verb+.woof+ files into a library.
\item Each \verb+.cat+ can refer to other source files.  If a source file \verb+a.cat+ contains a
line \verb+open b+, then \verb+a.cat+ depends on the file \verb+b.woof+, and \verb+a.cat+ must be
recompiled if \verb+b.woof+ changes.  The \verb+catc+ function takes a \verb+-I+ option to define a
search path for dependencies.
\end{itemize}

To define a build configuration, we have to do three things.
\begin{enumerate}
\item Define a \verb+.SCANNER+ rule for discovering dependency information for the source files.
\item Define a generic rule for compiling a \verb+.cat+ file to a \verb+.woof+ file.
\item Define a rule (as a function) for linking \verb+.woof+ files to produce a \verb+.dog+ executable.
\end{enumerate}

Initially, these definitions will be placed in the project root \verb+OMakefile+.

\subsection{Defining a default compilation rule}

Let's start with part 2, defining a generic compilation rule.  We'll define the build rule as an
\emph{implicit} rule.  To handle the include path, we'll define a variable \verb+CAT_INCLUDES+ that
specifies the include path.  This will be an array of directories.  To define the options, we'll use
a lazy variable (Section~\ref{section:lazy}).  In case there
are any other standard flags, we'll define a \verb+CAT_FLAGS+ variable.

\begin{verbatim}
   # Define the catc command, in case we ever want to override it
   CATC = catc

   # The default flags are empty
   CAT_FLAGS =
   
   # The directories in the include path (empty by default)
   INCLUDES[] =

   # Compute the include options from the include path
   PREFIXED_INCLUDES[] = $`(mapprefix -I, $(INCLUDES))

   # The default way to build a .woof file
   %.woof: %.cat
       $(CATC) $(PREFIXED_INCLUDES) $(CAT_FLAGS) -c $<
\end{verbatim}

The final part is the build rule itself, where we call the \verb+catc+ compiler with the include
path, and the \verb+CAT_FLAGS+ that have been defined.  The \verb+$<+ variable represents the source
file.

\subsection{Defining a rule for linking}

For linking, we'll define another rule describing how to perform linking.  Instead of defining an
implicit rule, we'll define a function that describes the linking step.  The function will take two
arguments; the first is the name of the executable (without suffix), and the second is the files to
link (also without suffixes).  Here is the code fragment.

\begin{verbatim}
    # Optional link options
    CAT_LINK_FLAGS =

    # The function that defines how to build a .dog program
    CatProgram(program, files) =
        # Add the suffixes
        file_names = $(addsuffix .woof, $(files))
        prog_name = $(addsuffix .dog, $(files))

        # The build rule
        $(prog_name): $(file_names)
            $(CATC) $(PREFIXED_INCLUDES) $(CAT_FLAGS) $(CAT_LINK_FLAGS) -o $@ $+

        # Return the program name
        value $(prog_name)
\end{verbatim}

The \verb+CAT_LINK_FLAGS+ variable is defined just in case we want to pass additional flags specific
to the link step.  Now that this function is defined, whenever we want to define a rule for building
a program, we simply call the rule.  The previous implicit rule specifies how to compile each source file,
and the \verb+CatProgram+ function specifies how to build the executable.

\begin{verbatim}
    # Build a rover.dog program from the source
    # files neko.cat and chat.cat.
    # Compile it by default.
    .DEFAULT: $(CatProgram rover, neko chat)
\end{verbatim}

\subsection{Dependency scanning}
\index{.SCANNER}

That's it, almost.  The part we left out was automated dependency scanning.  This is one of the
nicer features of OMake, and one that makes build specifications easier to write and more robust.
Strictly speaking, it isn't required, but you definitely want to do it.

The mechanism is to define a \verb+.SCANNER+ rule, which is like a normal rule, but it specifies how
to compute dependencies, not the target itself.  In this case, we want to define a \verb+.SCANNER+
rule of the following form.

\begin{verbatim}
    .SCANNER: %.woof: %.cat
        <commands>
\end{verbatim}

This rule specifies that a \verb+.woof+ file may have additional dependencies that can be extracted
from the corresponding \verb+.cat+ file by executing the \verb+<commands>+.  The \emph{result} of
executing the \verb+<commands>+ should be a sequence of dependencies in OMake format, printed to the
standard output.

As we mentioned, each \verb+.cat+ file specifies dependencies on \verb+.woof+ files with an
\verb+open+ directive.  For example, if the \verb+neko.cat+ file contains a line \verb+open chat+,
then \verb+neko.woof+ depends on \verb+chat.woof+.  In this case, the \verb+<commands>+ should print
the following line.

\begin{verbatim}
    neko.woof: chat.woof
\end{verbatim}

For an analogy that might make this clearer, consider the C programming language, where a \verb+.o+
file is produced by compiling a \verb+.c+ file.  If a file \verb+foo.c+ contains a line like
\verb+#include "fum.h"+, then \verb+foo.c+ should be recompiled whenever \verb+fum.h+ changes.  That
is, the file \verb+foo.o+ \emph{depends} on the file \verb+fum.h+.  In the OMake parlance, this is
called an \emph{implicit} dependency, and the \verb+.SCANNER+ \verb+<commands>+ would print a line
like the following.

\begin{verbatim}
    foo.o: fum.h
\end{verbatim}

\index{awk} Now, returning to the animal world, to compute the dependencies of \verb+neko.woof+, we
should scan \verb+neko.cat+, line-by-line, looking for lines of the form \verb+open <name>+.  We
could do this by writing a program, but it is easy enough to do it in \verb+omake+ itself.  We can
use the builtin \verb+awk+ function (Section~\ref{fun:awk}) to scan the source file.  One slight complication
is that the dependencies depend on the \verb+INCLUDE+ path.  We'll use the
\verb+find-in-path+ function (Section~\ref{fun:find-in-path}) to find them.  Here we go.

\begin{verbatim}
    .SCANNER: %.woof: %.cat
        section
            # Scan the file
            deps[] =
            awk($<)
            case $'^open'
                deps[] += $2
                export

            # Remove duplicates, and find the files in the include path
            deps = $(find-in-path $(INCLUDES), $(set $(deps)))

            # Print the dependencies
            println($"$@: $(deps)")
\end{verbatim}

Let's look at the parts.  First, the entire body is defined in a \verb+section+ because we are
computing it internally, not as a sequence of shell commands.

We use the \verb+deps+ variable to collect all the dependencies.  The \verb+awk+ function scans the
source file (\verb+$<+) line-by-line.  For lines that match the regular expression \verb+^open+
(meaning that the line begins with the word \verb+open+), we add the second word on the line to the
\verb+deps+ variable.  For example, if the input line is \verb+open chat+, then we would add the
\verb+chat+ string to the \verb+deps+ array.  All other lines in the source file are ignored.

Next, the \verb+$(set $(deps))+ expression removes any duplicate values in the \verb+deps+ array
(sorting the array alphabetically in the process).  The \verb+find-in-path+ function then finds the
actual location of each file in the include path.

The final step is print the result as the string \verb+$"$@: $(deps)"+ The quotations are added to
flatten the \verb+deps+ array to a simple string.

\subsection{Pulling it all together}

To complete the example, let's pull it all together into a single project, much like our previous
example.

\begin{verbatim}
    my_project/
    |--> OMakeroot
    |--> OMakefile
    `--> src/
         |---> OMakefile
         |---> lib/
         |     |---> OMakefile
         |     |---> neko.cat
         |     `---> chat.cat
         `---> main/
               |---> OMakefile
               `---> main.cat
\end{verbatim}

The listing for the entire project is as follows.  Here, we also include a function
\verb+CatLibrary+ to link several \verb+.woof+ files into a library.

\begin{verbatim}
my_project/OMakeroot:
    # Process the command-line vars
    DefineCommandVars()
    
    # Include the OMakefile in this directory.
    .SUBDIRS: .

my_project/OMakefile:
   ########################################################################
   # Standard config for compiling .cat files
   #

   # Define the catc command, in case we ever want to override it
   CATC = catc

   # The default flags are empty
   CAT_FLAGS =
   
   # The directories in the include path (empty by default)
   INCLUDES[] =

   # Compute the include options from the include path
   PREFIXED_INCLUDES[] = $`(mapprefix -I, $(INCLUDES))

   # Dependency scanner for .cat files
   .SCANNER: %.woof: %.cat
        section
            # Scan the file
            deps[] =
            awk($<)
            case $'^open'
                deps[] += $2
                export

            # Remove duplicates, and find the files in the include path
            deps = $(find-in-path $(INCLUDES), $(set $(deps)))

            # Print the dependencies
            println($"$@: $(deps)")

   # The default way to compile a .cat file
   %.woof: %.cat
       $(CATC) $(PREFIXED_INCLUDES) $(CAT_FLAGS) -c $<

   # Optional link options
   CAT_LINK_FLAGS =

   # Build a library for several .woof files
   CatLibrary(lib, files) =
       # Add the suffixes
       file_names = $(addsuffix .woof, $(files))
       lib_name = $(addsuffix .woof, $(lib))

       # The build rule
       $(lib_name): $(file_names)
           $(CATC) $(PREFIXED_INCLUDES) $(CAT_FLAGS) $(CAT_LINK_FLAGS) -a $@ $+

       # Return the program name
       value $(lib_name)

   # The function that defines how to build a .dog program
   CatProgram(program, files) =
       # Add the suffixes
       file_names = $(addsuffix .woof, $(files))
       prog_name = $(addsuffix .dog, $(program))

       # The build rule
       $(prog_name): $(file_names)
           $(CATC) $(PREFIXED_INCLUDES) $(CAT_FLAGS) $(CAT_LINK_FLAGS) -o $@ $+

       # Return the program name
       value $(prog_name)

   ########################################################################
   # Now the program proper
   #

   # Include the src subdirectory
   .SUBDIRS: src

my_project/src/OMakefile:
   .SUBDIRS: lib main

my_project/src/lib/OMakefile:
   CatLibrary(cats, neko chat)

my_project/src/main/OMakefile:
   # Allow includes from the ../lib directory
   INCLUDES[] += ../lib

   # Build the program
   .DEFAULT: $(CatProgram main, main ../cats)
\end{verbatim}

Some notes.  The configuration in the project \verb+OMakeroot+ defines the standard configuration, including
the dependency scanner, the default rule for compiling source files, and functions for building
libraries and programs.

These rules and functions are inherited by subdirectories, so the \verb+.SCANNER+ and build rules
are used automatically in each subdirectory, so you don't need to repeat them.

\subsection{Finishing up}

At this point we are done, but there are a few things we can consider.

First, the rules for building cat programs is defined in the project \verb+OMakefile+.  If you had
another cat project somewhere, you would need to copy the \verb+OMakeroot+ (and modify it as
needed).  Instead of that, you should consider moving the configuration to a shared library
directory, in a file like \verb+Cat.om+.  That way, instead of copying the code, you could include
the shared copy with an \OMake command \verb+open Cat+.  The share directory should be added to your
\verb+OMAKEPATH+ environment variable to ensure that \verb+omake+ knows how to find it.

Better yet, if you are happy with your work, consider submitting it as a standard configuration (by
sending a request to \verb+omake@metaprl.org+) so that others can make use of it too.

\section{Collapsing the hierarchy, .SUBDIRS bodies}
\index{.SUBDIRS bodies}

Some projects have many subdirectories that all have the same configuration.  For instance, suppose
you have a project with many subdirectories, each containing a set of images that are to be composed
into a web page.  Apart from the specific images, the configuration of each file is the same.

To make this more concrete, suppose the project has four subdirectories \verb+page1+, \verb+page2+,
\verb+page3+, and \verb+page4+.  Each contains two files \verb+image1.jpg+ and \verb+image2.jpg+
that are part of a web page generated by a program \verb+genhtml+.

Instead of of defining a \verb+OMakefile+ in each directory, we can define it as a body to the
\verb+.SUBDIRS+ command.

\begin{verbatim}
    .SUBDIRS: page1 page2 page3 page4
        index.html: image1.jpg image2jpg
            genhtml $+ > $@
\end{verbatim}

The body of the \verb+.SUBDIRS+ is interpreted exactly as if it were the \verb+OMakefile+, and it
can contain any of the normal statements.  The body is evaluated \emph{in the subdirectory} for each
of the subdirectories.  We can see this if we add a statement that prints the current directory
(\verb+$(CWD)+).

\begin{verbatim}
    .SUBDIRS: page1 page2 page3 page4
        println($(absname $(CWD)))
        index.html: image1.jpg image2jpg
            genhtml $+ > $@
  # prints
    /home/jyh/.../page1
    /home/jyh/.../page2
    /home/jyh/.../page3
    /home/jyh/.../page4
\end{verbatim}

\subsection{Using glob patterns}

Of course, this specification is quite rigid.  In practice, it is likely that each subdirectory will
have a different set of images, and all should be included in the web page.  One of the easier
solutions is to use one of the directory-listing functions, like
\verb+glob+ (Section~\ref{fun:glob}) or \verb+ls+ (Section~\ref{fun:ls}).
The \verb+glob+ function takes a shell pattern, and returns an array of
file with matching filenames in the current directory.

\begin{verbatim}
    .SUBDIRS: page1 page2 page3 page4
        IMAGES = $(glob *.jpg)
        index.html: $(IMAGES)
            genhtml $+ > $@
\end{verbatim}

\subsection{Simplified sub-configurations}

Another option is to add a configuration file in each of the subdirectories that defines
directory-specific information.  For this example, we might define a file \verb+BuildInfo.om+ in
each of the subdirectories that defines a list of images in that directory.  The \verb+.SUBDIRS+
line is similar, but we include the BuildInfo file.

\begin{verbatim}
    .SUBDIRS: page1 page2 page3 page4
        include BuildInfo   # Defines the IMAGES variable

        index.html: $(IMAGES)
            genhtml $+ > $@
\end{verbatim}

Where we might have the following configurations.

\begin{verbatim}
   page1/BuildInfo.om:
       IMAGES[] = image.jpg
   page2/BuildInfo.om:
       IMAGES[] = ../common/header.jpg winlogo.jpg
   page3/BuildInfo.om:
       IMAGES[] = ../common/header.jpg unixlogo.jpg daemon.jpg
   page4/BuildInfo.om:
       IMAGES[] = fee.jpg fi.jpg foo.jpg fum.jpg
\end{verbatim}

\subsection{Computing the subdirectory list}

The other hardcoded specification is the list of subdirectories \verb+page1+, ..., \verb+page4+.
Rather than editing the project \verb+OMakefile+ each time a directory is added, we could compute it
(again with \verb+glob+).

\begin{verbatim}
    .SUBDIRS: $(glob page*)
        index.html: $(glob *.jpg)
            genhtml $+ > $@
\end{verbatim}

Alternately, the directory structure may be hierarchical.  Instead of using \verb+glob+, we could
use the \verb+subdirs+ function, returns each of the directories in a hierarchy.  For example, this
is the result of evaluating the \verb+subdirs+ function in the omake project root.  The \verb+P+
option, passed as the first argument, specifies that the listing is ``proper,'' it should not
include the \verb+omake+ directory itself.

\begin{verbatim}
    osh> subdirs(P, .)
    - : <array
            /home/jyh/.../omake/mk : Dir
            /home/jyh/.../omake/RPM : Dir
            ...
            /home/jyh/.../omake/osx_resources : Dir>
\end{verbatim}

Using \verb+subdirs+, our example is now as follows.

\begin{verbatim}
    .SUBDIRS: $(subdirs P, .)
        index.html: $(glob *.jpg)
            genhtml $+ > $@
\end{verbatim}

In this case, \emph{every} subdirectory will be included in the project.

If we are using the \verb+BuildInfo.om+ option.  Instead of including every subdirectory, we could
include only those that contain a \verb+BuildInfo.om+ file.  For this purpose, we can use the
\verb+find+ function, which traverses the directory hierarchy looking for files that match a test
expression.  In our case, we want to search for files with the name \verb+BuildInfo.om+.
Here is an example call.

\begin{verbatim}
    osh> FILES = $(find . -name BuildInfo.om)
    - : <array
            /home/jyh/.../omake/doc/html/BuildInfo.om : File
            /home/jyh/.../omake/src/BuildInfo.om : File
            /home/jyh/.../omake/tests/simple/BuildInfo.om : File>
    osh> DIRS = $(dirof $(FILES))
    - : <array
            /home/jyh/.../omake/doc/html : Dir
            /home/jyh/.../omake/src : Dir
            /home/jyh/.../omake/tests/simple : Dir>
\end{verbatim}

In this example, there are three \verb+BuildInfo.om+ files, in the \verb+doc/html+, \verb+src+, and
\verb+tests/simple+ directories.  The \verb+dirof+ function returns the directories for each of the
files.

Returning to our original example, we modify it as follows.

\begin{verbatim}
    .SUBDIRS: $(dirof $(find . -name BuildInfo.om))
        include BuildInfo   # Defines the IMAGES variable

        index.html: $(IMAGES)
            genhtml $+ > $@
\end{verbatim}

\subsection{Temporary directories}

Sometimes, your project may include temporary directories--directories where you place intermediate
results.  these directories are deleted whenever the project is cleanup up.  This means, in
particular, that you can't place an \verb+OMakefile+ in a temporary directory, because it will be
removed when the directory is removed.

Instead, if you need to define a configuration for any of these directories, you will need to define
it using a \verb+.SUBDIRS+ body.

\begin{verbatim}
    section
        CREATE_SUBDIRS = true

        .SUBDIRS: tmp
            # Compute an MD5 digest
            %.digest: %.comments
               echo $(digest $<) > $@

            # Extract comments from the source files
            %.comments: ../src/%.src
               grep '^#' $< > $@

            .DEFAULT: foo.digest

    .PHONY: clean

    clean:
        rm -rf tmp        
\end{verbatim}

In this example, we define the \verb+CREATE_SUBDIRS+ variable as true, so that the \verb+tmp+
directory will be created if it does not exist.  The \verb+.SUBDIRS+ body in this example is a bit
contrived, but it illustrates the kind of specification you might expect.  The \verb+clean+
phony-target indicates that the \verb+tmp+ directory should be removed when the project is cleaned
up.

% -*-
% Local Variables:
% Mode: LaTeX
% fill-column: 100
% TeX-master: "paper"
% TeX-command-default: "LaTeX/dvips Interactive"
% End:
% -*-

%%%%%%%%%%%%%%%%%%%%%%%%%%%%%%%%%%%%%%%%%%%%%%%%%%%%%%%%%%%%%%%%%%%%%%%%
% Description
%
\chapter{\OMake{} concepts and syntax}
\label{chapter:language}
\cutname{omake-language.html}

Projects are specified to \Prog{omake} with \File{OMakefile}s.  The \File{OMakefile} has a format
similar to a \File{Makefile}.  An \File{OMakefile} has three main kinds of syntactic objects:
variable definitions, function definitions, and rule definitions.

\section{Variables}
\label{section:variables}

Variables are defined with the following syntax.  The name is any sequence of alphanumeric
characters, underscore \verb+_+, and hyphen \verb+-+.

\begin{verbatim}
   <name> = <value>
\end{verbatim}

Values are defined as a sequence of literal characters and variable expansions.  A variable
expansion has the form \verb+$(<name>)+, which represents the value of the \verb+<name>+
variable in the current environment.  Some examples are shown below.

\begin{verbatim}
   CC = gcc
   CFLAGS = -Wall -g
   COMMAND = $(CC) $(CFLAGS) -O2
\end{verbatim}

In this example, the value of the \verb+COMMAND+ variable is the string \verb+gcc -Wall -g -O2+.

Unlike \Cmd{make}{1}, variable expansion is \emph{eager} and \emph{functional} (see also the section
on Scoping).  That is, variable values are expanded immediately and new variable definitions do not
affect old ones.  For example, suppose we extend the previous example with following variable
definitions.

\begin{verbatim}
   X = $(COMMAND)
   COMMAND = $(COMMAND) -O3
   Y = $(COMMAND)
\end{verbatim}

In this example, the value of the \verb+X+ variable is the string \verb+gcc -Wall -g -O2+ as
before, and the value of the \verb+Y+ variable is \verb+gcc -Wall -g -O2 -O3+.

\section{Adding to a variable definition}

Variables definitions may also use the += operator, which adds the new text to an existing
definition.  The following two definitions are equivalent.

\begin{verbatim}
   # Add options to the CFLAGS variable
   CFLAGS = $(CFLAGS) -Wall -g

   # The following definition is equivalent
   CFLAGS += -Wall -g
\end{verbatim}

\section{Arrays}
\index{arrays}

Arrays can be defined by appending the \verb+[]+ sequence to the variable name and defining initial
values for the elements as separate lines.  Whitespace is significant on each line.  The following
code sequence prints \verb+c d e+.

\begin{verbatim}
    X[] =
        a b
        c d e
        f

    println($(nth 2, $(X)))
\end{verbatim}

\section{Special characters and quoting}
\index{quotations}

The following characters are special to \Prog{omake}: \verb+$():,=#\+.  To treat
any of these characters as normal text, they should be escaped with the backslash
character \verb+\+.

\begin{verbatim}
    DOLLAR = \$
\end{verbatim}

Newlines may also be escaped with a backslash to concatenate several lines.

\begin{verbatim}
    FILES = a.c\
            b.c\
            c.c
\end{verbatim}

Note that the backslash is \emph{not} an escape for any other character, so the following
works as expected (that is, it preserves the backslashes in the string).

\begin{verbatim}
    DOSTARGET = C:\WINDOWS\control.ini
\end{verbatim}

An alternative mechanism for quoting special text is the use \verb+$"..."+ escapes.  The number of
double-quotations is arbitrary.  The outermost quotations are not included in the text.

\begin{verbatim}
    A = $""String containing "quoted text" ""
    B = $"""Multi-line
        text.
        The # character is not special"""
\end{verbatim}

\section{Function definitions}
\label{section:functions}
\index{functions}

Functions are defined using the following syntax.

\begin{verbatim}
   <name>(<params>) =
      <indented-body>
\end{verbatim}

The parameters are a comma-separated list of identifiers, and the body must be placed on a separate
set of lines that are indented from the function definition itself.  For example, the following text
defines a function that concatenates its arguments, separating them with a colon.

\begin{verbatim}
    ColonFun(a, b) =
        return($(a):$(b))
\end{verbatim}

\index{return}%
The \verb+return+ expression can be used to return a value from the function.  A \verb+return+
statement is not required; if it is omitted, the returned value is the value of the last expression
in the body to be evaluated.  NOTE: as of version \verb+0.9.6+, \verb+return+ is a control
operation, causing the function to immediately return.  In the following example, when the argument
\verb+a+ is true, the function \verb+f+ immediately returns the value 1 without evaluating the print
statement.

\begin{verbatim}
    f(a) =
       if $(a)
          return 1
       println(The argument is false)
       return 0
\end{verbatim}

\index{value}%
In many cases, you may wish to return a value from a section or code block without returning from
the function.  In this case, you would use the \verb+value+ operator.  In fact, the \verb+value+
operator is not limited to functions, it can be used any place where a value is required.  In the
following definition, the variable \verb+X+ is defined as $1$ or $2$, depending on the value of $a$,
then result is printed, and returned from the function.

\begin{verbatim}
    f_value(a) =
       X =
          if $(a)
             value 1
          else
             value 2
       println(The value of X is $(X))
       value $(X)
\end{verbatim}

Functions are called using the GNU-make syntax, \verb+$(<name> <args))+,
where \verb+<args>+ is a comma-separated list of values.  For example,
in the following program, the variable \verb+X+ contains the
value \verb+foo:bar+.

\begin{verbatim}
   X = $(ColonFun foo, bar)
\end{verbatim}

If the value of a function is not needed, the function may also be called
using standard function call notation.  For example, the following program
prints the string ``She says: Hello world''.

\begin{verbatim}
    Printer(name) =
        println($(name) says: Hello world)

    Printer(She)
\end{verbatim}

\section{Comments}

Comments begin with the \verb+#+ character and continue to the end of the line.

\section{File inclusion}
\label{section:include}
\index{include}\index[fun]{include}\index{open}

Files may be included with the \verb+include+ or \verb+open+ form.  The included file must use
the same syntax as an \File{OMakefile}.

\begin{verbatim}
    include $(Config_file)
\end{verbatim}

The \verb+open+ operation is similar to an \verb+include+, but the file is included at most once.
\begin{verbatim}
    open Config

    # Repeated opens are ignored, so this
    # line has no effect.
    open Config
\end{verbatim}

If the file specified is not an absolute filenmame, both \verb+include+ and
\verb+open+ operations search for the file based on the
\hypervar{OMAKEPATH}. In case of the \verb+open+ directive, the search is
performed at \emph{parse} time, and the argument to \verb+open+ may not
contain any expressions.

\section{Scoping, sections}
\label{section:section}
\index{section}

Scopes in \Prog{omake} are defined by indentation level.  When indentation is
increased, such as in the body of a function, a new scope is introduced.

The \verb+section+ form can also be used to define a new scope.  For example, the following code
prints the line \verb+X = 2+, followed by the line \verb+X = 1+.

\begin{verbatim}
    X = 1
    section
        X = 2
        println(X = $(X))

    println(X = $(X))
\end{verbatim}

This result may seem surprising--the variable definition within the
\verb+section+ is not visible outside the scope of the \verb+section+.

The \verb+export+ form, which will be described in detail in
Section~\ref{section:export}, can be used to circumvent this restriction by
exporting variable values from an inner scope.  It must be the final
expression in a scope.  For example, if we modify the previous example
by adding an \verb+export+ expression, the new value for the \verb+X+
variable is retained, and the code prints the line \verb+X = 2+ twice.

\begin{verbatim}
    X = 1
    section
        X = 2
        println(X = $(X))
        export

    println(X = $(X))
\end{verbatim}

There are also cases where separate scoping is quite important.  For example,
each \File{OMakefile} is evaluated in its own scope.  Since each part of a project
may have its own configuration, it is important that variable definitions in one
\File{OMakefile} do not affect the definitions in another.

To give another example, in some cases it is convenient to specify a
separate set of variables for different build targets.  A frequent
idiom in this case is to use the \verb+section+ command to define a
separate scope.

\begin{verbatim}
   section
      CFLAGS += -g
      %.c: %.y
          $(YACC) $<
      .SUBDIRS: foo

   .SUBDIRS: bar baz
\end{verbatim}

In this example, the \verb+-g+ option is added to the \verb+CFLAGS+
variable by the \verb+foo+ subdirectory, but not by the \verb+bar+ and
\verb+baz+ directories. The implicit rules are scoped as well and in this
example, the newly added yacc rule will be inherited by the \verb+foo+
subdirectory, but not by the \verb+bar+ and \verb+baz+ ones; furthermore
this implicit rule will not be in scope in the current directory.

\section{Conditionals}
\label{section:conditionals}
\index{conditionals}
\index{if}

Top level conditionals have the following form.

\begin{verbatim}
    if <test>
       <true-clause>
    elseif <text>
       <elseif-clause>
    else
       <else-clause>
\end{verbatim}

The \verb+<test>+ expression is evaluated, and if it evaluates to a \emph{true} value (see
Section~\ref{section:logic} for more information on logical values, and Boolean functions), the code
for the \verb+<true-clause>+ is evaluated; otherwise the remaining clauses are evaluated.  There may
be multiple \verb+elseif+ clauses; both the \verb+elseif+ and \verb+else+ clauses are optional.
Note that the clauses are indented, so they introduce new scopes.

When viewed as a predicate, a value corresponds to the Boolean \emph{false}, if its string
representation is the empty string, or one of the strings \verb+false+, \verb+no+, \verb+nil+,
\verb+undefined+, or \verb+0+.  All other values are \emph{true}.

The following example illustrates a typical use of a conditional.  The
\verb+OSTYPE+ variable is the current machine architecture.

\begin{verbatim}
    # Common suffixes for files
    if $(equal $(OSTYPE), Win32)
       EXT_LIB = .lib
       EXT_OBJ = .obj
       EXT_ASM = .asm
       EXE = .exe
       export
    elseif $(mem $(OSTYPE), Unix Cygwin)
       EXT_LIB = .a
       EXT_OBJ = .o
       EXT_ASM = .s
       EXE =
       export
    else
       # Abort on other architectures
       eprintln(OS type $(OSTYPE) is not recognized)
       exit(1)
\end{verbatim}

\section{Matching}
\label{section:match}
\index{match}\index[fun]{match}
\index{switch}\index[fun]{switch}

Pattern matching is performed with the \verb+switch+ and \verb+match+ forms.

\begin{verbatim}
    switch <string>
    case <pattern1>
        <clause1>
    case <pattern2>
        <clause2>
    ...
    default
       <default-clause>
\end{verbatim}

The number of cases is arbitrary.
The \verb+default+ clause is optional; however, if it is used it should
be the last clause in the pattern match.

For \verb+switch+, the string is compared with the patterns literally.

\begin{verbatim}
    switch $(HOST)
    case mymachine
        println(Building on mymachine)
    default
        println(Building on some other machine)
\end{verbatim}

Patterns need not be constant strings.  The following function tests
for a literal match against \verb+pattern1+, and a match against
\verb+pattern2+ with \verb+##+ delimiters.

\begin{verbatim}
   Switch2(s, pattern1, pattern2) =
      switch $(s)
      case $(pattern1)
          println(Pattern1)
      case $"##$(pattern2)##"
          println(Pattern2)
      default
          println(Neither pattern matched)
\end{verbatim}

For \verb+match+ the patterns are \Cmd{egrep}{1}-style regular expressions.
The numeric variables \verb+$1, $2, ...+ can be used to retrieve values
that are matched by \verb+\(...\)+ expressions.

\begin{verbatim}
    match $(NODENAME)@$(SYSNAME)@$(RELEASE)
    case $"mymachine.*@\(.*\)@\(.*\)"
        println(Compiling on mymachine; sysname $1 and release $2 are ignored)

    case $".*@Linux@.*2\.4\.\(.*\)"
        println(Compiling on a Linux 2.4 system; subrelease is $1)

    default
        eprintln(Machine configuration not implemented)
        exit(1)
\end{verbatim}

%%%%%%%%%%%%%%%%%%%%%%%%%%%%%%%%%%%%%%%%%%%%%%%%%%%%%%%%%%%%%%%%%%%%%%%%
% Objects
%
\section{Objects}
\label{section:objects}
\index{objects}

\OMake{} is an object-oriented language.  Generally speaking, an object is a value that contains fields
and methods.  An object is defined with a \verb+.+ suffix for a variable.  For example, the
following object might be used to specify a point $(1, 5)$ on the two-dimensional plane.

\begin{verbatim}
    Coord. =
        x = 1
        y = 5
        print(message) =
           println($"$(message): the point is ($(x), $(y)")

    # Define X to be 5
    X = $(Coord.x)

    # This prints the string, "Hi: the point is (1, 5)"
    Coord.print(Hi)
\end{verbatim}

The fields \verb+x+ and \verb+y+ represent the coordinates of the point.  The method \verb+print+
prints out the position of the point.

\section{Classes}
\index{classes}

We can also define \emph{classes}.  For example, suppose we wish to define a generic \verb+Point+
class with some methods to create, move, and print a point.  A class is really just an object with
a name, defined with the \verb+class+ directive.

\begin{verbatim}
    Point. =
        class Point

        # Default values for the fields
        x = 0
        y = 0

        # Create a new point from the coordinates
        new(x, y) =
           this.x = $(x)
           this.y = $(y)
           return $(this)

        # Move the point to the right
        move-right() =
           x = $(add $(x), 1)
           return $(this)

        # Print the point
        print() =
           println($"The point is ($(x), $(y)")

    p1 = $(Point.new 1, 5)
    p2 = $(p1.move-right)

    # Prints "The point is (1, 5)"
    p1.print()

    # Prints "The point is (2, 5)"
    p2.print()
\end{verbatim}

Note that the variable \verb+$(this)+ is used to refer to the current object.  Also, classes and
objects are \emph{functional}---the \verb+new+ and \verb+move-right+ methods return new objects.  In
this example, the object \verb+p2+ is a different object from \verb+p1+, which retains the original
$(1, 5)$ coordinates.

\section{Inheritance}
\index{inheritance}

Classes and objects support inheritance (including multiple inheritance) with the \verb+extends+
directive.  The following definition of \verb+Point3D+ defines a point with \verb+x+, \verb+y+, and
\verb+z+ fields.  The new object inherits all of the methods and fields of the parent classes/objects.

\begin{verbatim}
    Z. =
       z = 0

    Point3D. =
       extends $(Point)
       extends $(Z)
       class Point3D

       print() =
          println($"The 3D point is ($(x), $(y), $(z))")

    # The "new" method was not redefined, so this
    # defines a new point (1, 5, 0).
    p = $(Point3D.new 1, 5)
\end{verbatim}

\section{Special objects/sections}

Objects provide one way to manage the \OMake{} namespace.  There are also four special objects that are
further used to control the namespace.

\section{private.}
\label{section:private}\index{private.}

The \verb+private.+ section is used to define variables that are private to the current file/scope.
The values are not accessible outside the scope.  Variables defined in a \verb+private.+ object can
be accessed only from within the section where they are defined.

\begin{verbatim}
    Obj. =
       private. =
          X = 1

       print() =
          println(The value of X is: $(X))

    # Prints:
    #    The private value of X is: 1
    Obj.print()

    # This is an error--X is private in Obj
    y = $(Obj.X)
\end{verbatim}

In addition, private definitions do not affect the global value of a variable.

\begin{verbatim}
   # The public value of x is 1
   x = 1
   f() =
       println(The public value of x is: $(x))

   # This object uses a private value of x
   Obj. =
       private. =
          x = 2

       print() =
          x = 3
          println(The private value of x is: $(x))
          f()

   # Prints:
   #    The private value of x is: 3
   #    The public value of x is: 1
   Obj.print()
\end{verbatim}

Private variables have two additional properties.

\begin{enumerate}
\item Private variables are local to the file in which they are defined.
\item Private variables are not exported by the \verb+export+ directive, unless they are
  mentioned explicitly.

  \begin{verbatim}
       private. =
          FLAG = true

       section
          FLAG = false
          export

       # FLAG is still true
       section
          FLAG = false
          export FLAG

       # FLAG is now false
  \end{verbatim}
\end{enumerate}

\section{protected.}
\index{protected.}

The \verb+protected.+ object is used to define fields that are local to an object.  They can
be accessed as fields, but they are not passed dynamically to other functions.  The purpose of a
protected variable is to prevent a variable definition within the object from affecting other parts
of the project.

\begin{verbatim}
    X = 1
    f() =
       println(The public value of X is: $(X))

    # Prints:
    #    The public value of X is: 2
    section
       X = 2
       f()

    # X is a protected field in the object
    Obj. =
       protected. =
          X = 3

       print() =
          println(The protected value of X is: $(X))
          f()

    # Prints:
    #    The protected value of X is: 3
    #    The public value of X is: 1
    Obj.print()

    # This is legal, it defines Y as 3
    Y = $(Obj.X)
\end{verbatim}

In general, it is a good idea to define object variables as protected.  The resulting code is more
modular because variables in your object will not produce unexpected clashes with variables defined
in other parts of the project.

\section{public.}
\label{section:public}\index{public.}

The \verb+public.+ object is used to specify public dynamically-scoped variables.  In the following
example, the \verb+public.+ object specifies that the value \verb+X = 4+ is to be dynamically
scoped.  Public variables \emph{are not} defined as fields of an object.

\begin{verbatim}
    X = 1
    f() =
       println(The public value of X is: $(X))

    # Prints:
    #    The public value of X is: 2
    section
       X = 2
       f()

    Obj. =
       protected. =
          X = 3

       print() =
          println(The protected value of X is: $(X))
          public. =
             X = 4
          f()

    # Prints:
    #    The protected value of X is: 3
    #    The public value of X is: 4
    Obj.print()
\end{verbatim}

\section{static.}
\index{static.}

The \verb+static.+ object is used to specify values that are persistent across runs of \OMake{}.  They
are frequently used for configuring a project.  Configuring a project can be expensive, so the
\verb+static.+ object ensure that the configuration is performed just once.  In the following
(somewhat trivial) example, a \verb+static+ section is used to determine if the \LaTeX\ command is
available.  The \verb+$(where latex)+ function returns the full pathname for \verb+latex+, or
\verb+false+ if the command is not found.

\begin{verbatim}
   static. =
      LATEX_ENABLED = false
      print(--- Determining if LaTeX is installed )
      if $(where latex)
          LATEX_ENABLED = true
          export

      if $(LATEX_ENABLED)
         println($'(enabled)')
      else
         println($'(disabled)')
\end{verbatim}

As a matter of style, a \verb+static.+ section that is used for configuration should print what it
is doing, using \verb+---+ as a print prefix.

\section{Short syntax for scoping objects}

The usual dot-notation can be used for private, protected, and public variables (but not
static variables).

\begin{verbatim}
    # Public definition of X
    public.X = 1

    # Private definition of X
    private.X = 2

    # Prints:
    #    The public value of X is: 1
    #    The private value of X is: 2
    println(The public value of X is: $(public.X))
    println(The private value of X is: $(private.X))
\end{verbatim}

\section{Modular programming}

The scoping objects help provide a form of modularity.  When you write a new file or program,
explicit scoping declarations can be used to define an explicit interface for your code, and help
avoid name clashes with other parts of the project.  Variable definitions are public by default, but
you can control this with private definitions.

\begin{verbatim}
    # These variables are private to this file
    private. =
       FILES = foo1 foo2 foo3
       SUFFIX = .o
       OFILES = $(addsuffix $(SUFFIX), $(FILES))

    # These variables are public
    public. =
       CFLAGS += -g

    # Build the files with the -g option
    $(OFILES):
\end{verbatim}

%%%
%%% This stuff is not finished, so we are not advertizing its availability.
%%%

% \section{Policy directives for scoping}
%
% In some cases, you may wish to be careful that you don't accidentally shadow public variables that
% may be used in other parts of the project.  Running \Prog{omake} with the \verb+--strict+ option
% will do two things.
%
% \begin{enumerate}
% \item In \verb+--strict+ mode, all variables must be declared before being used.
% \item In \verb+--strict+ mode, all new definitions are \verb+protected+ by default.
% \end{enumerate}
%
% You can also control the scoping policy on a per-file basis with the \verb+policy+ directive.  The
% \verb+policy scope=<options>+ directive is used to specify scoping options.  The options are a
% comma-separated list of options as follows.
%
% \begin{description}
% \item[strict] Variables must be defined or declared before they are used.
% \item[relaxed] Variables do not have to be defined or declared before being used.
% \item[private] The default scope is private.
% \item[protected] The default scope is protected.
% \item[public] The default scope is public.
% \end{description}
%
% One common style is to declare all variables in a file as private except for specific variables that
% are public.  The \verb+declare+ operator declares variables without defining them.
%
% \begin{verbatim}
%     # strict: variables must be declared or
%     #    defined before being used.
%     # private: the default scope for variables
%     #    is private.
%     policy scope=strict,private
%
%     # CC and CFLAGS are public variables.
%     public. =
%         declare CC CFLAGS
%
%     # The following variables are private definitions
%     FILES = foo1 foo2 foo3
%     SUFFIX = .o
%     OFILES = $(addsuffix $(SUFFIX), $(FILES))
%
%     # Build the files with gcc -g
%     CC = gcc
%     CFLAGS += -g
%     $(OFILES):
%
%     # The following private function definition
%     # refers to a public variable X explicitly;
%     # it is not an error.
%     f() =
%        println(The public value of X is: $(public.X))
%
%     # The following private function definition
%     # produces an error because the variable X
%     # has not been defined.
%     f() =
%        println(The value of X is: $(X))
% \end{verbatim}
%
% When scoping is \verb+strict+, the \verb+open+ directive can be used to import the variables from
% another file.  Since the variables \verb+CC+ and \verb+CFLAGS+ are defined in the system build file
% \verb+build/C+, instead of declaring them explicitly, we can open the \verb+C+ file.
%
% \begin{verbatim}
%     # strict: variables must be declared or
%     #    defined before being used.
%     # private: the default scope for variables
%     #    is private.
%     policy scope=strict,private
%
%     # Import public variables from the file C
%     open build/C
%
%     # The following variables are private definitions
%     FILES = foo1 foo2 foo3
%     ...
% \end{verbatim}

% -*-
% Local Variables:
% Mode: LaTeX
% fill-column: 100
% TeX-master: "paper"
% TeX-command-default: "LaTeX/dvips Interactive"
% End:
% -*-

%%%%%%%%%%%%%%%%%%%%%%%%%%%%%%%%%%%%%%%%%%%%%%%%%%%%%%%%%%%%%%%%%%%%%%%%
% Description
%
\chapter{Variables and Naming}
\label{chapter:naming}
\cutname{omake-language-naming.html}

During evaluation, there are three different kinds of namespaces.  Variables can be \emph{private},
or they may refer to fields in the current \emph{this} object, or they can be part of the global
\emph{public} namespace.  The namespace can be specified directly by including an explicit qualifier
before the variable name.  The three namespaces are separate; a variable can be bound in one or more
simultaneously.

\begin{verbatim}
    # private namespace
    private.X = 1
    # current object
    this.X = 2
    # public, globally defined
    public.X = 3
\end{verbatim}

\section{private.}
\label{section:private}\index{private.}

The \verb+private.+ qualifier is used to define variables that are private to the current file/scope.
The values are not accessible outside the scope.  Private variables are statically (lexically) scoped.

\begin{verbatim}
    Obj. =
       private.X = 1

       print() =
          println(The value of X is: $X)

    # Prints:
    #    The private value of X is: 1
    Obj.print()

    # This is an error--X is private in Obj
    y = $(Obj.X)
\end{verbatim}

In addition, private definitions do not affect the global value of a variable.

\begin{verbatim}
   # The public value of x is 1
   x = 1

   # This object uses a private value of x
   Obj. =
       private.x = 2

       print() =
          x = 3
          println(The private value of x is: $x)
          println(The public value of x is: $(public.x))
          f()

   # Prints:
   #    The private value of x is: 3
   #    The public value of x is: 1
   Obj.print()
\end{verbatim}

Private variables have two additional properties.

\begin{enumerate}
\item Private variables are local to the file in which they are defined.
\item Private variables are not exported by the \verb+export+ directive, unless they are
  mentioned explicitly.

  \begin{verbatim}
       private. =
          FLAG = true

       section
          FLAG = false
          export

       # FLAG is still true
       section
          FLAG = false
          export FLAG

       # FLAG is now false
  \end{verbatim}
\end{enumerate}

\section{this.}
\index{this.}

The \verb+this.+ qualifier is used to define fields that are local to an object.
Object variables are dynamically scoped.

\begin{verbatim}
    X = 1
    f() =
       println(The public value of X is: $(X))

    # Prints:
    #    The public value of X is: 2
    section
       X = 2
       f()

    # X is a protected field in the object
    Obj. =
       this.X = 3

       print() =
          println(The value of this.X is: $(X))
          f()

    # Prints:
    #    The value of this.X is: 3
    #    The public value of X is: 1
    Obj.print()

    # This is legal, it defines Y as 3
    Y = $(Obj.X)
\end{verbatim}

In general, it is a good idea to define object variables as protected.  The resulting code is more
modular because variables in your object will not produce unexpected clashes with variables defined
in other parts of the project.

\section{public.}
\label{section:public}\index{public.}

The \verb+public.+ qualifier is used to specify public dynamically-scoped variables.  In the following
example, the \verb+public.+ definition specifies that the binding \verb+X = 4+ is to be dynamically
scoped.  Public variables \emph{are not} defined as fields of an object.

\begin{verbatim}
    X = 1
    f() =
       println(The public value of X is: $(X))

    # Prints:
    #    The public value of X is: 2
    section
       X = 2
       f()

    Obj. =
       protected.X = 3

       print() =
          println(The protected value of X is: $(X))
          public.X = 4
          f()

    # Prints:
    #    The protected value of X is: 3
    #    The public value of X is: 4
    Obj.print()
\end{verbatim}

\section{Qualified blocks}

If several qualified variables are defined simultaneously, a block form of qualifier can be defined.
The syntax is similar to an object definition, where the name of the object is the qualifier itself.
For example, the following program defines two private variables \verb+X+ and \verb+Y+.

\begin{verbatim}
    private. =
        X = 1
        Y = 2
\end{verbatim}
%
The qualifier specifies a default namespace for new definitions in the block.  The contents of the
block is otherwise completely general.

\begin{verbatim}
    private. =
        X = 1
        Y = 2
        public.Z = $(add $X, $Y)
        # Prints "The value of Z is 3"
        echo The value of Z is $Z
\end{verbatim}

\section{declare}
\label{section:declare}\index{declare}

When a variable name is unqualified, its namespace is determined by the most recent definition or
declaration that is in scope for that variable.  We have already seen this in the examples, where a
variable definition is qualified, but the subsequent uses are not qualified explicitly.  In the
following example, the first occurrence of \verb+$X+ refers to the \emph{private} definition,
because that is the most recent.  The public definition of \verb+X+ is still \verb+0+, but the
variable must be qualified explicitly.

\begin{verbatim}
    public.X = 0
    private.X = 1
    
    public.print() =
        println(The value of private.X is: $X)
        println(The value of public.X is: $(public.X))
\end{verbatim}
%
Sometimes it can be useful to declare a variable without defining it.  For example, we might have a
function that uses a variable \verb+X+ that is to be defined later in the program.  The
\verb+declare+ directive can be used for this.

\begin{verbatim}
    declare public.X

    public.print() =
        println(The value of X is $X)

    # Prints "The value of X is 2"
    X = 2
    print()
\end{verbatim}

Finally, what about variables that are used but not explicitly qualified?  In this case, the following rules are used.

\begin{itemize}
\item If the variable is a function parameter, it is private.
\item If the variable is defined in an object, it is qualified with \verb+this.+.
\item Otherwise, the variable is public.
\end{itemize}

% -*-
% Local Variables:
% Mode: LaTeX
% fill-column: 100
% TeX-master: "paper"
% TeX-command-default: "LaTeX/dvips Interactive"
% End:
% -*-

%
% Extra detail.
%
\chapter{Expressions and values}
\label{chapter:extra}
\cutname{omake-detail.html}

\Prog{omake} provides a full programming-language including many
system and IO functions.  The language is object-oriented -- everything is
an object, including the base values like numbers and strings.  However,
the \Prog{omake} language differs from other scripting languages in
three main respects.

\begin{itemize}
\item Scoping is dynamic.
\item Apart from IO, the language is entirely functional -- there is no
  assignment operator in the language.
\item Evaluation is normally eager -- that is, expressions are evaluated as soon
  as they are encountered.
\end{itemize}

To illustrate these features, we will use the \Cmd{osh}{1} omake program shell.
The \Cmd{osh}{1} program provides a toploop, where expressions can be entered
and the result printed.  \Cmd{osh}{1} normally interprets input as command text
to be executed by the shell, so in many cases we will use the \verb+value+
form to evaluate an expression directly.

\begin{verbatim}
    osh> 1
    *** omake error: File -: line 1, characters 0-1 command not found: 1
    osh> value 1
    - : "1" : Sequence
    osh> ls -l omake
    -rwxrwxr-x  1 jyh jyh 1662189 Aug 25 10:24 omake*
\end{verbatim}

\section{Dynamic scoping}

Dynamic scoping means that the value of a variable is determined by the most
recent binding of the variable in scope at runtime.  Consider the following
program.

\begin{verbatim}
    OPTIONS = a b c
    f() =
       println(OPTIONS = $(OPTIONS))
    g() =
       OPTIONS = d e f
       f()
\end{verbatim}

If \verb+f()+ is called without redefining the \verb+OPTIONS+ variable,
the function should print the string \verb+OPTIONS = a b c+.

In contrast, the function \verb+g()+ redefines the \verb+OPTIONS+
variable and evaluates \verb+f()+ in that scope, which now prints the
string \verb+OPTIONS = d e f+.

The body of \verb+g+ defines a local scope -- the redefinition of the
\verb+OPTIONS+ variable is local to \verb+g+ and does not persist
after the function terminates.

\begin{verbatim}
    osh> g()
    OPTIONS = d e f
    osh> f()
    OPTIONS = a b c
\end{verbatim}

Dynamic scoping can be tremendously helpful for simplifying the code
in a project.  For example, the \File{OMakeroot} file defines a set of
functions and rules for building projects using such variables as
\verb+CC+, \verb+CFLAGS+, etc.  However, different parts of a project
may need different values for these variables.  For example, we may
have a subdirectory called \verb+opt+ where we want to use the
\verb+-03+ option, and a subdirectory called \verb+debug+ where we
want to use the \verb+-g+ option.  Dynamic scoping allows us to redefine
these variables in the parts of the project without having to
redefine the functions that use them.

\begin{verbatim}
    section
       CFLAGS = -O3
       .SUBDIRS: opt
    section
       CFLAGS = -g
       .SUBDIRS: debug
\end{verbatim}

However, dynamic scoping also has drawbacks.  First, it can become
confusing: you might have a variable that is intended to be private,
but it is accidentally redefined elsewhere.  For example, you might
have the following code to construct search paths.

\begin{verbatim}
   PATHSEP = :
   make-path(dirs) =
      return $(concat $(PATHSEP), $(dirs))

   make-path(/bin /usr/bin /usr/X11R6/bin)
   - : "/bin:/usr/bin:/usr/X11R6/bin" : String
\end{verbatim}

However, elsewhere in the project, the \verb+PATHSEP+ variable is
redefined as a directory separator \verb+/+, and your function
suddenly returns the string \verb+/bin//usr/bin//usr/X11R6/bin+,
obviously not what you want.

The \verb+private+ block is used to solve this problem.  Variables
that are defined in a \verb+private+ block use static scoping -- that
is, the value of the variable is determined by the most recent
definition in scope in the source text.

\begin{verbatim}
   private
      PATHSEP = :
   make-path(dirs) =
      return $(concat $(PATHSEP), $(dirs))

   PATHSEP = /
   make-path(/bin /usr/bin /usr/X11R6/bin)
   - : "/bin:/usr/bin:/usr/X11R6/bin" : String
\end{verbatim}

\section{Functional evaluation}

Apart from I/O, \Prog{omake} programs are entirely functional.  This has two parts:

\begin{itemize}
\item There is no assignment operator.
\item Functions are values, and may be passed as arguments, and returned from
      functions just like any other value.
\end{itemize}

The second item is straightforward.  For example, the following program defines
an increment function by returning a function value.

\begin{verbatim}
   incby(n) =
      g(i) =
         return $(add $(i), $(n))
      return $(g)

   f = $(incby 5)

   value $(f 3)
   - : 8 : Int
\end{verbatim}

The first item may be the most confusing initially.  Without assignment, how is
it possible for a subproject to modify the global behavior of the project?  In fact,
the omission is intentional.  Build scripts are much easier to write when there
is a guarantee that subprojects do not interfere with one another.

However, there are times when a subproject needs to propagate
information back to its parent object, or when an inner scope needs to
propagate information back to the outer scope.

\section{Exporting the environment}
\label{section:export}\index{export}
The \verb+export+ directive can be used to propagate all or part of an inner scope back to its
parent.  The \verb+export+ directive should be the last statement in a block.  If used without
arguments, the entire scope is propagated back to the parent; otherwise the arguments specify which
part of the environment to propagate.  The most common usage is to export the definitions in a
conditional block.  In the following example, the variable \verb+B+ is bound to 2 after the
conditional.  The \verb+A+ variable is not redefined.

\begin{verbatim}
    if $(test)
       A = 1
       B = $(add $(A), 1)
       export B
    else
       B = 2
       export
\end{verbatim}

If the \verb+export+ directive is used without an argument, all of the following is exported:
\begin{itemize}
\item The values of all the dynamically scoped variables (as described in
Section~\ref{section:public}).
\item The current working directory.
\item The current Unix environment.
\item The current implicit rules and implicit dependencies (see also
Section~\ref{section:implicit-scoping}).
\item The current set of ``phony'' target declarations (see Sections~\ref{target:.PHONY}
and~\ref{section:PHONY-scoping}).
\end{itemize}

If the \verb+export+ directive is used with an argument, the argument expression is evaluated
and the resulting value is interpreted as follows:
\begin{itemize}
\item If the value is empty, everything is exported, as described above.
\item If the value represents a environment (or a partial environment) captured using the
\verb+export+ function (Section~\ref{fun:export}), then the corresponding environment or partial
environment is exported.
\item Otherwise, the value must be a sequence of strings specifying which items are to be propagated
back. The following strings have special meaning:
\begin{itemize}
\item \index{.RULE}\verb+.RULE+ --- implicit rules and implicit dependencies.
\item \index{.PHONY}\verb+.PHONY+ --- the set of ``phony'' target declarations.
\end{itemize}
All other strings are interpreted as names of the variables that need to be propagated back.
\end{itemize}

For example, in the following (somewhat artificial) example, the variables \verb+A+ and \verb+B+
will be exported, and the implicit rule will remain in the environment after the section ends, but
the variable \verb+TMP+ and the target \verb+tmp_phony+ will remain unchanged.

\begin{verbatim}
WANT_TO_EXPORT[] = A B .RULE
section
   A = 1
   B = 2
   TMP = $(add $(A), $(B))

   .PHONY: tmp_phony

   tmp_phony:
      prepare_foo

   %.foo: %.bar tmp_phony
      compute_foo $(TMP) $< $@
   export $(WANT_TO_EXPORT)
\end{verbatim}

\section{Eager evaluation}
\label{section:eager}

Evaluation in \Prog{omake} is eager.  That is, expressions are evaluated as soon as they are
encountered by the evaluator.  One effect of this is that the right-hand-side of a variable
definition is expanded when the variable is defined.

\begin{verbatim}
    osh> A = 1
    - : "1"
    osh> A = $(A)$(A)
    - : "11"
\end{verbatim}

In the second definition, \verb+A = $(A)$(A)+, the right-hand-side is evaluated first, producing the
sequence \verb+11+.  Then the variable \verb+A+ is \emph{redefined} as the new value.  When combined
with dynamic scoping, this has many of the same properties as conventional imperative programming.

\begin{verbatim}
    osh> A = 1
    - : "1"
    osh> printA() =
        println($"A = $A")
    osh> A = $(A)$(A)
    - : "11"
    osh> printA()
    11
\end{verbatim}

In this example, the print function is defined in the scope of \verb+A+.  When it is called on
the last line, the dynamic value of \verb+A+ is \verb+11+, which is what is printed.

However, dynamic scoping and imperative programming should not be confused.  The following example
illustrates a difference.  The second \verb+printA+ is not in the scope of the definition
\verb+A = x$(A)$(A)x+, so it prints the original value, \verb+1+.

\begin{verbatim}
    osh> A = 1
    - : "1"
    osh> printA() =
        println($"A = $A")
    osh> section
             A = x$(A)$(A)x
             printA()
    x11x
    osh> printA()
    1
\end{verbatim}

See also Section~\ref{section:lazy} for further ways to control the evaluation order through the use
of ``lazy'' expressions.

\section{Objects}

\Prog{omake} is an object-oriented language.  Everything is an object, including
base values like numbers and strings.  In many projects, this may not be so apparent
because most evaluation occurs in the default toplevel object, the \verb+Pervasives+
object, and few other objects are ever defined.

However, objects provide additional means for data structuring, and in some cases
judicious use of objects may simplify your project.

Objects are defined with the following syntax.  This defines \verb+name+
to be an object with several methods an values.

\begin{verbatim}
    name. =                     # += may be used as well
       extends parent-object    # optional
       class class-name         # optional

       # Fields
       X = value
       Y = value

       # Methods
       f(args) =
          body
       g(arg) =
          body
\end{verbatim}

An \verb+extends+ directive specifies that this object inherits from
the specified \verb+parent-object+.  The object may have any number of
\verb+extends+ directives.  If there is more than on \verb+extends+
directive, then fields and methods are inherited from all parent
objects.  If there are name conflicts, the later definitions override
the earlier definitions.

The \verb+class+ directive is optional.  If specified, it defines a name
for the object that can be used in \verb+instanceof+ operations, as well
as \verb+::+ scoping directives discussed below.

The body of the object is actually an arbitrary program.  The
variables defined in the body of the object become its fields, and the
functions defined in the body become its methods.

\section{Field and method calls}

The fields and methods of an object are named using \verb+object.name+ notation.
For example, let's define a one-dimensional point value.

\begin{verbatim}
   Point. =
      class Point

      # Default value
      x = $(int 0)

      # Create a new point
      new(x) =
         x = $(int $(x))
         return $(this)

      # Move by one
      move() =
         x = $(add $(x), 1)
         return $(this)

   osh> p1 = $(Point.new 15)
   osh> value $(p1.x)
   - : 15 : Int

   osh> p2 = $(p1.move)
   osh> value $(p2.x)
   - : 16 : Int
\end{verbatim}

The \verb+$(this)+ variable always represents the current object.
The expression \verb+$(p1.x)+ fetches the value of the \verb+x+ field
in the \verb+p1+ object.  The expression \verb+$(Point.new 15)+
represents a method call to the \verb+new+ method of the \verb+Point+
object, which returns a new object with 15 as its initial value.  The
expression \verb+$(p1.move)+ is also a method call, which returns a
new object at position 16.

Note that objects are functional --- it is not possible to modify the fields
or methods of an existing object in place.  Thus, the \verb+new+ and \verb+move+
methods return new objects.

\section{Method override}

Suppose we wish to create a new object that moves by 2 units, instead of
just 1.  We can do it by overriding the \verb+move+ method.

\begin{verbatim}
   Point2. =
      extends $(Point)

      # Override the move method
      move() =
         x = $(add $(x), 2)
         return $(this)

   osh> p2 = $(Point2.new 15)
   osh> p3 = $(p2.move)
   osh> value $(p3.x)
   - : 17 : Int
\end{verbatim}

However, by doing this, we have completely replaced the old \verb+move+ method.

\section{Super calls}

Suppose we wish to define a new \verb+move+ method that just calls the old one twice.
We can refer to the old definition of move using a super call, which uses the notation
\verb+$(classname::name <args>)+.  The \verb+classname+ should be the name of the
superclass, and \verb+name+ the field or method to be referenced.  An alternative
way of defining the \verb+Point2+ object is then as follows.

\begin{verbatim}
   Point2. =
      extends $(Point)

      # Call the old method twice
      move() =
         this = $(Point::move)
         return $(Point::move)
\end{verbatim}

Note that the first call to \verb+$(Point::move)+ redefines the
current object (the \verb+this+ variable).  This is because the method
returns a new object, which is re-used for the second call.

% -*-
% Local Variables:
% Mode: LaTeX
% fill-column: 100
% TeX-master: "paper"
% TeX-command-default: "LaTeX/dvips Interactive"
% End:
% -*-
% vim:tw=100:fo=tcq:


%
% Some examples.
%
\chapter{Additional language examples}
\label{chapter:language-examples}
\cutname{omake-language-examples.html}

In this section, we'll explore the core language through a series of examples (examples of the build
system are the topic of the Chapter~\ref{chapter:build-examples}).

For most of these examples, we'll use the \verb+osh+ command interpreter.  For simplicity, the
values printed by \verb+osh+ have been abbreviated.

\section{Strings and arrays}

The basic OMake values are strings, sequences, and arrays of values.  Sequences are like arrays of
values separated by whitespace; the sequences are split on demand by functions that expect arrays.

\begin{verbatim}
   osh> X = 1 2
   - : "1 2" : Sequence
   osh> addsuffix(.c, $X)
   - : <array 1.c 2.c> : Array
\end{verbatim}

Sometimes you want to define an array explicitly.  For this, use the \verb+[]+ brackets after the
variable name, and list each array entry on a single indented line.

\begin{verbatim}
   osh> A[] =
           Hello world
           $(getenv HOME)
   - : <array "Hello world" "/home/jyh"> : Array
\end{verbatim}

One central property of arrays is that whitespace in the elements is significant.  This can be
useful, especially for filenames that contain whitespace. 

\begin{verbatim}
   # List the current files in the directory
    osh> ls -Q
    "fee"  "fi"  "foo"  "fum"
    osh> NAME[] = 
            Hello world
    - : <array "Hello world"> : Array
    osh> touch $(NAME)
    osh> ls -Q
    "fee"  "fi"  "foo"  "fum"  "Hello world"
\end{verbatim}       

\section{Files and directories}

OMake projects usually span multiple directories, and different parts of the project execute
commands in different directories.  There is a need to define a location-independent name for a file
or directory.

This is done with the \verb+$(file <names>)+ and \verb+$(dir <names>)+ functions.

\begin{verbatim}
   osh> mkdir tmp
   osh> F = $(file fee)
   osh> section:
            cd tmp
            echo $F
   ../fee
   osh> echo $F
   fee
\end{verbatim}

Note the use of a \verb+section:+ to limit the scope of the \verb+cd+ command.  The section
temporarily changes to the \verb+tmp+ directory where the name of the file is \verb+../fee+.  Once
the section completes, we are still in the current directory, where the name of the file is
\verb+fee+.

One common way to use the file functions is to define proper file names in your project
\verb+OMakefile+, so that references within the various parts of the project will refer to the same
file.

\begin{verbatim}
    osh> cat OMakefile
    ROOT = $(dir .)
    TMP  = $(dir tmp)
    BIN  = $(dir bin)
    ...
\end{verbatim}

\section{Iteration, mapping, and foreach}

Most builtin functions operate transparently on arrays.

\begin{verbatim}
    osh> addprefix(-D, DEBUG WIN32)
    - : -DDEBUG -DWIN32 : Array
    osh> mapprefix(-I, /etc /tmp)
    - : -I /etc -I /tmp : Array
    osh> uppercase(fee fi foo fum)
    - : FEE FI FOO FUM : Array
\end{verbatim}

The \verb+mapprefix+ and \verb+addprefix+ functions are slightly different (the \verb+addsuffix+ and
\verb+mapsuffix+ functions are similar).  The \verb+addprefix+ adds the prefex to each array
element.  The \verb+mapprefix+ doubles the length of the array, adding the prefix as a new array
element before each of the original elements.

Even though most functions work on arrays, there are times when you will want to do it yourself.
The \verb+foreach+ function is the way to go.  The \verb+foreach+ function has two forms, but the
form with a body is most useful.  In this form, the function takes two arguments and a body.  The
second argument is an array, and the first is a variable.  The body is evaluated once for each
element of the array, where the variable is bound to the element.  Let's define a function to add 1
to each element of an array of numbers.

\begin{verbatim}
   osh> add1(l) =
            foreach(i, $l):
                add($i, 1)
   osh> add1(7 21 75)
   - : 8 22 76 : Array
\end{verbatim}

Sometimes you have an array of filenames, and you want to define a rule for each of them.  Rules are
not special, you can define them anywhere a statement is expected.  Say we want to write a function
that describes how to process each file, placing the result in the \verb+tmp/+ directory.

\begin{verbatim}
   TMP = $(dir tmp)

   my-special-rule(files) =
      foreach(name, $(files))
         $(TMP)/$(name): $(name)
            process $< > $@
\end{verbatim}

Later, in some other part of the project, we may decide that we want to use this function to process some files.

\begin{verbatim}
   # These are the files to process in src/lib
   MY_SPECIAL_FILES[] =
       fee.src
       fi.src
       file with spaces in its name.src
   my-special-rule($(MY_SPECIAL_FILES))
\end{verbatim}

The result of calling \verb+my-special-rule+ is
exactly the same as if we had written the following three rules explicitly.

\begin{verbatim}
    $(TMP)/fee.src: fee.src
        process fee > $@
    $(TMP)/fi.src: fi.src
        process fi.src > $@
    $(TMP)/$"file with spaces in its name.src": $"file with spaces in its name.src"
        process $< > $@
\end{verbatim}

Of course, writing these rules is not nearly as pleasant as calling the function.  The usual
properties of function abstraction give us the usual benefits.  The code is less redundant, and
there is a single location (the \verb+my-special-rule+ function) that defines the build rule.
Later, if we want to modify/update the rule, we need do so in only one location.

\section{Lazy expressions}
\label{section:lazy}

Lazy expressions are expressions that are not evaluated until their result is needed.  Some people,
including this author, frown on overuse of lazy expressions, mainly because it is difficult to know
when evaluation actually happens.  However, there are cases where they pay off.

One example comes from option processing.  Consider the specification of ``include'' directories on
the command line for a C compiler.  If we want to include files from /home/jyh/include and ../foo,
we specify it on the command line with the options \verb+-I/home/jyh/include -I../foo+.

Suppose we want to define a generic rule for building C files.  We could define a \verb+INCLUDES+
array to specify the directories to be included, and then define a generic implicit rule in our root
OMakefile.

\begin{verbatim}
    # Generic way to compile C files.
    CFLAGS = -g
    INCLUDES[] =
    %.o: %.c
       $(CC) $(CFLAGS) $(INCLUDES) -c $<

    # The src directory builds my_widget+ from 4 source files.
    # It reads include files from the include directory.
    .SUBDIRS: src
        FILES = fee fi foo fum
        OFILES = $(addsuffix .o, $(FILES))
        INCLUDES[] += -I../include
        my_widget: $(OFILES)
           $(CC) $(CFLAGS) -o $@ $(OFILES)
\end{verbatim}

But this is not quite right.  The problem is that INCLUDES is an array of options, not directories.
If we later wanted to recover the directories, we would have to strip the leading \verb+-I+ prefix,
which is a hassle.  Furthermore, we aren't using proper names for the directories.  The solution
here is to use a lazy expression.  We'll define INCLUDES as a directory array, and a new variable
\verb+PREFIXED_INCLUDES+ that adds the -I prefix.  The \verb+PREFIXED_INCLUDES+ is computed lazily,
ensuring that the value uses the most recent value of the INCLUDES variable.

\begin{verbatim}
    # Generic way to compile C files.
    CFLAGS = -g
    INCLUDES[] =
    PREFIXED_INCLUDES[] = $`(addprefix -I, $(INCLUDES))
    %.o: %.c
       $(CC) $(CFLAGS) $(PREFIXED_INCLUDES) -c $<

    # For this example, we define a proper name for the include directory
    STDINCLUDE = $(dir include)

    # The src directory builds my_widget+ from 4 source files.
    # It reads include files from the include directory.
    .SUBDIRS: src
        FILES = fee fi foo fum
        OFILES = $(addsuffix .o, $(FILES))
        INCLUDES[] += $(STDINCLUDE)
        my_widget: $(OFILES)
           $(CC) $(CFLAGS) -o $@ $(OFILES)
\end{verbatim}

Note that there is a close connection between lazy values and functions.  In the example above, we
could equivalently define \verb+PREFIXED_INCLUDES+ as a function with zero arguments.

\begin{verbatim}
    PREFIXED_INCLUDES() =
        addprefix(-I, $(INCLUDES))
\end{verbatim}

\section{Scoping and exports}

The OMake language is functional (apart from IO and shell commands).  This comes in two parts:
functions are first-class, and variables are immutable (there is no assignment operator).  The
latter property may seem strange to users used to GNU make, but it is actually a central point of
OMake.  Since variables can't be modified, it is impossible (or at least hard) for one part of the
project to interfere with another.

To be sure, pure functional programming can be awkward.  In OMake, each new indentation level
introduces a new scope, and new definitions in that scope are lost when the scope ends.  If OMake
were overly strict about scoping, we would wind up with a lot of convoluted code.

\begin{verbatim}
   osh> X = 1
   osh> setenv(BOO, 12)
   osh> if $(equal $(OSTYPE), Win32)
            setenv(BOO, 17)
            X = 2
   osh> println($X $(getenv BOO))
   1 12
\end{verbatim}

The \verb+export+ command presents a way out.  It takes care of ``exporting'' a value (or the entire
variable environment) from an inner scope to an outer one.

\begin{verbatim}
   osh> X = 1
   osh> setenv(BOO, 12)
   osh> if $(equal $(OSTYPE), Win32)
            setenv(BOO, 17)
            X = 2
            export
   osh> println($X $(getenv BOO))
   2 17
\end{verbatim}

Exports are especially useful in loop to export values from one iteration of a loop to the next.

\begin{verbatim}
   # Ok, let's try to add up the elements of the array
   osh>sum(l) =
           total = 0
           foreach(i, $l)
               total = $(add $(total), $i)
           value $(total)
   osh>sum(1 2 3)
   - : 0 : Int

   # Oops, that didn't work!
   osh>sum(l) =
           total = 0
           foreach(i, $l)
               total = $(add $(total), $i)
               export
           value $(total)
   osh>sum(1 2 3)
   - : 6 : Int
\end{verbatim}

A \verb+while+ loop is another form of loop, with an auto-export.

\begin{verbatim}
    osh>i = 0
    osh>total = 0
    osh>while $(lt $i, 10)
            total = $(add $(total), $i)
            i = $(add $i, 1)
    osh>println($(total))
    45
\end{verbatim}

\section{Shell aliases}

Sometimes you may want to define an \emph{alias}, an OMake command that masquerades as a real shell
command.  You can do this by adding your function as a method to the \verb+Shell+ object.

For an example, suppose we use the \verb+awk+ function to print out all the comments in a file.

\begin{verbatim}
    osh>cat comment.om
    # Comment function
    comments(filename) =
        awk($(filename))
        case $'^#'
            println($0)
    # File finished
    osh>include comment
    osh>comments(comment.om)
    # Comment function
    # File finished
\end{verbatim}

To add it as an alias, add the method (using += to preserve the existing entries in the Shell).

\begin{verbatim}
   osh>Shell. +=
           printcom(argv) =
               comments($(nth 0, $(argv)))
   osh>printcom comment.om > output.txt
   osh>cat output.txt
   # Comment function
   # File finished
\end{verbatim}

A shell command is passed an array of arguments \verb+argv+.  This does \emph{not} include the name
of the alias.

\section{Input/output redirection on the cheap}

As it turns out, scoping also provides a nice alternate way to perform redirection.  Suppose you
have already written a lot of code that prints to the standard output channel, but now you decide
you want to redirect it.  One way to do it is using the technique in the previous example: define
your function as an alias, and then use shell redirection to place the output where you want.

There is an alternate method that is easier in some cases.  The variables \verb+stdin+,
\verb+stdout+, and \verb+stderr+ define the standard I/O channels.  To redirect output, redefine
these variables as you see fit.  Of course, you would normally do this in a nested scope, so that
the outer channels are not affected.

\begin{verbatim}
    osh>f() =
            println(Hello world)
    osh>f()
    Hello world
    osh>section:
            stdout = $(fopen output.txt, w)
            f()
            close($(stdout))
    osh>cat output.txt
    Hello world
\end{verbatim}

This also works for shell commands.  If you like to gamble, you can try the following example.

\begin{verbatim}
    osh>f() =
            println(Hello world)
    osh>f()
    Hello world
    osh>section:
            stdout = $(fopen output.txt, w)
            f()
            cat output.txt
            close($(stdout))
    osh>cat output.txt
    Hello world
    Hello world
\end{verbatim}

% -*-
% Local Variables:
% Mode: LaTeX
% fill-column: 100
% TeX-master: "paper"
% TeX-command-default: "LaTeX/dvips Interactive"
% End:
% -*-

%
%
%
\chapter{Rules}
\label{chapter:rules}
\cutname{omake-rules.html}

Rules are used by \OMake{} to specify how to build files.  At its simplest, a rule has the following
form.

\begin{verbatim}
    <target>: <dependencies>
        <commands>
\end{verbatim}

The \verb+<target>+ is the name of a file to be built.  The \verb+<dependencies>+ are a list of
files that are needed before the \verb+<target>+ can be built.  The \verb+<commands>+ are a list of
indented lines specifying commands to build the target.  For example, the following rule specifies
how to compile a file \verb+hello.c+.

\begin{verbatim}
    hello.o: hello.c
        $(CC) $(CFLAGS) -c -o hello.o hello.c
\end{verbatim}

This rule states that the \File{hello.o} file depends on the \File{hello.c} file.  If the
\File{hello.c} file has changed, the command \verb+$(CC) $(CFLAGS) -c -o hello.o hello.c+ is to
be executed to update the target file \verb+hello.o+.

A rule can have an arbitrary number of commands.  The individual command lines are executed
independently by the command shell.  The commands do not have to begin with a tab, but they must be
indented from the dependency line.

In addition to normal variables, the following special variables may be used in the body of a rule.

% We use \char* even for printable chars for consistency.
\begin{itemize}
\item \verb+$*+\index{\char36\char42}\index[var]{\char42}: the target name, without a suffix.
\item \verb+$@+\index{\char36\char64}\index[var]{\char64}: the target name.
\item \verb+$^+\index{\char36\char94}\index[var]{\char94}: a list of the sources, in alphabetical order, with
duplicates removed.
\item \verb.$+.\index{\char36\char43}\index[var]{\char43}: all the sources, in the original order.
\item \verb+$<+\index{\char36\char60}\index[var]{\char60}: the first source.
\end{itemize}

For example, the above \verb+hello.c+ rule may be simplified as follows.

\begin{verbatim}
    hello.o: hello.c
        $(CC) $(CFLAGS) -c -o $@ $<
\end{verbatim}

Unlike normal values, the variables in a rule body are expanded lazily, and binding is dynamic.  The
following function definition illustrates some of the issues.

\begin{verbatim}
    CLibrary(name, files) =
        OFILES = $(addsuffix .o, $(files))

        $(name).a: $(OFILES)
            $(AR) cq $@ $(OFILES)
\end{verbatim}

This function defines a rule to build a program called \verb+$(name)+ from a list of \verb+.o+
files.  The files in the argument are specified without a suffix, so the first line of the function
definition defines a variable \verb+OFILES+ that adds the \verb+.o+ suffix to each of the file
names.  The next step defines a rule to build a target library \verb+$(name).a+ from the
\verb+$(OFILES)+ files. The expression \verb+$(AR)+ is evaluated when the function is called, and
the value of the variable \verb+AR+ is taken from the caller's scope (see also the section on
Scoping).

\section{Implicit rules}
\index{rules, implicit}

Rules may also be implicit.  That is, the files may be specified by wildcard patterns.
The wildcard character is \verb+%+.  For example, the following rule specifies a default
rule for building \verb+.o+ files.

\begin{verbatim}
    %.o: %.c
        $(CC) $(CFLAGS) -c -o $@ $*.c
\end{verbatim}

This rule is a template for building an arbitrary \verb+.o+ file from
a \verb+.c+ file.

By default, implicit rules are only used for the targets in the current
directory. However subdirectories included via the \verb+.SUBDIRS+ rules
inherit all the implicit rules that are in scope (see also the section on
Scoping).

\section{Bounded implicit rules}
\index{rules, bounded implicit}

Implicit rules may specify the set of files they apply to.  The following syntax is used.

\begin{verbatim}
    <targets>: <pattern>: <dependencies>
        <commands>
\end{verbatim}

For example, the following rule applies only to the files \verb+a.o+ and \verb+b.o+.

\begin{verbatim}
   a.o b.o: %.o: %.c
        $(CC) $(CFLAGS) -DSPECIAL -c $*.c
\end{verbatim}

\section{section}
\formref{section}

Frequently, the commands in a rule body are expressions to be evaluated by the shell.  \Prog{omake}
also allows expressions to be evaluated by \Prog{omake} itself.

The syntax of these ``computed rules'' uses the \verb+section+ expression.  The following rule uses
the \Prog{omake} IO functions to produce the target \verb+hello.c+.

\begin{verbatim}
    hello.c:
        section
            FP = fopen(hello.c, w)
            fprintln($(FP), $""#include <stdio.h> int main() { printf("Hello world\n"); }"")
            close($(FP))
\end{verbatim}

This example uses the quotation \verb+$""...""+ (see also Section~\ref{section:quotes}) to quote the text being
printed.  These quotes are not included in the output file.  The \verb+fopen+, \verb+fprintln+, and
\verb+close+ functions perform file IO as discussed in the IO section.

In addition, commands that are function calls, or special expressions, are interpreted correctly.
Since the \verb+fprintln+ function can take a file directly, the above rule can be abbreviated as
follows.

\begin{verbatim}
    hello.c:
       fprintln($@, $""#include <stdio.h> int main() { printf("Hello world\n"); }"")
\end{verbatim}

\section{section rule}
\formref{section rule}

Rules can also be computed using the \verb+section rule+ form, where a rule body is expected instead
of an expression.  In the following rule, the file \verb+a.c+ is copied onto the \verb+hello.c+ file
if it exists, otherwise \verb+hello.c+ is created from the file \verb+default.c+.

\begin{verbatim}
    hello.c:
        section rule
           if $(target-exists a.c)
              hello.c: a.c
                 cat a.c > hello.c
           else
              hello.c: default.c
                 cp default.c hello.c
\end{verbatim}

\section{Special dependencies}
\index{rule, options}

\subsection{:exists:}
\index{:exists:}

In some cases, the contents of a dependency do not matter, only whether the file exists or not.  In
this case, the \verb+:exists:+ qualifier can be used for the dependency.

\begin{verbatim}
    foo.c: a.c :exists: .flag
       if $(test -e .flag)
           $(CP) a.c $@
\end{verbatim}

\subsection{:effects:}
\index{:effects:}

Some commands produce files by side-effect.  For example, the
\Cmd{latex}{1} command produces a \verb+.aux+ file as a side-effect of
producing a \verb+.dvi+ file.  In this case, the \verb+:effects:+
qualifier can be used to list the side-effect explicitly.
\Prog{omake} is careful to avoid simultaneously running programs that
have overlapping side-effects.

\begin{verbatim}
    paper.dvi: paper.tex :effects: paper.aux
        latex paper
\end{verbatim}

\subsection{:value:}
\index{:value:}

The \verb+:value:+ dependency is used to specify that the rule execution depends on the value of an
expression.  For example, the following rule

\begin{verbatim}
    a: b c :value: $(X)
        ...
\end{verbatim}

specifies that ``a'' should be recompiled if the value of \verb+$(X)+ changes
(X does not have to be a filename).  This is intended to allow greater
control over dependencies.

In addition, it can be used instead of other kinds of dependencies. For example,
the following rule:

\begin{verbatim}
    a: b :exists: c
        commands
\end{verbatim}

is the same as

\begin{verbatim}
    a: b :value: $(target-exists c)
        commands
\end{verbatim}

Notes:
\begin{itemize}
\item The values are arbitrary (they are not limited to variables)
\item The values are evaluated at rule expansion time, so expressions
containing variables like \verb+$@+, \verb+$^+, etc are legal.
\end{itemize}

\section{.SCANNER rules}
\targetref{.SCANNER}

Scanner rules define a way to specify automatic dependency scanning.  A \verb+.SCANNER+ rule has the
following form.

\begin{verbatim}
    .SCANNER: target: dependencies
        commands
\end{verbatim}

The rule is used to compute additional dependencies that might be defined in the source files for
the specified target. The result of executing the scanner commands \emph{must} be a sequence of
dependencies in OMake format, printed to the standard output.  For example, on GNU systems the
\verb+gcc -MM foo.c+ produces dependencies for the file \verb+foo.c+ (based on \verb+#include+
information).

We can use this to specify a scanner for C files that adds the scanned dependencies for the
\verb+.o+ file.  The following scanner specifies that dependencies for a file, say \verb+foo.o+ can
be computed by running \verb+gcc -MM foo.c+.  Furthermore, \verb+foo.c+ is a dependency, so the
scanner should be recomputed whenever the \verb+foo.c+ file changes.

\begin{verbatim}
    .SCANNER: %.o: %.c
        gcc -MM $<
\end{verbatim}

Let's suppose that the command \verb+gcc -MM foo.c+ prints the following line.

\begin{verbatim}
    foo.o: foo.h /usr/include/stdio.h
\end{verbatim}

The result is that the files \verb+foo.h+ and \verb+/usr/include/stdio.h+ are considered to be
dependencies of \verb+foo.o+---that is, \verb+foo.o+ should be rebuilt if either of these files
changes.

This works, to an extent.  One nice feature is that the scanner will be re-run whenever the
\verb+foo.c+ file changes.  However, one problem is that dependencies in C are \emph{recursive}.
That is, if the file \verb+foo.h+ is modified, it might include other files, establishing further
dependencies.  What we need is to re-run the scanner if \verb+foo.h+ changes too.

We can do this with a \emph{value} dependency.  The variable \verb+$&+ is defined as the dependency
results from any previous scan.  We can add these as dependencies using the \verb+digest+ function,
which computes an MD5 digest of the files.

\begin{verbatim}
    .SCANNER: %.o: %.c :value: $(digest $&)
        gcc -MM $<
\end{verbatim}

Now, when the file \verb+foo.h+ changes, its digest will also change, and the scanner will be re-run
because of the value dependency (since \verb+$&+ will include \verb+foo.h+).

This still is not quite right.  The problem is that the C compiler uses a \emph{search-path} for
include files.  There may be several versions of the file \verb+foo.h+, and the one that is chosen
depends on the include path.  What we need is to base the dependencies on the search path.

The \verb+$(digest-in-path-optional ...)+ function computes the digest based on a search path,
giving us a solution that works.

\begin{verbatim}
    .SCANNER: %.o: %.c :value: $(digest-in-path-optional $(INCLUDES), $&)
       gcc -MM $(addprefix -I, $(INCLUDES)) $<
\end{verbatim}

The standard output of the scanner rules will be captured by OMake and is not allowed to contain any
content that OMake will not be able to parse as a dependency. The output is allowed to contain
dependency specifications for unrelated targets, however such dependencies will be ignored. The
scanner rules are allowed to produce arbitrary output on the standard error channel --- such output
will be handled in the same way as the output of the ordinary rules (in other words, it will be
presented to the user, when dictated by the \verb+--output-+$\ldots$ options enabled).

Additional examples of the \verb+.SCANNER+ rules can be found in Section~\ref{sec:scanner-exm}.

\subsection{Named scanners, and the :scanner: target}
\index{:scanner:}

Sometimes it may be useful to specify explicitly which scanner should be used in a rule.  For
example, we might compile \verb+.c+ files with different options, or (heaven help us) we may be
using both \verb+gcc+ and the Microsoft Visual C++ compiler \verb+cl+. In general, the target of a
\verb+.SCANNER+ is not tied to a particular target, and we may name it as we like.

\begin{verbatim}
    .SCANNER: scan-gcc-%.c: %.c :value: $(digest-in-path-optional $(INCLUDES), $&)
        gcc -MM $(addprefix -I, $(INCLUDES)) $<

    .SCANNER: scan-cl-%.c: %.c :value: $(digest-in-path-optional $(INCLUDES), $&)
        cl --scan-dependencies-or-something $(addprefix /I, $(INCLUDES)) $<
\end{verbatim}

The next step is to define explicit scanner dependencies.  The \verb+:scanner:+ dependency is used
for this.  In this case, the scanner dependencies are specified explicitly.

\begin{verbatim}
    $(GCC_FILES): %.o: %.c :scanner: scan-gcc-%c
        gcc ...

    $(CL_FILES): %.obj: %.c :scanner: scan-cl-%c
        cl ...
\end{verbatim}

Explicit \verb+:scanner:+ scanner specification may also be used to state that a single
\verb+.SCANNER+ rule should be used to generate dependencies for more than one target. For example,

\begin{verbatim}
    .SCANNER: scan-all-c: $(GCC_FILES) :value: $(digest-in-path-optional $(INCLUDES), $&)
        gcc -MM $(addprefix -I, $(INCLUDES)) $(GCC_FILES)

    $(GCC_FILES): %.o: %.c :scanner: scan-all-c
        ...
\end{verbatim}

The above has the advantage of only running \Prog{gcc} once and a disadvantage that when a single
source file changes, all the files will end up being re-scanned.

\subsection{Notes}

In most cases, you won't need to define scanners of your own.  The standard installation includes
default scanners (both explicitly and implicitly named ones) for C, OCaml, and \LaTeX{} files.

The \hypervarx{SCANNER_MODE}{SCANNER\_MODE} controls the usage of implicit scanner dependencies.

The explicit \verb+:scanner:+ dependencies reduce the chances of scanner mis-specifications. In
large complicated projects it might be a good idea to set \verb+SCANNER_MODE+ to \verb+error+ and
use only the named \verb+.SCANNER+ rules and explicit \verb+:scanner:+ specifications.

\section{.DEFAULT}
\targetref{.DEFAULT}

The \verb+.DEFAULT+ target specifies a target to be built by default
if \Prog{omake} is run without explicit targets.  The following rule
instructs \Prog{omake} to build the program \verb+hello+ by default

\begin{verbatim}
   .DEFAULT: hello
\end{verbatim}

\section{.SUBDIRS}
\targetref{.SUBDIRS}

The \verb+.SUBDIRS+ target is used to specify a set of subdirectories
that are part of the project.  Each subdirectory should have its own
\File{OMakefile}, which is evaluated in the context of the current
environment.

\begin{verbatim}
   .SUBDIRS: src doc tests
\end{verbatim}

This rule specifies that the \verb+OMakefile+s in each of the \verb+src+, \verb+doc+, and
\verb+tests+ directories should be read.

In some cases, especially when the \verb+OMakefile+s are very similar in a large number of
subdirectories, it is inconvenient to have a separate \verb+OMakefile+ for each directory.  If the
\verb+.SUBDIRS+ rule has a body, the body is used instead of the \verb+OMakefile+.

\begin{verbatim}
   .SUBDIRS: src1 src2 src3
      println(Subdirectory $(CWD))
      .DEFAULT: lib.a
\end{verbatim}

In this case, the \verb+src1+, \verb+src2+, and \verb+src3+ files do not need \verb+OMakefile+s.
Furthermore, if one exists, it is ignored.  The following includes the file if it exists.

\begin{verbatim}
   .SUBDIRS: src1 src2 src3
       if $(file-exists OMakefile)
          include OMakefile
       .DEFAULT: lib.a
\end{verbatim}

\section{.INCLUDE}
\targetref{.INCLUDE}

The \verb+.INCLUDE+ target is like the \verb+include+ directive, but it specifies a rule to build
the file if it does not exist.

\begin{verbatim}
   .INCLUDE: config
       echo "CONFIG_READ = true" > config

    echo CONFIG_READ is $(CONFIG_READ)
\end{verbatim}

You may also specify dependencies to an \verb+.INCLUDE+ rule.

\begin{verbatim}
   .INCLUDE: config: config.defaults
      cp config.defaults config
\end{verbatim}

A word of caution is in order here.  The usual policy is used for determining when the rule is
out-of-date.  The rule is executed if any of the following hold.

\begin{itemize}
\item the target does not exist,
\item the rule has never been executed before,
\item any of the following have changed since the last time the rule was executed,
\begin{itemize}
\item the target,
\item the dependencies,
\item the commands-text.
\end{itemize}
\end{itemize}

In some of the cases, this will mean that the rule is executed even if the target file already
exists.  If the target is a file that you expect to edit by hand (and therefore you don't want to
overwrite it), you should make the rule evaluation conditional on whether the target already exists.

\begin{verbatim}
   .INCLUDE: config: config.defaults
       # Don't overwrite my carefully hand-edited file
       if $(not $(file-exists config))
          cp config.defaults config
\end{verbatim}

\section{.PHONY}
\targetref{.PHONY}

A ``phony'' target is a target that is not a real file, but exists to collect a set of dependencies.
Phony targets are specified with the \verb+.PHONY+ rule.  In the following example, the
\verb+install+ target does not correspond to a file, but it corresponds to some commands that should
be run whenever the \verb+install+ target is built (for example, by running \verb+omake install+).

\begin{verbatim}
   .PHONY: install

   install: myprogram.exe
      cp myprogram.exe /usr/bin
\end{verbatim}

\section{Rule scoping}
\index{rule, scoping}

As we have mentioned before, \Prog{omake} is a \emph{scoped} language.  This provides great
flexibility---different parts of the project can define different configurations without interfering
with one another (for example, one part of the project might be compiled with \verb+CFLAGS=-O3+ and
another with \verb+CFLAGS=-g+).

But how is the scope for a target file selected?  Suppose we are building a file \verb+dir/foo.o+.
\Prog{omake} uses the following rules to determine the scope.

\begin{itemize}
\item First, if there is an \emph{explicit} rule for building \verb+dir/foo.o+ (a rule with no
wildcards), the context for that rule determines the scope for building the target.
\item Otherwise, the directory \verb+dir/+ must be part of the project.  This normally means that a
configuration file \verb+dir/OMakefile+ exists (although, see the \verb+.SUBDIRS+ section for
another way to specify the \verb+OMakefile+).  In this case, the scope of the target is the scope at
the end of the \verb+dir/OMakefile+.
\end{itemize}

To illustrate rule scoping, let's go back to the example of a ``Hello world'' program with two
files.  Here is an example \verb+OMakefile+ (the two definitions of \verb+CFLAGS+ are for
illustration).

\begin{verbatim}
    # The executable is compiled with debugging
    CFLAGS = -g
    hello: hello_code.o hello_lib.o
       $(CC) $(CFLAGS) -o $@ $+

    # Redefine CFLAGS
    CFLAGS += -O3
\end{verbatim}

In this project, the target \verb+hello+ is \emph{explicit}.  The scope of the \verb+hello+ target
is the line beginning with \verb+hello:+, where the value of \verb+CFLAGS+ is \verb+-g+.  The other
two targets, \verb+hello_code.o+ and \verb+hello_lib.o+ do not appear as explicit targets, so their
scope is at the end of the \verb+OMakefile+, where the \verb+CFLAGS+ variable is defined to be
\verb+-g -O3+.  That is, \verb+hello+ will be linked with \verb+CFLAGS=-g+ and the \verb+.o+ files
will be compiled with \verb+CFLAGS=-g -O3+.

We can change this behavior for any of the targets by specifying them as explicit targets.  For
example, suppose we wish to compile \verb+hello_lib.o+ with a preprocessor variable \verb+LIBRARY+.

\begin{verbatim}
    # The executable is compiled with debugging
    CFLAGS = -g
    hello: hello_code.o hello_lib.o
       $(CC) $(CFLAGS) -o $@ $+

    # Compile hello_lib.o with CFLAGS = -g -DLIBRARY
    section
        CFLAGS += -DLIBRARY
        hello_lib.o:

    # Redefine CFLAGS
    CFLAGS += -O3
\end{verbatim}

In this case, \verb+hello_lib.o+ is also mentioned as an explicit target, in a scope where
\verb+CFLAGS=-g -DLIBRARY+.  Since no rule body is specified, it is compiled using the usual
implicit rule for building \verb+.o+ files (in a context where \verb+CFLAGS=-g -DLIBRARY+).

\subsection{Scoping of implicit rules}
\label{section:implicit-scoping}

Implicit rules (rules containing wildcard patterns) are \emph{not} global, they follow the normal
scoping convention.  This allows different parts of a project to have different sets of implicit
rules.  If we like, we can modify the example above to provide a new implicit rule for building
\verb+hello_lib.o+.

\begin{verbatim}
    # The executable is compiled with debugging
    CFLAGS = -g
    hello: hello_code.o hello_lib.o
       $(CC) $(CFLAGS) -o $@ $+

    # Compile hello_lib.o with CFLAGS = -g -DLIBRARY
    section
        %.o: %.c
            $(CC) $(CFLAGS) -DLIBRARY -c $<
        hello_lib.o:

    # Redefine CFLAGS
    CFLAGS += -O3
\end{verbatim}

In this case, the target \verb+hello_lib.o+ is built in a scope with a new implicit rule for
building \verb+%.o+ files.  The implicit rule adds the \verb+-DLIBRARY+ option.  This implicit rule
is defined only for the target \verb+hello_lib.o+; the target \verb+hello_code.o+ is built as
normal.

\subsection{Scoping of .SCANNER rules}
\index{.SCANNER}\index[target]{.SCANNER}

Scanner rules are scoped the same way as normal rules.  If the \verb+.SCANNER+ rule is explicit
(containing no wildcard patterns), then the scope of the scan target is the same as the the rule.
If the \verb+.SCANNER+ rule is implicit, then the environment is taken from the \verb+:scanner:+
dependency.

\begin{verbatim}
    # The executable is compiled with debugging
    CFLAGS = -g
    hello: hello_code.o hello_lib.o
       $(CC) $(CFLAGS) -o $@ $+

    # scanner for .c files
    .SCANNER: scan-c-%.c: %.c
       $(CC) $(CFLAGS) -MM $<

    # Compile hello_lib.o with CFLAGS = -g -DLIBRARY
    section
        CFLAGS += -DLIBRARY
        hello_lib.o: hello_lib.c :scanner: scan-c-hello_lib.c
           $(CC) $(CFLAGS) -c $<

    # Compile hello_code.c with CFLAGS = -g -O3
    section
        CFLAGS += -O3
        hello_code.o: hello_code.c :scanner: scan-c-hello_code.c
           $(CC) $(CFLAGS) -c $<
\end{verbatim}

Again, this is for illustration---it is unlikely you would need to write a complicated configuration
like this!  In this case, the \verb+.SCANNER+ rule specifies that the C-compiler should be called
with the \verb+-MM+ flag to compute dependencies.  For the target \verb+hello_lib.o+, the scanner
is called with \verb+CFLAGS=-g -DLIBRARY+, and for \verb+hello_code.o+ it is called with
\verb+CFLAGS=-g -O3+.

\subsection{Scoping for .PHONY targets}
\label{section:PHONY-scoping}\index{.PHONY}\index[target]{.PHONY}

Phony targets (targets that do not correspond to files) are defined with a \verb+.PHONY:+ rule.
Phony targets are scoped as usual.  The following illustrates a common mistake, where the
\verb+.PHONY+ target is declared \emph{after} it is used.

\begin{verbatim}
    # !!This example is broken!!
    all: hello

    hello: hello_code.o hello_lib.o
        $(CC) $(CFLAGS) -o $@ $+

    .PHONY: all
\end{verbatim}

This doesn't work as expected because the \verb+.PHONY+ declaration occurs too late.  The proper way
to write this example is to place the \verb+.PHONY+ declaration first.

\begin{verbatim}
    # Phony targets must be declared before being used
    .PHONY: all

    all: hello

    hello: hello_code.o hello_lib.o
        $(CC) $(CFLAGS) -o $@ $+
\end{verbatim}

Phony targets are passed to subdirectories.  As a practical matter, it is wise to declare all
\verb+.PHONY+ targets in your root \verb+OMakefile+, before any \verb+.SUBDIRS+.  This will ensure
that 1) they are considered as phony targets in each of the sbdirectories, and 2) you can build them
from the project root.

\begin{verbatim}
    .PHONY: all install clean

    .SUBDIRS: src lib clib
\end{verbatim}

\section{Pathnames in rules}

In rules, the targets and dependencies are first translated to \emph{file} values (as in the
\hyperfun{file}).  They are then translated to strings for the command line.
This can cause some unexpected behavior.  In the following example, the \hyperfun{absname}
is the absolute pathname for the file \verb+a+, but the rule still prints
the relative pathname.

\begin{verbatim}
    .PHONY: demo
    demo: $(absname a)
        echo $<

    # omake demo
    a
\end{verbatim}

There is arguably a good reason for this.  On Win32 systems, the \verb+/+ character is viewed as an
``option specifier.''  The pathname separator is the \verb+\+ character.  \OMake{} translates the
filenames automatically so that things work as expected on both systems.

\begin{verbatim}
   demo: a/b
       echo $<

   # omake demo (on a Unix system)
   a/b
   # omake demo (on a Win32 system)
   a\b
\end{verbatim}

Sometimes you may wish that target strings to be passed literally to the commands in the rule.
One way to do this is to specify them literally.

\begin{verbatim}
    SRC = a/b $(absname c/d)
    demo: $(SRC)
        echo $(SRC)

    # omake demo (on a Win32 system)
    a/b c:\...\c\d
\end{verbatim}

Alternately, you might wish that filenames be automatically expanded to absolute pathnames.  For
example, this might be useful when parsing the \OMake{} output to look for errors.  For this, you can
use the \verb+--absname+ option (Section~\ref{option:--absname}).  If you call \verb+omake+ with the
\verb+--absname+ option, all filenames will be exapnded to absolute names.

\begin{verbatim}
    # omake --absname demo (on a Unix system)
    /home/.../a/b /home/.../c/d
\end{verbatim}

Alternately, the \verb+--absname+ option is scoped.  If you want to use it for only a few rules, you
can use the \hyperfun{OMakeFlags} to control how it is applied.
    
\begin{verbatim}
   section
      OMakeFlags(--absname)
      demo: a
          echo $<

   # omake demo
   /home/.../a
\end{verbatim}

\textbf{N.B.} The \verb+--absname+ option is currently an experimental feature.

% -*-
% Local Variables:
% Mode: LaTeX
% fill-column: 100
% TeX-master: "paper"
% TeX-command-default: "LaTeX/dvips Interactive"
% End:
% vim:tw=100:fo=tcq:
% -*-

\input{../tex/omake-base}
\input{../tex/omake-system}
\input{../tex/omake-pervasives}
{\renewcommand\idsection[1]{\subsubsection{#1}}\input{../tex/omake-root}}
\input{../tex/omake-autoconf}
%
%
%

\section{The OSH shell}

OMake also includes a standalone command-line interpreter \Prog{osh} that can be used as an
interactive shell.  The shell uses the same syntax, and provides the same features on all platforms
\Prog{omake} supports, including Win32.

\subsection{Startup}

On startup, \Prog{osh} reads the file \verb+~/.oshrc+ if it exists.  The syntax of this file is the
same as an \Prog{OMakefile}.  The following additional variables are significant.

\begin{description}
\item[prompt]  The \verb+prompt+ variable specifies the command-line prompt.
It can be a simple string.

\begin{verbatim}
    prompt = osh>
\end{verbatim}

Or you may choose to define it as a function of no arguments.

\begin{verbatim}
    prompt() =
        return $"<$(USER):$(HOST) $(homename $(CWD))>"
\end{verbatim}

An example of the latter prompt is as follows.

\begin{verbatim}
    <jyh:kenai.yapper.org ~>cd links/omake
    <jyh:kenai.yapper.org ~/links/omake>
\end{verbatim}

\item[ignoreeof]
   If the \verb+ignoreeof+ is \verb+true+, then \verb+osh+ will not exit on
   a terminal end-of-file (usually \verb+^D+ on Unix systems).
\end{description}

\subsection{Aliases}

Command aliases are defined by adding functions to the \verb+Shell.+ object.  The following alias
adds the \verb+-AF+ option to the \verb+ls+ command.

\begin{verbatim}
    Shell. +=
       ls(argv) =
          "ls" -AF $(argv)
\end{verbatim}

Quoted commands do not undergo alias expansion.  The quotation \verb+"ls"+ prevents the alias from
being recursive.

\subsection{Interactive syntax}

The interactive syntax in \verb+osh+ is the same as the syntax of an \verb+OMakefile+, with one
exception in regard to indentation.  The line before an indented block must have a colon at the end
of the line.  A block is terminated with a \verb+.+ on a line by itself, or \verb+^D+.  In the
following example, the first line \verb+if true+ has no body, because there is no colon.

\begin{verbatim}
   # The following if has no body
   osh>if true
   # The following if has a body
   osh>if true:
   if>       if true:
   if>          println(Hello world)
   if>          .
   Hello world
\end{verbatim}

Note that \verb+osh+ makes some effort to modify the prompt while in an indented body, and it
auto-indents the text.

The colon signifier is also allowed in files, although it is not required.

\subsection{See also}

See Section \href{omake-shell.html}{omake-shell} for more information on the shell language,
and Section \href{omake-system.html}{omake-system} for more information on job control.

% -*-
% Local Variables:
% Mode: LaTeX
% fill-column: 100
% TeX-master: "paper"
% TeX-command-default: "LaTeX/dvips Interactive"
% End:
% -*-

\appendix
%%%%%%%%%%%%%%%%%%%%%%%%%%%%%%%%%%%%%%%%%%%%%%%%%%%%%%%%%%%%%%%%%%%%%%%%
% Usage
%
\section{Synopsis}

\Prog{omake}
    \oOpt{-k}
    \oOptArg{-j}{count}
    \oOpt{-n}
    \oOpt{-s} \oOpt{-S}
    \oOpt{-p}
    \oOpt{-P}
    \oOpt{-w}
    \oOpt{-t}
    \oOpt{-u}
    \oOpt{-U}
    \oOpt{-R}
    \oOpt{--project}
    \oOpt{--progress} \oOpt{--no-progress}
    \oOpt{--print-status} \oOpt{--no-print-status}
    \oOpt{--print-exit} \oOpt{--no-print-exit}
    \oOpt{--print-dependencies}
    \oOptArg{--show-dependencies}{\ target}
    \oOpt{--force-dotomake}
    \oOptArg{--dotomake}{\ dir}
    \oOpt{--flush-includes}
    \oOpt{--configure}
    \oOpt{--install}
    \oOpt{--install-all}
    \oOpt{--install-force}
    \oOpt{--version}
    \oArg{filename...}
    \oOpt{var-definition...}

%%%%%%%%%%%%%%%%%%%%%%%%%%%%%%%%%%%%%%%%%%%%%%%%%%%%%%%%%%%%%%%%%%%%%%%%
% Options
%
\section{Command-line options}

\begin{Description}\setlength{\itemsep}{0cm}
\item[\Opt{-k}] Do not abort when a build command fails;
continue to build as much of the project as possible.

\item[\Opt{-n}] Print the commands that would be executed, but do no execute them.
This can be used to see what would happen if the project were to be built.

\item[\Opt{-s}] Do not print commands as they are executed (be ``silent'').

\item[\Opt{-S}] Do not print commands as they are executed \emph{unless} they produce output.

\item[\Opt{--progress}] Print a progress indicator.
This is normally used with the \Opt{-s} or \Opt{-S} options.

\item[\Opt{--no-progress}] Do not print a progress indicator (default).

\item[\Opt{--print-exit}] Print termination codes when commands complete.

\item[\Opt{--no-print-exit}] Do not print termination codes when commands complete (default).

\item[\Opt{-w}] Print directory information in \Prog{make} format as commands are executed.
This is mainly useful for editors that expect \Prog{make}-style
directory information for determining the location of errors.

\item[\Opt{-p}] Watch the filesystem for changes, and continue the build until it succeeds.  If this
option is specified, \Prog{omake} will restart the build whenever source files are modified.

\item[\Opt{-P}] Watch the filesystem for changes forever.  If this option is specified, \Prog{omake}
will restart the build whenever source files are modified.

\item[\Opt{-R}] Ignore the current directory and build the project from its root directory.  When
\Prog{omake} is run in a subdirectory of a project, it normally builds files within the current
directory and its subdirectories.  If the \Opt{-R} option is specified, the build is performed as if
\Prog{omake} were run in the project root.

\item[\Opt{-t}] Update the \Prog{omake} database to force the project to be considered up-to-date.

\item[\Opt{-U}] Do not trust cached build information.  This will force the entire project to be rebuilt.

\item[\Opt{--depend}] Do not trust cached dependency information.  This will force files to be rescanned
for dependency information.

\item[\Opt{--configure}] Re-run \verb+static.\+ sections of the included omake files, instead of
trusting the cached results.

\item[\oOpt{--force-dotomake}] Always use the \verb+$HOME/.omake+ for the \verb+.omc+ cache files.

\item[\oOptArg{--dotomake}{\ dir}] Use the specified directory instead of the \verb+$HOME/.omake+
for the placement of the \verb+.omc+ cache files.

\item[\OptArg{-j}{count}] Run multiple build commands in parallel.  The \Arg{count} specifies a
bound on the number of commands to run simultaneously.  In addition, the count may specify servers
for remote execution of commands in the form \verb+server=count+.  For example, the option
\verb+-j 2:small.host.org=1:large.host.org=4+ would specify that up to 2 jobs can be executed
locally, 1 on the server \verb+small.host.org+ and 4 on \verb+large.host.org+.  Each remote server
must use the same filesystem location for the project.

Remote execution is currently an experimental feature.  Remote filesystems like NFS do not provide
adequate file consistency for this to work.

\item[\Opt{--print-dependencies}] Print dependency information for the targets on the command line.

\item[\OptArg{--show-dependencies}{\ target}] Print dependency information \emph{if} the \verb+target+ is built.

\item[\Opt{--install}] Install default files \File{OMakefile} and \File{OMakeroot} into the current
  directory.  You would typically do this to start a project in the current directory.

\item[\Opt{--install-all}] In addition to installing files \File{OMakefile} and \File{OMakeroot},
  install default \File{OMakefile}s into each subdirectory of the current directory.
  \Cmd{cvs}{1} rules are used for filtering the subdirectory list.  For example, \File{OMakefile}s
  are not copied into directories called \verb+CVS+, \verb+RCCS+, etc.

\item[\Opt{--install-force}] Normally, \Prog{omake} will prompt before it overwrites any
  existing \File{OMakefile}.  If this option is given, all files are forcibly overwritten
  without prompting.

\item[\Opt{var-definition}] \Prog{omake} variables can also be defined on the command
  line in the form \verb+name=value+.  For example, the \verb+CFLAGS+ variable might be defined
  on the command line with the argument \verb+CFLAGS="-Wall -g"+.
\end{Description}

In addition, \Prog{omake} supports a number of debugging flags on the command line. Run
\verb+omake --help+ to get a summary of these flags.

% -*-
% Local Variables:
% Mode: LaTeX
% fill-column: 100
% TeX-master: "paper"
% TeX-command-default: "LaTeX/dvips Interactive"
% End:
% vim:tw=100:fo=tcq:
% -*-

%
% A more formal description of the grammar.
%
\chapter{\OMake{} grammar}
\label{chapter:grammar}
\cutname{omake-grammar.html}

\section{\OMake{} lexical conventions}

The \OMake{} language is based on the language for GNU/BSD make, where there are few lexical
conventions.  Strictly speaking, there are no keywords, and few special symbols.

\subsection{Comments}

Comments begin with the \verb+#+ character and continue to the end-of-line.
Text within a comment is unrestricted.

Examples.

\begin{verbatim}
   # This is a comment
   # This $comment contains a quote " character
\end{verbatim}

\subsection{Special characters}

The following characters are special in some contexts.

\begin{verbatim}
   $    (    )    ,    .   =    :    "    '    `    \    #
\end{verbatim}

\begin{itemize}
\item \verb+$+ is used to denote a variable reference, or function application.
\item Parentheses \verb+)+, \verb+(+ are argument deliminters.
\item The command \verb+,+ is an argument separator.
\item The period symbol \verb+.+ is a name separator.
\item The equality symbol \verb+=+ denotes a definition.
\item The colon symbol \verb+:+ is used to denote rules, and (optionally) to indicate
   that an expression is followed by an indented body.
\item The quotation symbols \verb+"+ and \verb+'+ delimit character strings.
\item The symbol \verb+#+ is the first character of a constant.
\item The escape symbol \verb+\+ is special \emph{only when} followed by another special
   character.  In this case, the special status of the second character is removed,
   and the sequence denotes the second character.  Otherwise, the \verb+\+ is not special.

   Examples:

   \begin{itemize}
   \item \verb+\$+: the \verb+$+ character (as a normal character).
   \item \verb+\#+: the \verb+#+ character (as a normal character).
   \item \verb+\\+: the \verb+\+ character (as a normal character).
   \item \verb+c\:\Windows\moo\#boo+: the string \verb+c:\Windows\moo#boo+.
   \end{itemize}
\end{itemize}

\subsection{Identifiers}

Identifiers (variable names) are drawn from the ASCII alphanumeric characters as well as \verb+_+,
\verb+-+, \verb+~+, \verb+@+.  Case is significant; the following identifiers are distinct:
\verb+FOO+, \verb+Foo+, \verb+foo+.  The identifier may begin with any of the valid characters,
including digits.

Using \verb+egrep+ notation, the regular expression for identifiers is defined as follows.

\begin{verbatim}
    identifier ::= [-@~_A-Za-z0-9]+
\end{verbatim}

The following are legal identifiers.

\begin{verbatim}
    Xyz    hello_world    seventy@nine
    79-32  Gnus~Gnats     CFLAGS
\end{verbatim}

The following are not legal identifiers.

\begin{verbatim}
    x+y    hello&world
\end{verbatim}

\subsection{Command identifiers}

The following words have special significance when they occur as the \emph{first} word
of a program line.  They are not otherwise special.

\begin{verbatim}
    case     catch  class    declare    default
    do       else   elseif   export     extends
    finally  if     import   include    match
    open     raise  return   section    switch
    try      value  when     while
\end{verbatim}

\subsection{Variable references}

A variable reference is denoted with the \verb+$+ special character followed by an identifier.  If
the identifier name has more than one character, it must be enclosed in parentheses.  The
parenthesized version is most common.  The following are legal variable references.

\begin{verbatim}
    $(Xyz)    $(hello_world)   $(seventy@nine)
    $(79-32)  $(Gnus~Gnats)    $(CFLAGS)
\end{verbatim}

Single-character references also include several additional identifiers, including \verb+&*<^?][+.
The following are legal single-character references.

\begin{verbatim}
   $@   $&   $*   $<   $^   $+   $?   $[   $]
   $A   $_   $a   $b   $x   $1   $2   $3
\end{verbatim}

Note that a non-parenthesized variable reference is limited to a single character, even if it is
followed by additional legal identifier charqcters.  Suppose the value of the \verb+$x+ variable is
17.  The following examples illustrate evaluation.

\begin{verbatim}
    $x           evaluates to    17
    foo$xbar     evaluates to    foo17bar
    foo$(x)bar   evaluates to    foo17bar
\end{verbatim}

The special sequence \verb+$$+ represents the character literal \verb+$+.  That is, the
two-character sequences \verb+\$+ and \verb+$$+ are normally equalivalent.

\subsection{String constants}
\label{section:quotes}

Literal strings are defined with matching string delimiters.  A left string delimiter begins with
the dollar-sign \verb+$+, and a non-zero number of single-quote or double-quote characters.  The
string is terminated with a matching sequence of quotation symbols.  The delimiter quotation may not
be mixed; it must contain only single-quote characters, or double-quote characters.  The following
are legal strings.

\begin{verbatim}
    $'Hello world'
    $"""printf("Hello world\n")"""
    $''''
Large "block" of
text # spanning ''multiple'' lines''''
\end{verbatim}

The string delimiters are \emph{not} included in the string constant.  In the single-quote form,
the contents of the string are interpreted verbatim--there are no special characters.

The double-quote form permits expression evaluation within the string, denoted with the \verb+$+ symbol.
The following are some examples.

\begin{verbatim}
    X = Hello
    Y = $""$X world""             # Hello world
    Z = $'''$X world'''           # $X world
    I = 3
    W = $"6 > $(add $I, 2)"       # 6 > 5
\end{verbatim}

Note that quotation symbols without a leading \verb+$+ are not treated specially by \OMake{}.  The
quotation symbols is included in the sequence.

\begin{verbatim}
    osh>println('Hello world')
    'Hello world'
    osh>println($'Hello world')
    Hello world
    osh>X = Hello
    - : "Hello" : Sequence
    osh>println('$X world')
    Hello world
\end{verbatim}

\section{The \OMake{} grammar}

\OMake{} programs are constructed from expressions and statements.  Generally, an input program
consists of a sequence of statements, each of which consists of one or more lines.  Indentation is
significant--if a statement consists of more than one line, the second and remaining lines (called
the \emph{body}) are usually indented relative to the first line.

\subsection{Expressions}

The following table lists the syntax for expressions.

\begin{tabular}{rcl}
\emph{expr} & ::= &\\
&   & \emph{(empty)}\\
&   & -- Text (see note)\\
& | & \emph{text}\\
& | & \emph{string-literal}\\
&   & -- Applications\\
& | & \emph{dollar} \verb+<char>+\\
& | & \emph{dollar} \verb+(+ \emph{pathid} \emph{args} \verb+)+\\
&   & -- Concatenation\\
& | & \emph{expr} \emph{expr}\\
\\
\emph{dollar} & ::= & \verb+$+ | \verb+$`+ | \verb+$,+
\\
\emph{pathid} & ::= &\\
&   & \emph{id}\\
& | & \emph{pathid} \verb+.+ \emph{id}\\
\\
\emph{arg} & ::= & \emph{expr}   -- excluding special characters \verb+)(,+)\\
\emph{args} & ::= & \emph{(empty)} | \emph{arg}, ..., \emph{arg}
\end{tabular}

An \emph{expression} is a sequence composed of text, string-literals, variables references and
function applications.  Text is any sequence of non-special characters.

\subsubsection{Inline applications}

An \emph{application} is the application of a function to zero-or-more arguments.  Inline
applications begin with one of the ``dollar'' sequences \verb+$+, \verb+$`+, or \verb+$,+.  The
application itself is specified as a single character (in which case it is a variable reference), or
it is a parenthesized list including a function identifier \emph{pathid}, and zero-or-more
comma-separated arguments \emph{args}.  The arguments are themselves a variant of the expressions
where the special character \verb+)(,+ are not allowed (though any of these may be made non-special
with the \verb+\+ escape character).  The following are some examples of valid expressions.

\begin{itemize}
\item \verb+xyz abc+

The text sequence ``\verb+xyz abc+''

\item \verb+xyz$wabc+

A text sequence containing a reference to the variable \verb+w+.

\item \verb+$(addsuffix .c, $(FILES))+

An application of the function \verb+addsuffix+, with first argument \verb+.c+, and second argument \verb+$(FILES)+.

\item \verb+$(a.b.c 12)+

This is a method call.  The variable \verb+a+ must evaluate to an object with a field \verb+b+,
which must be an object with a method \verb+c+.  This method is called with argument \verb+12+.
\end{itemize}

The additional dollar sequences specify evaluation order, \verb+$`+ (lazy) and \verb+$,+ (eager), as
discussed in the section on dollar modifiers (Section~\ref{section:dollar}).

\subsection{Statements and programs}

The following table lists the syntax of statements and programs.

\begin{tabular}{rcl}
\emph{params} & ::= & \emph{(empty)} | \emph{id}, ..., \emph{id}\\
\\
\emph{target} & ::= & \emph{expr} -- excluding special character \verb+:+\\
\\
\emph{program} & ::= & \emph{stmt} \verb+<eol>+ ... \verb+<eol>+ \emph{stmt}\\
\\
\emph{stmt} & ::= &\\
&   & -- Special forms\\
& | & \texttt{command} \emph{expr} \emph{optcolon-body}\\
& | & \texttt{command} ( \emph{args} ) \emph{optcolon-body}\\
& | & \texttt{catch} \emph{id} ( \emph{id} ) \emph{optcolon-body}\\
& | & \texttt{class} \emph{id} ... \emph{id}\\
\\
&   & -- Variable definitions\\
& | & \emph{pathid} \{+\}= \emph{expr}\\
& | & \emph{pathid} \{+\}= \verb+<eol>+ \emph{indented-body}\\
& | & \emph{pathid}\verb+[]+ \{+\}= \emph{expr}\\
& | & \emph{pathid}\verb+[]+ \{+\}= \verb+<eol>+ \emph{indented-exprs}\\
\\
&   & -- Functions\\
& | & \emph{pathid}(\emph{args}) \emph{optcolon-body}\\
& | & \emph{pathid}(\emph{params}) = \verb+<eol>+ \emph{indented-body}\\
\\
&   & -- Objects\\
& | & \emph{pathid} \verb+.+ \{+\}= \verb+<eol>+ \emph{indented-body}\\
\\
&   & -- Rules\\
& | & \emph{target} \texttt{:} \emph{target} \emph{rule-options} \verb+<eol>+ \emph{indented-body}\\
& | & \emph{target} \texttt{::} \emph{target} \emph{rule-options} \verb+<eol>+ \emph{indented-body}\\
& | & \emph{target} \texttt{:} \emph{target} \texttt{:} \emph{target} \emph{rule-options} \verb+<eol>+ \emph{indented-body}\\
& | & \emph{target} \texttt{::} \emph{target} \texttt{:} \emph{target} \emph{rule-options} \verb+<eol>+ \emph{indented-body}\\
\\
&   & -- Shell commands\\
& | & \emph{expr}\\
\\
\emph{indented-body} & ::= & \emph{(empty)}\\
& | & \emph{indented-stmt} \verb+<eol>+ ... \verb+<eol>+ \emph{indented-stmt}\\
\\
\emph{indented-exprs} & ::= & \emph{(empty)}\\
& | & \emph{indented-expr} \verb+<eol>+ ... \verb+<eol>+ \emph{indented-expr}\\
\\
\emph{optcolon-body} & ::= & \emph{(empty)}\\
& | & \verb+<eol>+ \emph{indented-body}\\
& | & \texttt{:} \verb+<eol>+ \emph{indented-body}\\
\\
\emph{rule-option} & ::= & \emph{:id:} \emph{target}\\
\emph{rule-options} & ::= & \emph{(empty)}\\
& | & \emph{rule-options} \emph{rule-option}
\end{tabular}

\subsubsection{Special forms}

The special forms include the following.

\textbf{Conditionals} (see the section on conditionals --- Section~\ref{section:conditionals}).  The \verb+if+ command
should be followed by an expression that represents the condition, and an indented body.  The
conditional may be followed by \verb+elseif+ and \verb+else+ blocks.

\begin{verbatim}
    if expr
        indented-body
    elseif expr
        indented-body
    ...
    else
        indented-body
\end{verbatim}

\textbf{matching} (see the section on matching --- Section~\ref{section:match}).  The \verb+switch+ and
\verb+match+ commands perform pattern-matching.  All cases are optional.  Each case may include
\verb+when+ clauses that specify additional matching conditions.

\begin{verbatim}
    match(expr)
    case expr
       indented-body
    when expr
       indented-body
    ...
    case expr
       indented-body
    default
       indented-body
\end{verbatim}

\textbf{Exceptions} (see also the \hyperfun{try} documentation).  The \verb+try+ command
introduces an exception handler.  Each \verb+name+ is the name of a class.  All cases, including
\verb+catch+, \verb+default+, and \verb+finally+ are optional.  The \verb+catch+ and \verb+default+
clauses contain optional \verb+when+ clauses.

\begin{verbatim}
    try
        indented-body
    catch name1(id1)
        indented-body
    when expr
        indented-body
    ...
    catch nameN(idN)
        indented-body
    default
        indented-body
    finally
        indented-body
\end{verbatim}

The \verb+raise+ command is used to raise an exception.

\begin{verbatim}
    raise expr
\end{verbatim}        

\textbf{section} (see the \verb+section+ description in Section~\ref{section:section}).  The \verb+section+ command
introduces a new scope.

\begin{verbatim}
    section
        indented-body
\end{verbatim}

\textbf{include, open} (see the \verb+include+ description in Section~\ref{section:include}).  The \verb+include+ command
performs file inclusion.  The expression should evaluate to a file name.

The \verb+open+ form is like include, but it performs the inclusion only if the inclusion has not
already been performed.  The \verb+open+ form is usually used to include library files.  [jyh-- this
behavior will change in subsequent revisions.]

\begin{verbatim}
    include expr
    open expr
\end{verbatim}

\textbf{return} (see the description of functions in Section~\ref{section:functions}).  The \verb+return+ command
terminates execution and returns a value from a function.

\begin{verbatim}
    return expr
\end{verbatim}

\textbf{value} (see the description of functions in Section~\ref{section:functions}).  The \verb+value+ command is an identity.
Syntactically, it is used to coerce a n expression to a statement.

\begin{verbatim}
    value expr
\end{verbatim}

\textbf{export} (see the section on scoping --- Section~\ref{section:export}).  The \verb+export+ command exports
a environment from a nested block.  If no arguments are given, the entire environment is exported.
Otherwise, the export is limited to the specified identifiers.

\begin{verbatim}
    export expr
\end{verbatim}

\textbf{while} (see also the \hyperfun{while} description).  The \verb+while+ command introduces a \verb+while+ loop.

\begin{verbatim}
    while expr
        indented-body
\end{verbatim}

\textbf{class, extends} (see the section on objects --- Section~\ref{section:objects}).  The \verb+class+ command
specifies an identifier for an object.  The \verb+extends+ command specifies a parent object.

\begin{verbatim}
    class id
    extends expr
\end{verbatim}

\subsubsection{Variable definitions}

See the section on variables (Section~\ref{section:variables}).  The simplest variable definition has the
following syntax.  The \verb+=+ form is a new definition.  The += form appends the value to
an existing definition.

\begin{verbatim}
    id = expr
    id += expr

    osh> X = 1
    - : "1" : Sequence
    osh> X += 7
    - : "1" " " "7" : Sequence
\end{verbatim}

A multi-line form is allowed, where the value is computed by an indented body.

\begin{verbatim}
    id {+}=
        indented-body

    osh> X =
             Y = HOME
             println(Y is $Y)
             getenv($Y)
    Y is HOME
    - : "/home/jyh" : Sequence
\end{verbatim}

The name may be qualified qith one of the \verb+public+, \verb+prtected+, or \verb+private+
modifiers.  Public variables are dynamically scoped.  Protected variables are fields in the current
object.  Private variables are statically scoped.

[jyh: revision 0.9.9 introduces modular namespaces; the meaning of these qualifiers is slightly changed.]

\begin{verbatim}
    public.X = $(addsuffix .c, 1 2 3)
    protected.Y = $(getenv HOME)
    private.Z = $"Hello world"
\end{verbatim}

\subsubsection{Applications and function definitions}

See the section on functions (Section~\ref{section:functions}).  A function-application statement is specified as a
function name, followed a parenthesized list of comma-separated arguments.

\begin{verbatim}
    osh> println($"Hello world")

    osh> FILES = 1 2 3
    - : 1 2 3
    osh> addsuffix(.c, $(FILES))
    - : 1.c 2.c 3.c

    # The following forms are equivalent
    osh> value $(println $"Hello world")
    osh> value $(addsuffix .c, $(FILES))
    - : 1.c 2.c 3.c
\end{verbatim}

If the function application has a body, the body is passed (lazily) to the function as its first
argument.  [jyh: in revision 0.9.8 support is incomplete.]  When using \verb+osh+, the application
must be followed by a colon \verb+:+ to indicate that the application has a body.

\begin{verbatim}
    # In its 3-argument form, the foreach function takes
    # a body, a variable, and an array.  The body is evaluated
    # for each element of the array, with the variable bound to
    # the element value.
    #
    # The colon is required only for interactive sessions.
    osh> foreach(x, 1 2 3):
            add($x, 1)
    - : 2 3 4
\end{verbatim}

Functions are defined in a similar form, where the parameter list is specified as a comma-separated
list of identifiers, and the body of the function is indented.

\begin{verbatim}
    osh> f(i, j) =
            add($i, $j)
    - : <fun 2>
    osh> f(3, 7)
    - : 10 : Int
\end{verbatim}

\subsubsection{Objects}

See the section on objects (Section~\ref{section:objects}).  Objects are defined as an identifier with a
terminal period.  The body of the object is indented.

\begin{verbatim}
    Obj. =
        class Obj

        X = 1
        Y = $(sub $X, 12)
        new(i, j) =
           X = $i
           Y = $j
           value $(this)
        F() =
           add($X, $Y)
        println($Y)
\end{verbatim}

The body of the object has the usual form of an indented body, but new variable definitions are
added to the object, not the global environment.  The object definition above defines an object with
(at least) the fields \verb+X+ and \verb+Y+, and methods \verb+new+ and \verb+F+.  The name of the
object is defined with the \verb+class+ command as \verb+Obj+.

The \verb+Obj+ itself has fields \verb+X = 1+ and \verb+Y = -11+.  The \verb+new+ method has the
typical form of a constructor-style method, where the fields of the object are initialized to new
values, and the new object returned (\verb+$(this)+ refers to the current object).

The \verb+F+ method returns the sum of the two fields \verb+X+ and \verb+Y+.

When used in an object definition, the += form adds the new definitions to an existing object.

\begin{verbatim}
   pair. =
      x = 1
      y = 2

   pair. +=
      y = $(add $y, 3)
   # pair now has fields (x = 1, and y = 5)
\end{verbatim}

The \verb+extends+ form specifies inheritance.  Multiple inheritance is allowed.  At evaluation
time, the \verb+extends+ directive performs inclusion of the entire parent object.

\begin{verbatim}
   pair. =
      x = 1
      y = 2

   depth. =
      z = 3
      zoom(dz) =
         z = $(add $z, $(dz))
         return $(this)

   triple. =
      extends $(pair)
      extends $(depth)

      crazy() =
         zoom($(mul $x, $y))
\end{verbatim}

In this example, the \verb+triple+ object has three fields x, y, and z; and two methods \verb+zoom+
and \verb+crazy+.

\subsubsection{Rules}

See the chapter on rules (Chapter~\ref{chapter:rules}).  A rule has the following parts.
\begin{enumerate}
\item A sequence of targets;
\item one or two colons;
\item a sequence of \emph{dependencies} and \emph{rule options};
\item and an indented body.
\end{enumerate}

The targets are the files to be built, and the dependencies are the files it depends on.  If two
colons are specified, it indicates that there may be multiple rules to build the given targets;
otherwise only one rule is allowed.

If the target contains a \verb+%+ character, the rule is called \emph{implicit}, and is considered
whenever a file matching that pattern is to be built.  For example, the following rule specifies a
default rule for compiling OCaml files.

\begin{verbatim}
    %.cmo: %.ml %.mli
       $(OCAMLC) -c $<
\end{verbatim}

This rule would be consulted as a default way of building any file with a \verb+.cmo+ suffix.  The
dependencies list is also constructed based on the pattern match.  For example, if this rule were
used to build a file \verb+foo.cmo+, then the dependency list would be \verb+foo.ml foo.mli+.

There is also a three-part version of a rule, where the rule specification has three parts.

\begin{verbatim}
    targets : pattern : dependencies rule-options
       indented-body
\end{verbatim}

In this case, the pattern \emph{must} contain a single \verb+%+ character.  However this is
considered to be a sequence of \emph{explicit} rules, where each target is matched against the
pattern, and a new rule is computed based on the pattern match.  For example, the following rule
specifies how to build the explicit targets \verb+a.cmo+ and \verb+b.cmo+.

\begin{verbatim}
    a.cmo b.cmo: %.cmo: %.ml %.mli
       $(OCAMLC) -c $<
\end{verbatim}

This example is equivalent to the following two-rule sequence.

\begin{verbatim}
    a.cmo: a.ml a.mli
       $(OCAMLC) -c $<
    b.cmo: b.ml b.mli
       $(OCAMLC) -c $<
\end{verbatim}

There are several special targets, including the following.

\begin{itemize}
\item \verb+.PHONY+ : declare a ``phony'' target.  That is, the target does not correspond to a file.
\item \verb+.ORDER+ : declare a rule for dependency ordering.
\item \verb+.INCLUDE+ : define a rule to generate a file for textual inclusion.
\item \verb+.SUBDIRS+ : specify subdirectories that are part of the project.
\item \verb+.SCANNER+ : define a rule for dependency scanning.
\end{itemize}

There are several rule options.

\begin{itemize}
\item \verb+:optional: dependencies+ the subsequent dependencies are optional, it is acceptable if they do not exist.
\item \verb+:exists: dependencies+ the subsequent dependencies must exist, but changes to not affect
whether this rule is considered out-of-date.
\item \verb+:effects: targets+ the subsequent files are side-effects of the rule.  That is, they may be
created and/or modified while the rule is executing.  Rules with overlapping side-effects are never
executed in parallel.
\item \verb+:scanner: name+ the subsequent name is the name of the \verb+.SCANNER+ rule for the target to be built.
\item \verb+:value: expr+ the \verb+expr+ is a ``value'' dependency.  The rule is considered
out-of-date whenever the value of the \verb+expr+ changes.
\end{itemize}

Several variables are defined during rule evaluation.

% We use \char* even for printable chars for consistency.
\begin{itemize}
\item \verb+$*+\index{\char36\char42}\index[var]{\char42} : the name of the target with the outermost suffix removed.
\item \verb+$>+\index{\char36\char62}\index[var]{\char62} : the name of the target with all suffixes removed.
\item \verb+$@+\index{\char36\char64}\index[var]{\char64} : the name of the target.
\item \verb+$^+\index{\char36\char94}\index[var]{\char94} : the explicit file dependencies, sorted alphabetically, with duplicates removed.
\item \verb.$+.\index{\char36\char43}\index[var]{\char43} : all explicit file dependencies, with order preserved.
\item \verb+$<+\index{\char36\char60}\index[var]{\char60} : the first explicit file dependency.
\item \verb+$&+\index{\char36\char38}\index[var]{\char38} : the free values of the rule (often used in \verb+:value:+ dependencies).
\end{itemize}

\subsubsection{Shell commands}

See the chapter on shell commands (Chapter~\ref{chapter:shell}).

While it is possible to give a precise specification of shell commands, the informal description is
simpler.  Any non-empty statement where each prefix is \emph{not} one of the other statements, is
considered to be a shell command.  Here are some examples.

\begin{verbatim}
    ls                                 -- shell command
    echo Hello world > /dev/null       -- shell command
    echo(Hello world)                  -- function application
    echo(Hello world) > /dev/null      -- syntax error
    echo Hello: world                  -- rule
    X=1 getenv X                       -- variable definition
    env X=1 getenv X                   -- shell command
    if true                            -- special form
    \if true                           -- shell command
    "if" true                          -- shell command
\end{verbatim}
    

%%%%%%%%%%%%%%%%%%%%%%%%%%%%%%%%%%%%%%%%%%%%%%%%%%%%%%%%%%%%%%%%%%%%%%%%
% Evaluation
%
\section{Dollar modifiers}
\label{section:dollar}

Inline applications have a function and zero-or-more arguments.  Evaluation is normally strict: when
an application is evaluated, the function identifier is evaluated to a function, the arguments are
then evaluated and the function is called with the evaluated arguments.

The additional ``dollar'' sequences specify additional control over evaluation.  The token \verb+$`+
defines a ``lazy'' application, where evaluation is delayed until a value is required.  The
\verb+$,+ sequence performs an ``eager'' application within a lazy context.

To illustrate, consider the expression \verb+$(addsuffix .c, $(FILES))+.  The \verb+addsuffix+
function appends its first argument to each value in its second argument.  The following \verb+osh+
interaction demonstrates the normal bahavior.

\begin{verbatim}
osh> FILES[] = a b c
- : <array a b c>
osh> X = $(addsuffix .c, $(FILES))
- : <array ...>
osh> FILES[] = 1 2 3 # redefine FILES
- : <array 1 2 3>
osh> println($"$X")  # force the evaluation and print
a.c b.c c.c
\end{verbatim}

When the lazy operator \verb+$`+ is used instead, evaluation is delayed until it is printed.  In the
following sample, the value for \verb+X+ has changed to the \verb+$(apply ..)+ form, but otherwise
the result is unchanged because it it printed immediately.

\begin{verbatim}
osh> FILES[] = a b c
- : <array a b c>
osh> SUF = .c
- : ".c"
osh> X = $`(addsuffix $(SUF), $(FILES))
- : $(apply global.addsuffix ...)
osh> println($"$X")  # force the evaluation and print
a.c b.c c.c
\end{verbatim}

However, consider what happens if we redefine the \verb+FILES+ variable after the definition for
\verb+X+.  In the following sample, the result changes because evaluation occurs \emph{after} the
values for \verb+FILES+ has been redefined.

\begin{verbatim}
osh> FILES[] = a b c
- : <array a b c>
osh> SUF = .c
- : ".c"
osh> X = $`(addsuffix $(SUF), $(FILES))
- : $(apply global.addsuffix ...)
osh> SUF = .x
osh> FILES[] = 1 2 3
osh> println($"$X")  # force the evaluation and print
1.x 2.x 3.x
\end{verbatim}

In some cases, more explicit control is desired over evaluation.  For example, we may wish to
evaluate \verb+SUF+ early, but allow for changes to the \verb+FILES+ variable.  The \verb+$,(SUF)+
expression forces early evaluation.

\begin{verbatim}
osh> FILES[] = a b c
- : <array a b c>
osh> SUF = .c
- : ".c"
osh> X = $`(addsuffix $,(SUF), $(FILES))
- : $(apply global.addsuffix ...)
osh> SUF = .x
osh> FILES[] = 1 2 3
osh> println($"$X")  # force the evaluation and print
1.c 2.c 3.c
\end{verbatim}

% -*-
% Local Variables:
% Mode: LaTeX
% fill-column: 100
% TeX-master: "paper"
% TeX-command-default: "LaTeX/dvips Interactive"
% End:
% -*-


\printindex[default]
\cutname{omake-all-index.html}
\printindex[var]
\cutname{omake-var-index.html}
\printindex[fun]
\cutname{omake-fun-index.html}
\printindex[obj]
\cutname{omake-obj-index.html}
\printindex[target]
\cutname{omake-target-index.html}
\printindex[opt]
\cutname{omake-option-index.html}

\chapter{References}
\label{chapter:references}
\cutname{omake-references.html}

\section{See Also}

omake(1) (Chapter~\ref{chapter:omake}),
osh(1) (Chapter~\ref{chapter:osh}),
make(1)

\section{Version}

Version: \Version\ of \Date.

\section{License and Copyright}

\copyright\ 2003-2006, Mojave Group, Caltech

This program is free software; you can redistribute it and/or
modify it under the terms of the GNU General Public License
as published by the Free Software Foundation; version 2
of the License.

This program is distributed in the hope that it will be useful,
but WITHOUT ANY WARRANTY; without even the implied warranty of
MERCHANTABILITY or FITNESS FOR A PARTICULAR PURPOSE.  See the
GNU General Public License for more details.

You should have received a copy of the GNU General Public License
along with this program; if not, write to the Free Software
Foundation, Inc., 675 Mass Ave, Cambridge, MA 02139, USA.

\section{Author}

\noindent
\authors					\\
Caltech 256-80					\\
Pasadena, CA 91125, USA				\\
Email: \texttt{omake-devel@metaprl.org}		\\
WWW: \url{http://www.cs.caltech.edu/~jyh} and \url{http://nogin.org/}

\end{document}

%% \IfFileExists{rcsinfo.sty}{
%% \setDate{\rcsInfoLongDate}
%% }{
%% \setDate{April 11, 2006}    %%%% must be manually set, if rcsinfo is not present
%% }
