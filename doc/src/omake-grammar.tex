%
% A more formal description of the grammar.
%
\chapter{\OMake{} grammar}
\label{chapter:grammar}
\cutname{omake-grammar.html}

\section{\OMake{} lexical conventions}

The \OMake{} language is based on the language for GNU/BSD make, where there are few lexical
conventions.  Strictly speaking, there are no keywords, and few special symbols.

\subsection{Comments}

Comments begin with the \verb+#+ character and continue to the end-of-line.
Text within a comment is unrestricted.

Examples.

\begin{verbatim}
   # This is a comment
   # This $comment contains a quote " character
\end{verbatim}

\subsection{Special characters}

The following characters are special in some contexts.

\begin{verbatim}
   $    (    )    ,    .   =    :    "    '    `    \    #
\end{verbatim}

\begin{itemize}
\item \verb+$+ is used to denote a variable reference, or function application.
\item Parentheses \verb+)+, \verb+(+ are argument deliminters.
\item The command \verb+,+ is an argument separator.
\item The period symbol \verb+.+ is a name separator.
\item The equality symbol \verb+=+ denotes a definition.
\item The colon symbol \verb+:+ is used to denote rules, and (optionally) to indicate
   that an expression is followed by an indented body.
\item The quotation symbols \verb+"+ and \verb+'+ delimit character strings.
\item The symbol \verb+#+ is the first character of a constant.
\item The escape symbol \verb+\+ is special \emph{only when} followed by another special
   character.  In this case, the special status of the second character is removed,
   and the sequence denotes the second character.  Otherwise, the \verb+\+ is not special.

   Examples:

   \begin{itemize}
   \item \verb+\$+: the \verb+$+ character (as a normal character).
   \item \verb+\#+: the \verb+#+ character (as a normal character).
   \item \verb+\\+: the \verb+\+ character (as a normal character).
   \item \verb+c\:\Windows\moo\#boo+: the string \verb+c:\Windows\moo#boo+.
   \end{itemize}
\end{itemize}

\subsection{Identifiers}

Identifiers (variable names) are drawn from the ASCII alphanumeric characters as well as \verb+_+,
\verb+-+, \verb+~+, \verb+@+.  Case is significant; the following identifiers are distinct:
\verb+FOO+, \verb+Foo+, \verb+foo+.  The identifier may begin with any of the valid characters,
including digits.

Using \verb+egrep+ notation, the regular expression for identifiers is defined as follows.

\begin{verbatim}
    identifier ::= [-@~_A-Za-z0-9]+
\end{verbatim}

The following are legal identifiers.

\begin{verbatim}
    Xyz    hello_world    seventy@nine
    79-32  Gnus~Gnats     CFLAGS
\end{verbatim}

The following are not legal identifiers.

\begin{verbatim}
    x+y    hello&world
\end{verbatim}

\subsection{Command identifiers}

The following words have special significance when they occur as the \emph{first} word
of a program line.  They are not otherwise special.

\begin{verbatim}
    case     catch  class    declare    default
    do       else   elseif   export     extends
    finally  if     import   include    match
    open     raise  return   section    switch
    try      value  when     while
\end{verbatim}

\subsection{Variable references}

A variable reference is denoted with the \verb+$+ special character followed by an identifier.  If
the identifier name has more than one character, it must be enclosed in parentheses.  The
parenthesized version is most common.  The following are legal variable references.

\begin{verbatim}
    $(Xyz)    $(hello_world)   $(seventy@nine)
    $(79-32)  $(Gnus~Gnats)    $(CFLAGS)
\end{verbatim}

Single-character references also include several additional identifiers, including \verb+&*<^?][+.
The following are legal single-character references.

\begin{verbatim}
   $@   $&   $*   $<   $^   $+   $?   $[   $]
   $A   $_   $a   $b   $x   $1   $2   $3
\end{verbatim}

Note that a non-parenthesized variable reference is limited to a single character, even if it is
followed by additional legal identifier charqcters.  Suppose the value of the \verb+$x+ variable is
17.  The following examples illustrate evaluation.

\begin{verbatim}
    $x           evaluates to    17
    foo$xbar     evaluates to    foo17bar
    foo$(x)bar   evaluates to    foo17bar
\end{verbatim}

The special sequence \verb+$$+ represents the character literal \verb+$+.  That is, the
two-character sequences \verb+\$+ and \verb+$$+ are normally equalivalent.

\subsection{String constants}
\label{section:quotes}

Literal strings are defined with matching string delimiters.  A left string delimiter begins with
the dollar-sign \verb+$+, and a non-zero number of single-quote or double-quote characters.  The
string is terminated with a matching sequence of quotation symbols.  The delimiter quotation may not
be mixed; it must contain only single-quote characters, or double-quote characters.  The following
are legal strings.

\begin{verbatim}
    $'Hello world'
    $"""printf("Hello world\n")"""
    $''''
Large "block" of
text # spanning ''multiple'' lines''''
\end{verbatim}

The string delimiters are \emph{not} included in the string constant.  In the single-quote form,
the contents of the string are interpreted verbatim--there are no special characters.

The double-quote form permits expression evaluation within the string, denoted with the \verb+$+ symbol.
The following are some examples.

\begin{verbatim}
    X = Hello
    Y = $""$X world""             # Hello world
    Z = $'''$X world'''           # $X world
    I = 3
    W = $"6 > $(add $I, 2)"       # 6 > 5
\end{verbatim}

Note that quotation symbols without a leading \verb+$+ are not treated specially by \OMake{}.  The
quotation symbols is included in the sequence.

\begin{verbatim}
    osh>println('Hello world')
    'Hello world'
    osh>println($'Hello world')
    Hello world
    osh>X = Hello
    - : "Hello" : Sequence
    osh>println('$X world')
    Hello world
\end{verbatim}

\section{The \OMake{} grammar}

\OMake{} programs are constructed from expressions and statements.  Generally, an input program
consists of a sequence of statements, each of which consists of one or more lines.  Indentation is
significant--if a statement consists of more than one line, the second and remaining lines (called
the \emph{body}) are usually indented relative to the first line.

\subsection{Expressions}

The following table lists the syntax for expressions.

\begin{tabular}{rcl}
\emph{expr} & ::= &\\
&   & \emph{(empty)}\\
&   & -- Text (see note)\\
& | & \emph{text}\\
& | & \emph{string-literal}\\
&   & -- Applications\\
& | & \emph{dollar} \verb+<char>+\\
& | & \emph{dollar} \verb+(+ \emph{pathid} \emph{args} \verb+)+\\
&   & -- Concatenation\\
& | & \emph{expr} \emph{expr}\\
\\
\emph{dollar} & ::= & \verb+$+ | \verb+$`+ | \verb+$,+
\\
\emph{pathid} & ::= &\\
&   & \emph{id}\\
& | & \emph{pathid} \verb+.+ \emph{id}\\
\\
\emph{arg} & ::= & \emph{expr}   -- excluding special characters \verb+)(,+)\\
\emph{args} & ::= & \emph{(empty)} | \emph{arg}, ..., \emph{arg}
\end{tabular}

An \emph{expression} is a sequence composed of text, string-literals, variables references and
function applications.  Text is any sequence of non-special characters.

\subsubsection{Inline applications}

An \emph{application} is the application of a function to zero-or-more arguments.  Inline
applications begin with one of the ``dollar'' sequences \verb+$+, \verb+$`+, or \verb+$,+.  The
application itself is specified as a single character (in which case it is a variable reference), or
it is a parenthesized list including a function identifier \emph{pathid}, and zero-or-more
comma-separated arguments \emph{args}.  The arguments are themselves a variant of the expressions
where the special character \verb+)(,+ are not allowed (though any of these may be made non-special
with the \verb+\+ escape character).  The following are some examples of valid expressions.

\begin{itemize}
\item \verb+xyz abc+

The text sequence ``\verb+xyz abc+''

\item \verb+xyz$wabc+

A text sequence containing a reference to the variable \verb+w+.

\item \verb+$(addsuffix .c, $(FILES))+

An application of the function \verb+addsuffix+, with first argument \verb+.c+, and second argument \verb+$(FILES)+.

\item \verb+$(a.b.c 12)+

This is a method call.  The variable \verb+a+ must evaluate to an object with a field \verb+b+,
which must be an object with a method \verb+c+.  This method is called with argument \verb+12+.
\end{itemize}

The additional dollar sequences specify evaluation order, \verb+$`+ (lazy) and \verb+$,+ (eager), as
discussed in the section on dollar modifiers (Section~\ref{section:dollar}).

\subsection{Statements and programs}

The following table lists the syntax of statements and programs.

\begin{tabular}{rcl}
\emph{params} & ::= & \emph{(empty)} | \emph{id}, ..., \emph{id}\\
\\
\emph{target} & ::= & \emph{expr} -- excluding special character \verb+:+\\
\\
\emph{program} & ::= & \emph{stmt} \verb+<eol>+ ... \verb+<eol>+ \emph{stmt}\\
\\
\emph{stmt} & ::= &\\
&   & -- Special forms\\
& | & \texttt{command} \emph{expr} \emph{optcolon-body}\\
& | & \texttt{command} ( \emph{args} ) \emph{optcolon-body}\\
& | & \texttt{catch} \emph{id} ( \emph{id} ) \emph{optcolon-body}\\
& | & \texttt{class} \emph{id} ... \emph{id}\\
\\
&   & -- Variable definitions\\
& | & \emph{pathid} \{+\}= \emph{expr}\\
& | & \emph{pathid} \{+\}= \verb+<eol>+ \emph{indented-body}\\
& | & \emph{pathid}\verb+[]+ \{+\}= \emph{expr}\\
& | & \emph{pathid}\verb+[]+ \{+\}= \verb+<eol>+ \emph{indented-exprs}\\
\\
&   & -- Functions\\
& | & \emph{pathid}(\emph{args}) \emph{optcolon-body}\\
& | & \emph{pathid}(\emph{params}) = \verb+<eol>+ \emph{indented-body}\\
\\
&   & -- Objects\\
& | & \emph{pathid} \verb+.+ \{+\}= \verb+<eol>+ \emph{indented-body}\\
\\
&   & -- Rules\\
& | & \emph{target} \texttt{:} \emph{target} \emph{rule-options} \verb+<eol>+ \emph{indented-body}\\
& | & \emph{target} \texttt{::} \emph{target} \emph{rule-options} \verb+<eol>+ \emph{indented-body}\\
& | & \emph{target} \texttt{:} \emph{target} \texttt{:} \emph{target} \emph{rule-options} \verb+<eol>+ \emph{indented-body}\\
& | & \emph{target} \texttt{::} \emph{target} \texttt{:} \emph{target} \emph{rule-options} \verb+<eol>+ \emph{indented-body}\\
\\
&   & -- Shell commands\\
& | & \emph{expr}\\
\\
\emph{indented-body} & ::= & \emph{(empty)}\\
& | & \emph{indented-stmt} \verb+<eol>+ ... \verb+<eol>+ \emph{indented-stmt}\\
\\
\emph{indented-exprs} & ::= & \emph{(empty)}\\
& | & \emph{indented-expr} \verb+<eol>+ ... \verb+<eol>+ \emph{indented-expr}\\
\\
\emph{optcolon-body} & ::= & \emph{(empty)}\\
& | & \verb+<eol>+ \emph{indented-body}\\
& | & \texttt{:} \verb+<eol>+ \emph{indented-body}\\
\\
\emph{rule-option} & ::= & \emph{:id:} \emph{target}\\
\emph{rule-options} & ::= & \emph{(empty)}\\
& | & \emph{rule-options} \emph{rule-option}
\end{tabular}

\subsubsection{Special forms}

The special forms include the following.

\textbf{Conditionals} (see the section on conditionals --- Section~\ref{section:conditionals}).  The \verb+if+ command
should be followed by an expression that represents the condition, and an indented body.  The
conditional may be followed by \verb+elseif+ and \verb+else+ blocks.

\begin{verbatim}
    if expr
        indented-body
    elseif expr
        indented-body
    ...
    else
        indented-body
\end{verbatim}

\textbf{matching} (see the section on matching --- Section~\ref{section:match}).  The \verb+switch+ and
\verb+match+ commands perform pattern-matching.  All cases are optional.  Each case may include
\verb+when+ clauses that specify additional matching conditions.

\begin{verbatim}
    match(expr)
    case expr
       indented-body
    when expr
       indented-body
    ...
    case expr
       indented-body
    default
       indented-body
\end{verbatim}

\textbf{Exceptions} (see also the \hyperfun{try} documentation).  The \verb+try+ command
introduces an exception handler.  Each \verb+name+ is the name of a class.  All cases, including
\verb+catch+, \verb+default+, and \verb+finally+ are optional.  The \verb+catch+ and \verb+default+
clauses contain optional \verb+when+ clauses.

\begin{verbatim}
    try
        indented-body
    catch name1(id1)
        indented-body
    when expr
        indented-body
    ...
    catch nameN(idN)
        indented-body
    default
        indented-body
    finally
        indented-body
\end{verbatim}

The \verb+raise+ command is used to raise an exception.

\begin{verbatim}
    raise expr
\end{verbatim}        

\textbf{section} (see the \verb+section+ description in Section~\ref{section:section}).  The \verb+section+ command
introduces a new scope.

\begin{verbatim}
    section
        indented-body
\end{verbatim}

\textbf{include, open} (see the \verb+include+ description in Section~\ref{section:include}).  The \verb+include+ command
performs file inclusion.  The expression should evaluate to a file name.

The \verb+open+ form is like include, but it performs the inclusion only if the inclusion has not
already been performed.  The \verb+open+ form is usually used to include library files.  [jyh-- this
behavior will change in subsequent revisions.]

\begin{verbatim}
    include expr
    open expr
\end{verbatim}

\textbf{return} (see the description of functions in Section~\ref{section:functions}).  The \verb+return+ command
terminates execution and returns a value from a function.

\begin{verbatim}
    return expr
\end{verbatim}

\textbf{value} (see the description of functions in Section~\ref{section:functions}).  The \verb+value+ command is an identity.
Syntactically, it is used to coerce a n expression to a statement.

\begin{verbatim}
    value expr
\end{verbatim}

\textbf{export} (see the section on scoping --- Section~\ref{section:export}).  The \verb+export+ command exports
a environment from a nested block.  If no arguments are given, the entire environment is exported.
Otherwise, the export is limited to the specified identifiers.

\begin{verbatim}
    export expr
\end{verbatim}

\textbf{while} (see also the \hyperfun{while} description).  The \verb+while+ command introduces a \verb+while+ loop.

\begin{verbatim}
    while expr
        indented-body
\end{verbatim}

\textbf{class, extends} (see the section on objects --- Section~\ref{section:objects}).  The \verb+class+ command
specifies an identifier for an object.  The \verb+extends+ command specifies a parent object.

\begin{verbatim}
    class id
    extends expr
\end{verbatim}

\subsubsection{Variable definitions}

See the section on variables (Section~\ref{section:variables}).  The simplest variable definition has the
following syntax.  The \verb+=+ form is a new definition.  The += form appends the value to
an existing definition.

\begin{verbatim}
    id = expr
    id += expr

    osh> X = 1
    - : "1" : Sequence
    osh> X += 7
    - : "1" " " "7" : Sequence
\end{verbatim}

A multi-line form is allowed, where the value is computed by an indented body.

\begin{verbatim}
    id {+}=
        indented-body

    osh> X =
             Y = HOME
             println(Y is $Y)
             getenv($Y)
    Y is HOME
    - : "/home/jyh" : Sequence
\end{verbatim}

The name may be qualified qith one of the \verb+public+, \verb+prtected+, or \verb+private+
modifiers.  Public variables are dynamically scoped.  Protected variables are fields in the current
object.  Private variables are statically scoped.

[jyh: revision 0.9.9 introduces modular namespaces; the meaning of these qualifiers is slightly changed.]

\begin{verbatim}
    public.X = $(addsuffix .c, 1 2 3)
    protected.Y = $(getenv HOME)
    private.Z = $"Hello world"
\end{verbatim}

\subsubsection{Applications and function definitions}

See the section on functions (Section~\ref{section:functions}).  A function-application statement is specified as a
function name, followed a parenthesized list of comma-separated arguments.

\begin{verbatim}
    osh> println($"Hello world")

    osh> FILES = 1 2 3
    - : 1 2 3
    osh> addsuffix(.c, $(FILES))
    - : 1.c 2.c 3.c

    # The following forms are equivalent
    osh> value $(println $"Hello world")
    osh> value $(addsuffix .c, $(FILES))
    - : 1.c 2.c 3.c
\end{verbatim}

If the function application has a body, the body is passed (lazily) to the function as its first
argument.  [jyh: in revision 0.9.8 support is incomplete.]  When using \verb+osh+, the application
must be followed by a colon \verb+:+ to indicate that the application has a body.

\begin{verbatim}
    # In its 3-argument form, the foreach function takes
    # a body, a variable, and an array.  The body is evaluated
    # for each element of the array, with the variable bound to
    # the element value.
    #
    # The colon is required only for interactive sessions.
    osh> foreach(x, 1 2 3):
            add($x, 1)
    - : 2 3 4
\end{verbatim}

Functions are defined in a similar form, where the parameter list is specified as a comma-separated
list of identifiers, and the body of the function is indented.

\begin{verbatim}
    osh> f(i, j) =
            add($i, $j)
    - : <fun 2>
    osh> f(3, 7)
    - : 10 : Int
\end{verbatim}

\subsubsection{Objects}

See the section on objects (Section~\ref{section:objects}).  Objects are defined as an identifier with a
terminal period.  The body of the object is indented.

\begin{verbatim}
    Obj. =
        class Obj

        X = 1
        Y = $(sub $X, 12)
        new(i, j) =
           X = $i
           Y = $j
           value $(this)
        F() =
           add($X, $Y)
        println($Y)
\end{verbatim}

The body of the object has the usual form of an indented body, but new variable definitions are
added to the object, not the global environment.  The object definition above defines an object with
(at least) the fields \verb+X+ and \verb+Y+, and methods \verb+new+ and \verb+F+.  The name of the
object is defined with the \verb+class+ command as \verb+Obj+.

The \verb+Obj+ itself has fields \verb+X = 1+ and \verb+Y = -11+.  The \verb+new+ method has the
typical form of a constructor-style method, where the fields of the object are initialized to new
values, and the new object returned (\verb+$(this)+ refers to the current object).

The \verb+F+ method returns the sum of the two fields \verb+X+ and \verb+Y+.

When used in an object definition, the += form adds the new definitions to an existing object.

\begin{verbatim}
   pair. =
      x = 1
      y = 2

   pair. +=
      y = $(add $y, 3)
   # pair now has fields (x = 1, and y = 5)
\end{verbatim}

The \verb+extends+ form specifies inheritance.  Multiple inheritance is allowed.  At evaluation
time, the \verb+extends+ directive performs inclusion of the entire parent object.

\begin{verbatim}
   pair. =
      x = 1
      y = 2

   depth. =
      z = 3
      zoom(dz) =
         z = $(add $z, $(dz))
         return $(this)

   triple. =
      extends $(pair)
      extends $(depth)

      crazy() =
         zoom($(mul $x, $y))
\end{verbatim}

In this example, the \verb+triple+ object has three fields x, y, and z; and two methods \verb+zoom+
and \verb+crazy+.

\subsubsection{Rules}

See the chapter on rules (Chapter~\ref{chapter:rules}).  A rule has the following parts.
\begin{enumerate}
\item A sequence of targets;
\item one or two colons;
\item a sequence of \emph{dependencies} and \emph{rule options};
\item and an indented body.
\end{enumerate}

The targets are the files to be built, and the dependencies are the files it depends on.  If two
colons are specified, it indicates that there may be multiple rules to build the given targets;
otherwise only one rule is allowed.

If the target contains a \verb+%+ character, the rule is called \emph{implicit}, and is considered
whenever a file matching that pattern is to be built.  For example, the following rule specifies a
default rule for compiling OCaml files.

\begin{verbatim}
    %.cmo: %.ml %.mli
       $(OCAMLC) -c $<
\end{verbatim}

This rule would be consulted as a default way of building any file with a \verb+.cmo+ suffix.  The
dependencies list is also constructed based on the pattern match.  For example, if this rule were
used to build a file \verb+foo.cmo+, then the dependency list would be \verb+foo.ml foo.mli+.

There is also a three-part version of a rule, where the rule specification has three parts.

\begin{verbatim}
    targets : pattern : dependencies rule-options
       indented-body
\end{verbatim}

In this case, the pattern \emph{must} contain a single \verb+%+ character.  However this is
considered to be a sequence of \emph{explicit} rules, where each target is matched against the
pattern, and a new rule is computed based on the pattern match.  For example, the following rule
specifies how to build the explicit targets \verb+a.cmo+ and \verb+b.cmo+.

\begin{verbatim}
    a.cmo b.cmo: %.cmo: %.ml %.mli
       $(OCAMLC) -c $<
\end{verbatim}

This example is equivalent to the following two-rule sequence.

\begin{verbatim}
    a.cmo: a.ml a.mli
       $(OCAMLC) -c $<
    b.cmo: b.ml b.mli
       $(OCAMLC) -c $<
\end{verbatim}

There are several special targets, including the following.

\begin{itemize}
\item \verb+.PHONY+ : declare a ``phony'' target.  That is, the target does not correspond to a file.
\item \verb+.ORDER+ : declare a rule for dependency ordering.
\item \verb+.INCLUDE+ : define a rule to generate a file for textual inclusion.
\item \verb+.SUBDIRS+ : specify subdirectories that are part of the project.
\item \verb+.SCANNER+ : define a rule for dependency scanning.
\end{itemize}

There are several rule options.

\begin{itemize}
\item \verb+:optional: dependencies+ the subsequent dependencies are optional, it is acceptable if they do not exist.
\item \verb+:exists: dependencies+ the subsequent dependencies must exist, but changes to not affect
whether this rule is considered out-of-date.
\item \verb+:effects: targets+ the subsequent files are side-effects of the rule.  That is, they may be
created and/or modified while the rule is executing.  Rules with overlapping side-effects are never
executed in parallel.
\item \verb+:scanner: name+ the subsequent name is the name of the \verb+.SCANNER+ rule for the target to be built.
\item \verb+:value: expr+ the \verb+expr+ is a ``value'' dependency.  The rule is considered
out-of-date whenever the value of the \verb+expr+ changes.
\end{itemize}

Several variables are defined during rule evaluation.

% We use \char* even for printable chars for consistency.
\begin{itemize}
\item \verb+$*+\index{\char36\char42}\index[var]{\char42} : the name of the target with the outermost suffix removed.
\item \verb+$>+\index{\char36\char62}\index[var]{\char62} : the name of the target with all suffixes removed.
\item \verb+$@+\index{\char36\char64}\index[var]{\char64} : the name of the target.
\item \verb+$^+\index{\char36\char94}\index[var]{\char94} : the explicit file dependencies, sorted alphabetically, with duplicates removed.
\item \verb.$+.\index{\char36\char43}\index[var]{\char43} : all explicit file dependencies, with order preserved.
\item \verb+$<+\index{\char36\char60}\index[var]{\char60} : the first explicit file dependency.
\item \verb+$&+\index{\char36\char38}\index[var]{\char38} : the free values of the rule (often used in \verb+:value:+ dependencies).
\end{itemize}

\subsubsection{Shell commands}

See the chapter on shell commands (Chapter~\ref{chapter:shell}).

While it is possible to give a precise specification of shell commands, the informal description is
simpler.  Any non-empty statement where each prefix is \emph{not} one of the other statements, is
considered to be a shell command.  Here are some examples.

\begin{verbatim}
    ls                                 -- shell command
    echo Hello world > /dev/null       -- shell command
    echo(Hello world)                  -- function application
    echo(Hello world) > /dev/null      -- syntax error
    echo Hello: world                  -- rule
    X=1 getenv X                       -- variable definition
    env X=1 getenv X                   -- shell command
    if true                            -- special form
    \if true                           -- shell command
    "if" true                          -- shell command
\end{verbatim}
    

%%%%%%%%%%%%%%%%%%%%%%%%%%%%%%%%%%%%%%%%%%%%%%%%%%%%%%%%%%%%%%%%%%%%%%%%
% Evaluation
%
\section{Dollar modifiers}
\label{section:dollar}

Inline applications have a function and zero-or-more arguments.  Evaluation is normally strict: when
an application is evaluated, the function identifier is evaluated to a function, the arguments are
then evaluated and the function is called with the evaluated arguments.

The additional ``dollar'' sequences specify additional control over evaluation.  The token \verb+$`+
defines a ``lazy'' application, where evaluation is delayed until a value is required.  The
\verb+$,+ sequence performs an ``eager'' application within a lazy context.

To illustrate, consider the expression \verb+$(addsuffix .c, $(FILES))+.  The \verb+addsuffix+
function appends its first argument to each value in its second argument.  The following \verb+osh+
interaction demonstrates the normal bahavior.

\begin{verbatim}
osh> FILES[] = a b c
- : <array a b c>
osh> X = $(addsuffix .c, $(FILES))
- : <array ...>
osh> FILES[] = 1 2 3 # redefine FILES
- : <array 1 2 3>
osh> println($"$X")  # force the evaluation and print
a.c b.c c.c
\end{verbatim}

When the lazy operator \verb+$`+ is used instead, evaluation is delayed until it is printed.  In the
following sample, the value for \verb+X+ has changed to the \verb+$(apply ..)+ form, but otherwise
the result is unchanged because it it printed immediately.

\begin{verbatim}
osh> FILES[] = a b c
- : <array a b c>
osh> SUF = .c
- : ".c"
osh> X = $`(addsuffix $(SUF), $(FILES))
- : $(apply global.addsuffix ...)
osh> println($"$X")  # force the evaluation and print
a.c b.c c.c
\end{verbatim}

However, consider what happens if we redefine the \verb+FILES+ variable after the definition for
\verb+X+.  In the following sample, the result changes because evaluation occurs \emph{after} the
values for \verb+FILES+ has been redefined.

\begin{verbatim}
osh> FILES[] = a b c
- : <array a b c>
osh> SUF = .c
- : ".c"
osh> X = $`(addsuffix $(SUF), $(FILES))
- : $(apply global.addsuffix ...)
osh> SUF = .x
osh> FILES[] = 1 2 3
osh> println($"$X")  # force the evaluation and print
1.x 2.x 3.x
\end{verbatim}

In some cases, more explicit control is desired over evaluation.  For example, we may wish to
evaluate \verb+SUF+ early, but allow for changes to the \verb+FILES+ variable.  The \verb+$,(SUF)+
expression forces early evaluation.

\begin{verbatim}
osh> FILES[] = a b c
- : <array a b c>
osh> SUF = .c
- : ".c"
osh> X = $`(addsuffix $,(SUF), $(FILES))
- : $(apply global.addsuffix ...)
osh> SUF = .x
osh> FILES[] = 1 2 3
osh> println($"$X")  # force the evaluation and print
1.c 2.c 3.c
\end{verbatim}

% -*-
% Local Variables:
% Mode: LaTeX
% fill-column: 100
% TeX-master: "paper"
% TeX-command-default: "LaTeX/dvips Interactive"
% End:
% -*-
