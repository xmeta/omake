%%%%%%%%%%%%%%%%%%%%%%%%%%%%%%%%%%%%%%%%%%%%%%%%%%%%%%%%%%%%%%%%%%%%%%%%
% Short description
%
\chapter{Guide}
\label{chapter:omake}
\label{section:guide}
\cutname{omake.html}

If you are new to OMake, you the \href{omake-quickstart.html}{omake-quickstart} presents a short
introduction that describes how to set up a project.  The
\href{omake-build-examples.html}{omake-build-examples} gives larger examples of build projects, and
\href{omake-language-examples.html}{omake-language-examples} presents programming examples.

\begin{description}
\item[Quickstart~\ref{chapter:quickstart}]
%
   A quickstart guide to using \Prog{omake}.
\item[Build examples~\ref{chapter:build-examples}]
%
   Advanced build examples.
\item[The OMake language~\ref{chapter:language}]
%
   The \Prog{omake} language, including a description of objects, expressions, and values.
\item[Language discussion~\ref{chapter:extra}]
%
   Further discussion on the language, including scoping, evaluation, and objects.
\item[Language examples~\ref{chapter:language-examples}]
%
   Additional language examples.
\item[Build rules~\ref{chapter:rules}]
%
   Defining and using rules to build programs.
\item[Base builtin functions~\ref{chapter:base}]
%
   Functions and variables in the core standard library.
\item[System functions~\ref{chapter:system}]
%
   Functions on files, input/output, and system commands.
\item[Shell commands~\ref{chapter:shell}]
%
   Using the \Prog{omake} shell for command-line interpretation.
\item[The standard objects Pervasives.om~\ref{chapter:pervasives}]
%
   Pervasives defines the built-in objects.
\item[Standard build definitions~\ref{chapter:rule}]
%
   The build specifications for programming languages in the OMake standard library.
\item[The interactive command interpreter~\ref{chapter:osh}]
%
   The \Prog{osh} command-line interpreter.
\item[\textbf{Appendices}]
\begin{description}
\item[OMake command-line options~\ref{chapter:options}]
%
   Command-line options for \Prog{omake}.
%
\item[The OMake language grammar~\ref{chapter:grammar}]
%
   A more precise specification of the OMake language.
\end{description}
\item[\href{omake-doc.html}{All the documentation on a single page}]
%
   All the OMake documentation in a single page.
\end{description}

% -*-
% Local Variables:
% Mode: LaTeX
% fill-column: 100
% TeX-master: "paper"
% TeX-command-default: "LaTeX/dvips Interactive"
% End:
% -*-
