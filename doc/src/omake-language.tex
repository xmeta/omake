%%%%%%%%%%%%%%%%%%%%%%%%%%%%%%%%%%%%%%%%%%%%%%%%%%%%%%%%%%%%%%%%%%%%%%%%
% Description
%
\chapter{\OMake{} concepts and syntax}
\label{chapter:language}
\cutname{omake-language.html}

Projects are specified to \Prog{omake} with \File{OMakefile}s.  The \File{OMakefile} has a format
similar to a \File{Makefile}.  An \File{OMakefile} has three main kinds of syntactic objects:
variable definitions, function definitions, and rule definitions.

\section{Variables}
\label{section:variables}

Variables are defined with the following syntax.  The name is any sequence of alphanumeric
characters, underscore \verb+_+, and hyphen \verb+-+.

\begin{verbatim}
   <name> = <value>
\end{verbatim}

Values are defined as a sequence of literal characters and variable expansions.  A variable
expansion has the form \verb+$(<name>)+, which represents the value of the \verb+<name>+
variable in the current environment.  Some examples are shown below.

\begin{verbatim}
   CC = gcc
   CFLAGS = -Wall -g
   COMMAND = $(CC) $(CFLAGS) -O2
\end{verbatim}

In this example, the value of the \verb+COMMAND+ variable is the string \verb+gcc -Wall -g -O2+.

Unlike \Cmd{make}{1}, variable expansion is \emph{eager} and \emph{pure} (see also the section
on Scoping).  That is, variable values are expanded immediately and new variable definitions do not
affect old ones.  For example, suppose we extend the previous example with following variable
definitions.

\begin{verbatim}
   X = $(COMMAND)
   COMMAND = $(COMMAND) -O3
   Y = $(COMMAND)
\end{verbatim}

In this example, the value of the \verb+X+ variable is the string \verb+gcc -Wall -g -O2+ as
before, and the value of the \verb+Y+ variable is \verb+gcc -Wall -g -O2 -O3+.

\section{Adding to a variable definition}

Variables definitions may also use the += operator, which adds the new text to an existing
definition.  The following two definitions are equivalent.

\begin{verbatim}
   # Add options to the CFLAGS variable
   CFLAGS = $(CFLAGS) -Wall -g

   # The following definition is equivalent
   CFLAGS += -Wall -g
\end{verbatim}

\section{Arrays}
\index{arrays}

Arrays can be defined by appending the \verb+[]+ sequence to the variable name and defining initial
values for the elements as separate lines.  Whitespace on each line is
taken literally.  The following code sequence prints \verb+c d e+.

\begin{verbatim}
    X[] =
        a b
        c d e
        f

    println($(nth 2, $(X)))
\end{verbatim}

\section{Special characters and quoting}
\index{quotations}

The following characters are special to \Prog{omake}: \verb+$():,=#\+.  To treat
any of these characters as normal text, they should be escaped with the backslash
character \verb+\+.

\begin{verbatim}
    DOLLAR = \$
\end{verbatim}

Newlines may also be escaped with a backslash to concatenate several lines.

\begin{verbatim}
    FILES = a.c\
            b.c\
            c.c
\end{verbatim}

Note that the backslash is \emph{not} an escape for any other character, so the following
works as expected (that is, it preserves the backslashes in the string).

\begin{verbatim}
    DOSTARGET = C:\WINDOWS\control.ini
\end{verbatim}

An alternative mechanism for quoting special text is the use \verb+$"..."+ escapes.  The number of
double-quotations is arbitrary.  The outermost quotations are not included in the text.

\begin{verbatim}
    A = $""String containing "quoted text" ""
    B = $"""Multi-line
        text.
        The # character is not special"""
\end{verbatim}

\section{Function definitions}
\label{section:functions}
\index{functions}

Functions are defined using the following syntax.

\begin{verbatim}
   <name>(<params>) =
      <indented-body>
\end{verbatim}

The parameters are a comma-separated list of identifiers, and the body must be placed on a separate
set of lines that are indented from the function definition itself.  For example, the following text
defines a function that concatenates its arguments, separating them with a colon.

\begin{verbatim}
    ColonFun(a, b) =
        return($(a):$(b))
\end{verbatim}

\index{return}%
The \verb+return+ expression can be used to return a value from the function.  A \verb+return+
statement is not required; if it is omitted, the returned value is the value of the last expression
in the body to be evaluated.  NOTE: as of version \verb+0.9.6+, \verb+return+ is a control
operation, causing the function to immediately return.  In the following example, when the argument
\verb+a+ is true, the function \verb+f+ immediately returns the value 1 without evaluating the print
statement.

\begin{verbatim}
    f(a) =
       if $(a)
          return 1
       println(The argument is false)
       return 0
\end{verbatim}

\index{value}%
In many cases, you may wish to return a value from a section or code block without returning from
the function.  In this case, you would use the \verb+value+ operator.  In fact, the \verb+value+
operator is not limited to functions, it can be used any place where a value is required.  In the
following definition, the variable \verb+X+ is defined as $1$ or $2$, depending on the value of $a$,
then result is printed, and returned from the function.

\begin{verbatim}
    f_value(a) =
       X =
          if $(a)
             value 1
          else
             value 2
       println(The value of X is $(X))
       value $(X)
\end{verbatim}

Functions are called using the GNU-make syntax, \verb+$(<name> <args))+,
where \verb+<args>+ is a comma-separated list of values.  For example,
in the following program, the variable \verb+X+ contains the
value \verb+foo:bar+.

\begin{verbatim}
   X = $(ColonFun foo, bar)
\end{verbatim}

If the value of a function is not needed, the function may also be called
using standard function call notation.  For example, the following program
prints the string ``She says: Hello world''.

\begin{verbatim}
    Printer(name) =
        println($(name) says: Hello world)

    Printer(She)
\end{verbatim}

\subsection{Keyword arguments}
\label{section:keyword-arguments}
\index{keyword arguments}

\newinkeyword

Functions can also have keyword parameters and arguments.  The syntax of a keyword
parameter/argument is \verb+[~|?]<id> [= <expression>]+, where the keyword name \verb+<id>+
is preceeded by the character \verb+~+ (for required arguments), or \verb+?+ (for optional
arguments).  If a default value \verb+= <expression>+ is provided, the argument is
always optional.

Keyword arguments and normal anonymous arguments are completely separate.  Also, it is an error to
pass a keyword argument to a function that does not define it as a keyword parameter.

\begin{verbatim}
    osh>f(x, ?y = 1, z) =
           add($(mul $x, 100), $(mul $y, 10), $z)
    - : <fun 0>
    osh>f(1, ~y = 2, 3)
    - : 123 : Int
    osh>f(1, 3, ~y = 2)
    - : 123 : Int
    osh>f(1, 3)
    - : 113 : Int
    osh>f(1, 2, 3)
    *** omake error:
       File -: line 11, characters 0-10
       arity mismatch: expected 2 args, got 3
    osh>f(~z = 7)
    *** omake error:
       File -: line 12, characters 0-8
       no such keyword: z
\end{verbatim}

An optional keyword argument defaults to the empty value.

\begin{verbatim}
    osh> g(?x) =
             println($">>>$x<<<")
    - : <fun 0>
    osh> g()
    >>><<<
    osh> g(~x = xxx)
    >>>xxx<<<
\end{verbatim}

It is an error to omit a required keyword argument.

\begin{verbatim}
    osh> h(~x, ~y) =
             println(x = $x; y = $y)
    - : <fun 0>
    osh> h(~y = 2, ~x = 1)
    x = 1; y = 2
    osh> h(~y = 2)
    *** omake error:
       File -: line 11, characters 0-9
       keyword argument is required: x
\end{verbatim}

\section{Curried functions}

Functions that are marked with the classifier \verb+curry+ can be called with ``too many'' arguments.
It is expected that a curried function returns a function that consumes the remaining arguments.
All arguments must be specified.

\begin{verbatim}
    osh>curry.f(x, y) =
            println($"Got two arguments: x = $x, y = $y")
            g(z) =
               add($x, $y, $z)
    osh> f(1, 2, 3)
    Got two arguments: x = 1, y = 2
    - : 6 : Int
    osh> f(1, 2)
    Got two arguments: x = 1, y = 2
    *** omake error:
       File -: line 62, characters 0-7
       arity mismatch: expected 1 args, got 0
\end{verbatim}

The function \verb+apply+ can be used to compute partial applications, whether or not the function
is labeled as a curried function.

\begin{verbatim}
    osh> f1(a, ~b = 2, ~c = 3, d) =
            println($"a = $a, b = $b, c = $c, d = $d")
    - : <fun 0>
    osh> f2 = $(apply $(f1), ~c = 13, 11)
    - : <curry 0>
    osh> f2(14, ~b = 12)
    a = 11, b = 12, c = 13, d = 14
    osh> f2(24)
    a = 11, b = 2, c = 13, d = 24
\end{verbatim}

\section{Comments}

Comments begin with the \verb+#+ character and continue to the end of the line.

\section{File inclusion}
\label{section:include}
\index{include}\index[fun]{include}\index{open}

Files may be included with the \verb+include+ or \verb+open+ form.  The included file must use
the same syntax as an \File{OMakefile}.

\begin{verbatim}
    include $(Config_file)
\end{verbatim}

The \verb+open+ operation is similar to an \verb+include+, but the file is included at most once.
\begin{verbatim}
    open Config

    # Repeated opens are ignored, so this
    # line has no effect.
    open Config
\end{verbatim}

If the file specified is not an absolute filenmame, both \verb+include+ and
\verb+open+ operations search for the file based on the
\hypervar{OMAKEPATH}. In case of the \verb+open+ directive, the search is
performed at \emph{parse} time, and the argument to \verb+open+ may not
contain any expressions.

\section{Scoping, sections}
\label{section:section}
\index{section}

Scopes in \Prog{omake} are defined by indentation level.  When indentation is
increased, such as in the body of a function, a new scope is introduced.

The \verb+section+ form can also be used to define a new scope.  For example, the following code
prints the line \verb+X = 2+, followed by the line \verb+X = 1+.

\begin{verbatim}
    X = 1
    section
        X = 2
        println(X = $(X))

    println(X = $(X))
\end{verbatim}

This result may seem surprising--the variable definition within the
\verb+section+ is not visible outside the scope of the \verb+section+.

The \verb+export+ form, which will be described in detail in
Section~\ref{section:export}, can be used to circumvent this restriction by
exporting variable values from an inner scope.
For example, if we modify the previous example
by adding an \verb+export+ expression, the new value for the \verb+X+
variable is retained, and the code prints the line \verb+X = 2+ twice.

\begin{verbatim}
    X = 1
    section
        X = 2
        println(X = $(X))
        export

    println(X = $(X))
\end{verbatim}

There are also cases where separate scoping is quite important.  For example,
each \File{OMakefile} is evaluated in its own scope.  Since each part of a project
may have its own configuration, it is important that variable definitions in one
\File{OMakefile} do not affect the definitions in another.

To give another example, in some cases it is convenient to specify a
separate set of variables for different build targets.  A frequent
idiom in this case is to use the \verb+section+ command to define a
separate scope.

\begin{verbatim}
   section
      CFLAGS += -g
      %.c: %.y
          $(YACC) $<
      .SUBDIRS: foo

   .SUBDIRS: bar baz
\end{verbatim}

In this example, the \verb+-g+ option is added to the \verb+CFLAGS+
variable by the \verb+foo+ subdirectory, but not by the \verb+bar+ and
\verb+baz+ directories. The implicit rules are scoped as well and in this
example, the newly added yacc rule will be inherited by the \verb+foo+
subdirectory, but not by the \verb+bar+ and \verb+baz+ ones; furthermore
this implicit rule will not be in scope in the current directory.

\section{Conditionals}
\label{section:conditionals}
\index{conditionals}
\index{if}

Top level conditionals have the following form.

\begin{verbatim}
    if <test>
       <true-clause>
    elseif <test2>
       <elseif-clause>
    else
       <else-clause>
\end{verbatim}

The \verb+<test>+ expression is evaluated, and if it evaluates to a \emph{true} value (see
Section~\ref{section:logic} for more information on logical values, and Boolean functions), the code
for the \verb+<true-clause>+ is evaluated; otherwise the remaining clauses are evaluated.  There may
be multiple \verb+elseif+ clauses; both the \verb+elseif+ and \verb+else+ clauses are optional.
Note that the clauses are indented, so they introduce new scopes.

When viewed as a predicate, a value corresponds to the Boolean \emph{false}, if its string
representation is the empty string, or one of the strings \verb+false+, \verb+no+, \verb+nil+,
\verb+undefined+, or \verb+0+.  All other values are \emph{true}.

The following example illustrates a typical use of a conditional.  The
\verb+OSTYPE+ variable is the current machine architecture.

\begin{verbatim}
    # Common suffixes for files
    if $(equal $(OSTYPE), Win32)
       EXT_LIB = .lib
       EXT_OBJ = .obj
       EXT_ASM = .asm
       EXE = .exe
       export
    elseif $(mem $(OSTYPE), Unix Cygwin)
       EXT_LIB = .a
       EXT_OBJ = .o
       EXT_ASM = .s
       EXE =
       export
    else
       # Abort on other architectures
       eprintln(OS type $(OSTYPE) is not recognized)
       exit(1)
\end{verbatim}

\section{Matching}
\label{section:match}
\index{match}\index[fun]{match}
\index{switch}\index[fun]{switch}

Pattern matching is performed with the \verb+switch+ and \verb+match+ forms.

\begin{verbatim}
    switch <string>
    case <pattern1>
        <clause1>
    case <pattern2>
        <clause2>
    ...
    default
       <default-clause>
\end{verbatim}

The number of cases is arbitrary.
The \verb+default+ clause is optional; however, if it is used it should
be the last clause in the pattern match.

For \verb+switch+, the string is compared with the patterns literally.

\begin{verbatim}
    switch $(HOST)
    case mymachine
        println(Building on mymachine)
    default
        println(Building on some other machine)
\end{verbatim}

Patterns need not be constant strings.  The following function tests
for a literal match against \verb+pattern1+, and a match against
\verb+pattern2+ with \verb+##+ delimiters.

\begin{verbatim}
   Switch2(s, pattern1, pattern2) =
      switch $(s)
      case $(pattern1)
          println(Pattern1)
      case $"##$(pattern2)##"
          println(Pattern2)
      default
          println(Neither pattern matched)
\end{verbatim}

For \verb+match+ the patterns are \Cmd{egrep}{1}-style regular expressions.
The numeric variables \verb+$1, $2, ...+ can be used to retrieve values
that are matched by \verb+\(...\)+ expressions.

\begin{verbatim}
    match $(NODENAME)@$(SYSNAME)@$(RELEASE)
    case $"mymachine.*@\(.*\)@\(.*\)"
        println(Compiling on mymachine; sysname $1 and release $2 are ignored)

    case $".*@Linux@.*2\.4\.\(.*\)"
        println(Compiling on a Linux 2.4 system; subrelease is $1)

    default
        eprintln(Machine configuration not implemented)
        exit(1)
\end{verbatim}

%%%%%%%%%%%%%%%%%%%%%%%%%%%%%%%%%%%%%%%%%%%%%%%%%%%%%%%%%%%%%%%%%%%%%%%%
% Objects
%
\section{Objects}
\label{section:objects}
\index{objects}

\OMake{} is an object-oriented language.  Generally speaking, an object is a value that contains fields
and methods.  An object is defined with a \verb+.+ suffix for a variable.  For example, the
following object might be used to specify a point $(1, 5)$ on the two-dimensional plane.

\begin{verbatim}
    Coord. =
        x = 1
        y = 5
        print(message) =
           println($"$(message): the point is ($(x), $(y)")

    # Define X to be 5
    X = $(Coord.x)

    # This prints the string, "Hi: the point is (1, 5)"
    Coord.print(Hi)
\end{verbatim}

The fields \verb+x+ and \verb+y+ represent the coordinates of the point.  The method \verb+print+
prints out the position of the point.

\section{Classes}
\index{classes}

We can also define \emph{classes}.  For example, suppose we wish to define a generic \verb+Point+
class with some methods to create, move, and print a point.  A class is really just an object with
a name, defined with the \verb+class+ directive.

\begin{verbatim}
    Point. =
        class Point

        # Default values for the fields
        x = 0
        y = 0

        # Create a new point from the coordinates
        new(x, y) =
           this.x = $(x)
           this.y = $(y)
           return $(this)

        # Move the point to the right
        move-right() =
           x = $(add $(x), 1)
           return $(this)

        # Print the point
        print() =
           println($"The point is ($(x), $(y)")

    p1 = $(Point.new 1, 5)
    p2 = $(p1.move-right)

    # Prints "The point is (1, 5)"
    p1.print()

    # Prints "The point is (2, 5)"
    p2.print()
\end{verbatim}

Note that the variable \verb+$(this)+ is used to refer to the current object.  Also, classes and
objects are \emph{functional}---the \verb+new+ and \verb+move-right+ methods return new objects.  In
this example, the object \verb+p2+ is a different object from \verb+p1+, which retains the original
$(1, 5)$ coordinates.

\section{Inheritance}
\index{inheritance}

Classes and objects support inheritance (including multiple inheritance) with the \verb+extends+
directive.  The following definition of \verb+Point3D+ defines a point with \verb+x+, \verb+y+, and
\verb+z+ fields.  The new object inherits all of the methods and fields of the parent classes/objects.

\begin{verbatim}
    Z. =
       z = 0

    Point3D. =
       extends $(Point)
       extends $(Z)
       class Point3D

       print() =
          println($"The 3D point is ($(x), $(y), $(z))")

    # The "new" method was not redefined, so this
    # defines a new point (1, 5, 0).
    p = $(Point3D.new 1, 5)
\end{verbatim}

\section{static.}
\label{section:static.}
\index{static.}

The \verb+static.+ object is used to specify values that are persistent across runs of \OMake{}.  They
are frequently used for configuring a project.  Configuring a project can be expensive, so the
\verb+static.+ object ensure that the configuration is performed just once.  In the following
(somewhat trivial) example, a \verb+static+ section is used to determine if the \LaTeX\ command is
available.  The \verb+$(where latex)+ function returns the full pathname for \verb+latex+, or
\verb+false+ if the command is not found.

\begin{verbatim}
   static. =
      LATEX_ENABLED = false
      print(--- Determining if LaTeX is installed )
      if $(where latex)
          LATEX_ENABLED = true
          export

      if $(LATEX_ENABLED)
         println($'(enabled)')
      else
         println($'(disabled)')
\end{verbatim}

The \OMake standard library provides a number of useful functions for
programming the \verb+static.+ tests, as described in
Chapter~\ref{chapter:autoconf}. Using the standard library, the above can
be rewritten as

\begin{verbatim}
   open configure/Configure
   static. =
      LATEX_ENABLED = $(CheckProg latex)
\end{verbatim}

As a matter of style, a \verb+static.+ section that is used for configuration should print what it
is doing using the \hyperfunn{ConfMsgChecking} and
\hyperfunn{ConfMsgResult} functions (of course, most of helper functions in
the standard library would do that automatically).

\subsection{.STATIC}
\label{section:.STATIC}
\index{.STATIC}
\index[target]{.STATIC}

\newinreorg

There is also a rule form of static section.  The syntax can be any of the following three forms.

\begin{verbatim}
    # Export all variables defined by the body
    .STATIC:
        <body>

    # Specify file-dependencies
    .STATIC: <dependencies>
        <body>

    # Specify which variables to export, as well as file dependencies
    .STATIC: <vars>: <dependencies>
        <body>
\end{verbatim}

The \verb+<vars>+ are the variable names to be defined, the \verb+<dependencies>+ are file
dependencies---the rule is re-evaluated if one of the dependencies is changed.  The \verb+<vars>+
and \verb+<dependencies>+ can be omitted; if so, all variables defined in the \verb+<body>+ are
exported.

For example, the final example of the previous section can also be implemented as follows.

\begin{verbatim}
    open configure/Configure
    .STATIC:
        LATEX_ENABLED = $(CheckProg latex)
\end{verbatim}
%
The effect is much the same as using \verb+static.+ (instead of \verb+.STATIC+).  However, in most
cases \verb+.STATIC+ is preferred, for two reasons.

First, a \verb+.STATIC+ section is lazy, meaning that it is not evaluated until one of its variables
is resolved.  In this example, if \verb+$(LATEX_ENABLED)+ is never evaluated, the section need never
be evaluated either.  This is in contrast to the \verb+static.+ section, which always evaluates its
body at least once.

A second reason is that a \verb+.STATIC+ section allows for file dependencies, which are useful when
the \verb+.STATIC+ section is used for memoization.  For example, suppose we wish to create a
dictionary from a table that has key-value pairs.  By using a \verb+.STATIC+ section, we can perform
this computation only when the input file changes (not on every fun of \verb+omake+).  In the
following example the \hyperfun{awk} is used to parse the file \verb+table-file+.
When a line is encountered with the form \textit{key}\verb+ = +\textit{value}, the key/value pair is
added the the \verb+TABLE+.

\begin{verbatim}
    .STATIC: table-file
        TABLE = $(Map)
        awk(table-file)
        case $'^\([[:alnum:]]+\) *= *\(.*\)'
            TABLE = $(TABLE.add $1, $2)
            export
\end{verbatim}

It is appropriate to think of a \verb+.STATIC+ section as a rule that must be recomputed whenever
the dependencies of the rule change.  The targets of the rule are the variables it exports (in this
case, the \verb+TABLE+ variable).

\subsubsection{.MEMO}
\label{section:.MEMO}
\index{.MEMO}
\index[target]{.MEMO}

A \verb+.MEMO+ rule is just like a \verb+.STATIC+ rule, except that the results are not saved
between independent runs of \verb+omake+.

\subsubsection{:key:}
\index{:key:}

The \verb+.STATIC+ and \verb+.MEMO+ rules also accept a \verb+:key:+ value, which specifies a
``key'' associated with the values being computed.  It is useful to think of a \verb+.STATIC+ rule
as a dictionary that associates keys with their values.  When a \verb+.STATIC+ rule is evaluated,
the result is saved in the table with the \verb+:key:+ defined by the rule (if a \verb+:key:+ is not
specified, a default key is used instead).  In other words, a rule is like a function.  The
\verb+:key:+ specifies the function ``argument'', and the rule body computes the result.

To illustrate, let's use a \verb+.MEMO+ rule to implement a Fibonacci function.

\begin{verbatim}
    fib(i) =
        i = $(int $i)
        .MEMO: :key: $i
            println($"Computing fib($i)...")
            result =
                if $(or $(eq $i, 0), $(eq $i, 1))
                    value $i
                else
                    add($(fib $(sub $i, 1)), $(fib $(sub $i, 2)))
        value $(result)

    println($"fib(10) = $(fib 10)")
    println($"fib(12) = $(fib 12)")
\end{verbatim}
%
When this script is run, it produces the following output.

\begin{verbatim}
    Computing fib(10)...
    Computing fib(9)...
    Computing fib(8)...
    Computing fib(7)...
    Computing fib(6)...
    Computing fib(5)...
    Computing fib(4)...
    Computing fib(3)...
    Computing fib(2)...
    Computing fib(1)...
    Computing fib(0)...
    fib(10) = 55
    Computing fib(12)...
    Computing fib(11)...
    fib(12) = 144
\end{verbatim}
%
Note that the Fibonacci computation is performed just once for each value of the argument, rather
than an exponential number of times.  In other words, the \verb+.MEMO+ rule has performed a
memoization, hence the name.  Note that if \verb+.STATIC+ were used instead, the values would be
saved across runs of \verb+omake+.

As a general guideline, whenever you use a \verb+.STATIC+ or \verb+.MEMO+ rule within a function
body, you will usually want to use a \verb+:key:+ value to index the rule by the function argument.
However, this is not required.  In the following, the \verb+.STATIC+ rule is used to perform some
expensive computation once.

\begin{verbatim}
    f(x) =
        .STATIC:
            y = $(expensive-computation)
        add($x, $y)
\end{verbatim}

Additonal care should be taken for recursive functions, like the Fibonacci function.  If the
\verb+:key:+ is omitted, then the rule would be defined in terms of itself, resulting in a cyclic
dependency.  Here is the output of the Fibonacci program with an omitted \verb+:key:+.

\begin{verbatim}
    Computing fib(10)...
    Computing fib(8)...
    Computing fib(6)...
    Computing fib(4)...
    Computing fib(2)...
    Computing fib(0)...
    fib(10) = 0
    fib(12) = 0
\end{verbatim}
%
The reason for this behavior is that the \verb+result+ value is not saved until the base case
\verb+i = 0 || i = 1+ is reached, so \verb+fib+ calls itself recursively until reaching
\verb+fib(0)+, whereupon the \verb+result+ value is fixed at 0.

In any case, recursive definitions are perfectly acceptable, but you will usually want a
\verb+:key:+ argument so that each recursive call has a different \verb+:key:+.  In most cases, this
means that the \verb+:key:+ should include all arguments to the function.

\section{Constants}
\index{constants}

Internally, OMake represents values in several forms, which we list here.

\begin{itemize}
\item int

\begin{itemize}
\item Constructor: \verb+$(int <i>)+~\ref{function:int}.
\item Object: \verb+Int+~\ref{object:Int}.
\item An integer is a represented with finite precision using the OCaml representation (31 bits on a
  32 platform, and 63 bits on a 64 bit platform).
\item See also: arithmetic~\ref{function:add}.
\end{itemize}

\item float

\begin{itemize}
\item Constructor: \verb+$(float <x>)+~\ref{function:float}.
\item Object: \verb+Float+~\ref{object:Float}.
\item A float is a floating-point value, represented in IEEE 64-bit format.
\item See also: arithmetic~\ref{function:add}.
\end{itemize}

\item array

\begin{itemize}
\item Constructor: \verb+$(array <v1>, ..., <vn>)+~\ref{function:array}.
\item Object: \verb+Array+~\ref{object:Array}.
\item An array is a finite list of values.
  Arrays are also defined with an array definition
\begin{verbatim}
    X[] =
        <v1>
        ...
        <vn>
\end{verbatim}
\item See also: \verb+nth+~\ref{function:nth}, \verb+nth-tl+~\ref{function:nth-tl},
  \verb+length+~\ref{function:length}, $\ldots$
\end{itemize}

\item string

\begin{itemize}
\item Object: \verb+String+~\ref{object:String}.
\item By default, all constant character sequences represent strings, so the simple way to construct
  a string is to write it down.  Internally, the string may be parsed as several pieces.
  A string often represents an array of values separated by whitespace.
\begin{verbatim}
    osh>S = This is a string
    - : <sequence
       "This" : Sequence
       ' ' : White
       "is" : Sequence
       ' ' : White
       "a" : Sequence
       ' ' : White
       "string" : Sequence>
       : Sequence
    osh>length($S)
    - : 4 : Int
\end{verbatim}

\item A \emph{data} string is a string where whitespace is taken literally.  It represents a single value,
  not an array.  The constructors are the quotations \verb+$"..."+ and \verb+$'...'+.

\begin{verbatim}
    osh>S = $'''This is a string'''
    - : <data "This is a string"> : String
\end{verbatim}

\item See also: Quoted strings~\ref{section:quoted-strings}.
\end{itemize}

\item file

\begin{itemize}
\item Constructor: \verb+$(file <names>)+~\ref{function:file}.
\item Object: \verb+File+~\ref{object:File}.
\item A file object represents the abstract name for a file.  The file object can be viewed as an
  absolute name; the string representation depends on the current directory.

\begin{verbatim}
    osh>name = $(file foo)
    - : /Users/jyh/projects/omake/0.9.8.x/foo : File
    osh>echo $(name)
    foo
    osh>cd ..
    - : /Users/jyh/projects/omake : Dir
    osh>echo $(name)
    0.9.8.x/foo
\end{verbatim}

\item See also: \verb+vmount+~\ref{function:vmount}.
\end{itemize}

\item directory

\begin{itemize}
\item Constructor: \verb+$(dir <names>)+~\ref{function:dir}.
\item Object: \verb+Dir+~\ref{object:Dir}.
\item A directory object is like a file object, but it represents a directory.
\end{itemize}

\item map (dictionary)
\begin{itemize}
\item Object: \verb+Map+~\ref{object:Map}.
\item A map/dictionary is a table that maps values to values.  The \verb+Map+ object is the empty
  map.  The data structure is persistent, and all operations are pure and functional.  The special syntax
  \verb+$|key|+ can be used for keys that are strings.

\begin{verbatim}
    osh>table = $(Map)
    osh>table = $(table.add x, int)
    osh>table. +=
            $|y| = int
    osh>table.find(y)
    - : "int" : Sequence
\end{verbatim}
\end{itemize}

\item channel

\begin{itemize}
\item Constructor: \verb+$(fopen <filename>, <mode>)+~\ref{function:fopen}.
\item Objects: \verb+InChannel+~\ref{object:InChannel}, \verb+OutChannel+~\ref{object:OutChannel}.
\item Channels are used for buffered input/output.
\end{itemize}

\item function

\begin{itemize}
\item Constructor: \verb+$(fun <params>, <body>)+~\ref{function:fun}.
\item Object: \verb+Fun+~\ref{object:Fun}.
\item Functions can be defined in several ways.
\begin{itemize}
\item As an anonymous function,
\begin{verbatim}
    $(fun i, j, $(add $i, $j))
\end{verbatim}
\item As a named function,
\begin{verbatim}
    f(i, j) =
        add($i, $j)
\end{verbatim}
\item
\newinkeyword
As an anonymous function argument.
\begin{verbatim}
    osh>foreach(i => $(add $i, 1), 1 2 3)
    - : <array 2 3 4> : Array
\end{verbatim}
\end{itemize}
\end{itemize}

\item lexer

\begin{itemize}
\item Object: \verb+Lexer+~\ref{object:Lexer}.
\item This object represents a lexer.
\end{itemize}

\item parser

\begin{itemize}
\item Object: \verb+Parser+~\ref{object:Parser}.
\item This object represents a parser.
\end{itemize}
\end{itemize}

% -*-
% Local Variables:
% Mode: LaTeX
% fill-column: 100
% TeX-master: "paper"
% TeX-command-default: "LaTeX/dvips Interactive"
% End:
% -*-
